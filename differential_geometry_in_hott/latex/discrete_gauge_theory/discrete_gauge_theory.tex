\documentclass[12pt]{article}
\usepackage{greg}
\usepackage[baskerville,vvarbb]{newtxmath}
\usepackage{fontspec}
\usepackage{titlesec}
\usepackage[nottoc]{tocbibind}

\usepackage{fancyvrb}
\setmainfont{Baskerville}
\setsansfont{Quadrat-Serial}
\setmonofont{Courier New}

\titleformat*{\section}{\large\sffamily}
\titleformat*{\subsection}{\normalsize\sffamily}
\titleformat*{\subsubsection}{\normalsize\sffamily}
\titleformat{\chapter}[hang]{\LARGE\sffamily}{\LARGE\thechapter}{1ex}{}[]
\titleformat{name=\chapter,numberless}[hang]{\LARGE\sffamily}{}{0ex}{}[]

\title{Discrete gauge theory in homotopy type theory}
\author{Greg Langmead}
\begin{document}

\begin{abstract}
We identify connections, curvature, and gauge transformations within the structures of homotopy type theory. Whereas most classical treatments of these structures rely entirely on infinitesimal definitions, there is an equivalent discrete story of which the infinitesimal version is a limit, analogous to the relationship between smooth paths and tangent vectors, or between de Rham and Čech cohomology. We will show how to identify the elements of discrete gauge theory, provide some evidence that this is what we have found, and use it to prove some results from the 20th century mathematics of gauge theory that depend only on homotopy types.
\end{abstract}

%\tableofcontents

\mysection{Introduction}

Following David Jaz Myers in \cite{myersgood}, we define a cover as follows:

\begin{mydef}
  A map \(\pi:E\to B\) is a \emph{cover} if it is \(\shapeo\)-étale and its fibers are sets.
\end{mydef}

Recall that \(\pi\) being \(\shapeo\)-étale means that the naturality square 
\[
\begin{CD}
E @>(-)^{\shapeo}>> \shapeo E \\
@V{\pi}VV @VV{\shapeo \pi}V \\
B @>>(-)^{\shapeo}> \shapeo B
\end{CD}
\]
is a pullback, which means among other things that each corresponding fiber of the vertical maps is equivalent.

As David proves in \cite{myersgood}, the type of \(\shapeo\)-étale maps into \(B\) is equivalent to the function type \(\shapeo B \to \Typeo\). Since a cover has the further condition that the fibers are sets this implies

\begin{mylemma}
  The type of covers over \(B\) is equivalent to \(\shapeo B \to \Type_{\shapez}\).
\end{mylemma}

In homotopy type theory we pass easily between the vertical picture (maps into \(B\)) and the horizontal picture (classifying maps of \(B\) into some type of fibers). But when we unpack this a little bit we find important classical stories. If we equip \(B\) with a basepoint \(*_B\) then a map \(f:\shapeo B \dotto \Type_{\shapez}\) is an action of the group \(\shapeo B\) on the set \(f(*_B)\), which is the fiber over \(*_B\).

Let us move up a dimension:

\begin{mydef}
  A map \(\pi:E\to B\) is a \emph{2-cover} if it is \(\shapet\)-étale and its fibers are groupoids.
\end{mydef}

\begin{mylemma}
  The type of 2-covers over \(B\) is equivalent to \(\shapet B \to \Typeo\). Further, since \(\Typeo\) is 2-truncated, the type of 2-covers of \(B\) is equivalent to \(\shape B\to \Typeo\).
\end{mylemma}

This is the type we will examine: maps from discrete \( n \)-types to the type of discrete 1-types.

For example, let \(S^1\) be the higher inductive type generated by
\begin{itemize}
\item \( \mathsf{base}:S^1 \)
\item \( \mathsf{loop}:\mathsf{base}=\mathsf{base} \)
\end{itemize}

We can deloop \(S^1\) by forming its type of torsors. This is equivalent to \(\BAutoso\defeq\sit{X:\Type}||X=S^1||_0\), the type of pairs of a type together with an equivalence class of isomorphisms with \( S^1 \) (see \cite{buchholtz2023central}). 

\begin{mylemma}Terms of \( \BAutoso \) are discrete 1-types.\end{mylemma}\proof TBD.\qed

Consider a generalization of \( S^1 \),  an \emph{\( n \)-gon} \( C_n \), generated by
\begin{itemize}
\item \( v_1,\ldots,v_n:C_n \)
\item \( e_1:v_1=v_2\)
\item \( e_2:v_2=v_3 \)
\item \( \ldots \) 
\item \(e_{n-1}:v_{n-1}=v_n\)
\item \(e_n:v_n=v_1 \)
\end{itemize}

\begin{mylemma}
  We have \( g:\pit{n:\nn}C_n\simeq S^1 \)
\end{mylemma}
\proof Generalize \cite{hottbook} Lemma 6.5.1.\qed

\( n \)-gons are all discrete 1-types:

\begin{mydef}
Let \( c:\nn\to \Typeo \) be given by \( c(n)=(C_n, ||g(n)||_0)\).
\end{mydef}

The reason we're interested in these is that they form arbitrarily fine-grained approximations to the smooth circle. We can consider an \( n \)-gon to be a circle that has a notion of ``going around \( 1/n \)th of the way''.

Now let's re-point \( \BAutoso \) at \( C_4 \). We can use a + to denote a pointed type, and we can decorate it with the base point, like so: \( \BAutosoc \). We can then refer to the underlying type (first projection from the pointed type) with \( \BAutoso_- \). This is similar to the notation \( \BAutoso_{\div} \) you find in the Symmetry book \cite{Symmetry}.

\section{Where the curved connection is}

Let \( \Bpb:\Typesp \) be a pointed discrete type and let \( f:\Bpb\dotto\BAutosoc \). If we need to reference the proof of pointedness we'll call it \( *_f:f(b)=C_4 \). 

\bibliography{connections}
\end{document}
