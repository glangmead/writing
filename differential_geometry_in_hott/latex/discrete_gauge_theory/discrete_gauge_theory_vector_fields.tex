\section{Vector fields}
\subsection{Definition}
Vector fields are sections of the tangent bundle of a manifold. We do not have a general theory of tangent bundles, even for 2-dimensional cellular types, since we cannot yet prove that connections always exist on the 1-skeleton. But \emph{given} an extension \( T \) of the \( \link \) function, we can consider the type of sections \( \pit{x:\mm_1}T_1(x) \).

\begin{mydef}
A \defemph{vector field} on a higher combinatorial 2-manifold \( \mm \) equipped with type family \( T:\mm\to\EMzo \) that extends \( \link \) is a term \( X:\pit{x:\mm_1}T_1(x) \). It may be possible to extend \( X \) to one or more faces of \( \mm \), but we call faces for which \( X \) cannot be extended are called \defemph{zeros of the vector field}.\todo{prove that the index is an obstruction}
\end{mydef}

\begin{mynote}
A section \( \pit{x:\mm}f(x) \) for \( f:\mm\to\Kzt \) is a trivialization of the bundle. The fact that an orientation suffices to factor the tangent bundle of a 2-manifold (which \emph{a priori} maps to \( \EMzo \)) through a principal bundle classifier is special to dimension 2. For higher dimensional manifolds the tangent bundle is a bundle of spheres, and even if the bundle is oriented it is not necessarily a principal bundle. On the other hand, the \( n \)-truncation modal operator maps the type of \( n \)-spheres to the classifying space \( \K(\zz,n) \), and so another way to phrase this remark is that \( S^1 \) is the only \( n \)-sphere which is \( n \)-truncated.
\end{mynote}

On the 0-skeleton \( X \) picks a point in each link, i.e. a neighbor of each vertex. On a path \( p:x=_\mm y \), \( X \) assigns a dependent path over \( p \), which as we know is a term \( \pi:\tr(p)(X(x))=_{T_1 y} X(y) \).

\begin{mynote}Concrete pathover terms such as \( \pi \) couple together the connection (the transport) and \( X \). The transport is needed to provide a single type in which to compare the two values of \( X \). The path \( \pi \) therefore reflects both the movement of \( X \) along the path \( p \), and the moving lens that allows us to examine in a single fiber the two points \( X(x) \) and \( X(y) \) as well as \( X(p) \).\end{mynote}

\subsection{Swirling}
Classically, vector fields are seen to swirl around between points. We can see this in HoTT as well, though in contrast to the classical explanations we always have a connection that interacts with the vector field. 

Consider a vertex \( x:\mm \), a face \( F \) containing vertices \( x, y, z \), and the boundary path \( \ell\defeq e_{xy}\cdot e_{yz} \cdot e_{zx} \). We can calculate \( X(\ell) \) by concatenating the data along the path. As we visit each point we accumulate more information:

\begin{enumerate}
\item In \( T_1(x) \) we have 
\begin{enumerate}
\item the point \( X(x) \).
\end{enumerate}
\item In \( T_1(y) \) we have
\begin{enumerate}
\item point \( X(y) \)
\item point \( \tr(e_{xy})(X(x)) \)
\item the path \( X(e_{xy}):\tr(e_{xy})(X(x))=X(y) \) from (b) to (a).
\end{enumerate}
\item In \( T_1(z) \) we have 
\begin{enumerate}
\item point \( X(z) \)
\item point \( \tr(e_{yz})(X(y)) \)
\item point \( \tr(e_{yz})\circ\tr(e_{xy})(X(x)) \)
\item path \( X(e_{yz}):\tr(e_{yz})(X(y))=X(z) \) from (b) to (a)
\item path \( \tr(e_{yz})(X(e_{xy})) \) from (c) to (b)
\end{enumerate}
\item Then back at \( T_1(x) \) we have
\begin{enumerate}
\item point \( X(x) \)
\item point \( \tr(e_{zx})(X(z)) \)
\item point \( \tr(e_{zx})\circ\tr(e_{yz})(X(y)) \)
\item point \( \tr(e_{zx})\circ\tr(e_{yz})\circ\tr(e_{xy})(X(x)) \), i.e. \( \tr(\ell)(X(x)) \)
\item path \( X(e_{zx}): \tr(e_{zx})(X(z))=X(x) \) from (b) to (a)
\item path \( \tr(e_{zx})(X(e_{yz})) \) from (c) to (b)
\item path \( \tr(e_{zx})\circ\tr(e_{yz})(X(e_{xy})) \) from (d) to (c).
\end{enumerate}
\end{enumerate}

As we traverse an edge, say \( e_{xy} \), we get a path in \( T_1(y) \) which is the image of \( e_{xy} \) under \( X \). Although this path is then transported along further edges of the triangle, it will help us to think of it as maintaining its relationship to \( e_{xy} \):

\begin{mydef}
If \( e_{xy}:x=_{\mm_1} y, e_{yz}:y=_{\mm_1}z \) are paths in \( \mm_1 \) and \( X \) is a nonvanishing vector field on \( \mm_1 \), then we call \( \tr(e_{yz})(X(e_{xy})) \) \defemph{the contribution on \( e_{xy} \) of \( X \)}, even though this is a path in \( T_1(z) \).
\end{mydef}

As we traverse an additional edge this image is simply mapped to the next vertex through the lens of transport, which acts similarly on all the points in a fiber. The image is carried first to \( \tr(e_{yz})(X(e_{xy})) \) then to \( \tr(e_{zx})\circ\tr(e_{yz})(X(x)) \).

At stage 4 we have three paths, each consisting of \( X \) acting on a single edge of the triangle, mapped with some transport to \( T_1(x) \). We see that these three can be concatenated to form a path \( X(\ell):\tr(\ell)(X(x))=X(x) \).

\begin{figure}[h]
\centering
\begin{tikzpicture}
  [arrow/.style={-{Stealth[scale=1.1]}}, vec/.style={ultra thick, color=black}, vectr/.style={thick, color=black}, vectrtr/.style={thick, dashed, color=black}, vectrtrtr/.style={thick, dotted, color=black}]
  \tikzset{oo/.style={circle, scale=0.6, fill=black}}
  \tikzset{ii/.style={circle, scale=0.3, fill=gray}}
\newlength{\mylen}
  \setlength{\mylen}{3cm}
  \newlength{\mylin}
  \setlength{\mylin}{1.2cm}
    \node[oo, label=below right:\( v_1 \)] (V1) at (0, 0) {};
    \node[oo, label=below:\( v_2 \)] (V2) at (2*\mylen, 0) {};
    \node[oo, label=above:\( v_3 \)] (V3) at (\mylen, 1.732*\mylen) {};

    \draw[arrow] (V2) edge[very thick, color=teal, "\( e_{23} \)"] (V3);
    \draw[arrow] (V1) edge[very thick, color=magenta, "\( e_{12} \)"] (V2);
    \draw[arrow] (V3) edge[very thick, color=blue, "\( e_{31} \)"] (V1);
    
    \node [ii, above=\mylin of V1.north, label=above:\( v_{11} \)] (V11) {};
    \node [ii, right=\mylin of V1.east,  label=below right:\( v_{12} \)] (V12) {};
    \node [ii, below=\mylin of V1.south, label=below:\( v_{13} \)] (V13) {};
    \node [ii, left= \mylin of V1.west,  label=left: \( v_{14} \)] (V14) {};

    \node [ii, above=\mylin of V2.north, label=above:\( v_{21} \)] (V21) {};
    \node [ii, right=\mylin of V2.east,  label=right:\( v_{22} \)] (V22) {};
    \node [ii, below=\mylin of V2.south, label=below:\( v_{23} \)] (V23) {};
    \node [ii, left= \mylin of V2.west,  label=below left: \( v_{24} \)] (V24) {};

    \node [ii, above=\mylin of V3.north, label=above:\( v_{31} \)] (V31) {};
    \node [ii, right=\mylin of V3.east,  label=right:\( v_{32} \)] (V32) {};
    \node [ii, below=\mylin of V3.south, label=below:\( v_{33} \)] (V33) {};
    \node [ii, left= \mylin of V3.west,  label=left: \( v_{34} \)] (V34) {};

    \draw[dashed] (V11) -- (V12);
    \draw[dashed] (V12) -- (V13);
    \draw[dashed] (V13) -- (V14);
    \draw[dashed] (V14) -- (V11);

    \draw[dashed] (V21) -- (V22);
    \draw[dashed] (V22) -- (V23);
    \draw[dashed] (V23) -- (V24);
    \draw[dashed] (V24) -- (V21);
    
    \draw[dashed] (V31) -- (V32);
    \draw[dashed] (V32) -- (V33);
    \draw[dashed] (V33) -- (V34);
    \draw[dashed] (V34) -- (V31);
    
    \draw[arrow] (V1) edge[vec] (V11);
    \draw[arrow] (V2) edge[vectr] (V21);
    \draw[arrow] (V3) edge[vectrtr] (V34);
    \draw[arrow] (V1) edge[vectrtrtr] (V14);

    \draw[arrow] (V2) edge[vec] (V24);
    \draw[arrow] (V3) edge[vectr] (V33);
    \draw[arrow] (V1) edge[vectrtr] (V13);

    \draw[arrow] (V21) edge[thick, color=magenta] (V24);
    \draw[arrow] (V34) edge[thick, color=magenta] (V33);
    \draw[arrow] (V14) edge[thick, color=magenta] (V13);
    \draw[arrow] (V33) edge[thick, color=cyan] (V32);
    \draw[arrow] (V13) edge[thick, color=cyan] (V12);
    \draw[arrow] (V12) edge[thick, color=blue] (V11);

    \draw[arrow] (V3) edge[vec] (V32);
    \draw[arrow] (V1) edge[vectr] (V12);

\end{tikzpicture}
\caption{A vector field swirling around a face, in a bundle of squares. Thick black vectors are the value at the point. Thin vectors are transported once, dashed twice, and dotted three times. The vertices \( v_{ij} \) are in the tangent fibers. The colors of the arrows correspond: the red edge produced the red edge in the fibers.}
\label{fig:swirling}
\end{figure}
\chcomment[id=Greg]{new figure, needs a little more explanation}

\begin{mynote}
In Hopf \cite{hopf} and in Needham \cite{needham}, the value \( X(e_{xy}) \) is called ``the change in angle between \( X \) and \( X(x)_{||} \) along the edge \( e_{xy} \),'' where by \( X(x)_{||} \) we mean the transport of the fixed single vector \( X(x) \) to the point \( X(y) \), i.e. \( \tr(e_{xy})(X(x)) \). The concatenation \( X(\ell) \) is called ``the sum of the changes in angle along each edge.'' This remark is meant to help anyone intending to make further comparisons with classical results.
\end{mynote}
