\section{Vector fields}
\label{sec:vector_fields}
\subsection{Definition}
Vector fields are sections of the tangent bundle of a manifold. We do not have a general theory of tangent bundles, even for 2-dimensional cellular types, since we cannot yet prove that connections always exist on the 1-skeleton. But \emph{given} an extension \( T \) of the \( \link \) function, we can consider the type of sections \( \pit{x:\mm_1}T_1(x) \).

\begin{mydef}
A \defemph{vector field} on a higher combinatorial 2-manifold \( \mm \) equipped with type family \( T:\mm\to\EMzo \) that extends \( \link \) is a term \( X:\pit{x:\mm_1}T_1(x) \). It may be possible to extend \( X \) to one or more faces of \( \mm \), but we call faces for which \( X \) cannot be extended \defemph{zeros of the vector field}.\todo{prove that the index is an obstruction}
\end{mydef}

\begin{mynote}
A section \( \pit{x:\mm}T(x) \) for \( T:\mm\to\Kzt \) is a trivialization of the bundle. The fact that an orientation suffices to factor the tangent bundle of a 2-manifold (which \emph{a priori} maps to \( \EMzo \)) through a principal bundle classifier is special to dimension 2. For higher dimensional manifolds the tangent bundle is a bundle of spheres, and even if the bundle is oriented it is not necessarily a principal bundle. On the other hand, the \( n \)-truncation modal operator maps the type of \( n \)-spheres to the classifying space \( \K(\zz,n) \), and so another way to phrase this remark is that \( S^1 \) is the only \( n \)-sphere which is \( n \)-truncated.
\end{mynote}

On the 0-skeleton \( X \) picks a point in each link, i.e. a neighbor of each vertex. On a path \( p:x=_\mm y \), \( X \) assigns a dependent path over \( p \), which as we know is a term \( \pi:\tr(p)(X(x))=_{T y} X(y) \). We are very interested in working with the concatenation operation on dependent paths, which we call \emph{swirling}.

\subsection{Swirling}
Consider the vertex \( v_1:\mm \), a face \( F \) containing vertices \( v_1, v_2, v_3 \), and the boundary path \( \ell\defeq e_{12}\cdot e_{23} \cdot e_{31} \). For brevity, denote \( T(v_i) \) by \( T_i \) and \( T(e_{ij}) \) by \( T_{ji} \). The indices are swapped so that we can have expressions that respect function-composition order, such as \( T_{32}T_{21}(X_1):T_3 \). Figure~\ref{fig:vector_transport} shows in tabular form how we concatenate the dependent paths over \( e_{12}\cdot e_{23}\cdot e_{31} \). Figure~\ref{fig:swirling} shows visually a possible example.

\begin{figure}[h]
\centering
% https://q.uiver.app/#q=WzAsMTQsWzAsMSwiWF8xIl0sWzEsMSwiWF8yIl0sWzIsMSwiWF8zIl0sWzEsMiwiVF97MjF9WF8xIl0sWzIsMiwiVF97MzJ9WF8yIl0sWzIsMywiVF97MzJ9VF97MjF9WF8xIl0sWzMsMSwiWF8xIl0sWzMsMiwiVF97MTN9WF8zIl0sWzMsMywiVF97MTN9VF97MzJ9WF8yIl0sWzMsNCwiVF97MTN9VF97MzJ9VF97MjF9WF8xIl0sWzAsMCwiVF8xIl0sWzEsMCwiVF8yIl0sWzIsMCwiVF8zIl0sWzMsMCwiVF8xIl0sWzEwLDExLCJUX3syMX0iXSxbMTEsMTIsIlRfezMyfSJdLFsxMiwxMywiVF97MTN9Il0sWzEsMywiWF97MTJ9OiIsMix7ImxldmVsIjoyLCJzdHlsZSI6eyJoZWFkIjp7Im5hbWUiOiJub25lIn19fV0sWzIsNCwiWF97MjN9OiIsMix7ImxldmVsIjoyLCJzdHlsZSI6eyJoZWFkIjp7Im5hbWUiOiJub25lIn19fV0sWzYsNywiWF97MzF9OiIsMix7ImxldmVsIjoyLCJzdHlsZSI6eyJoZWFkIjp7Im5hbWUiOiJub25lIn19fV0sWzQsNSwiVF97MzJ9WF97MTJ9OiIsMix7ImxldmVsIjoyLCJzdHlsZSI6eyJoZWFkIjp7Im5hbWUiOiJub25lIn19fV0sWzcsOCwiVF97MTN9WF97MjN9OiIsMix7ImxldmVsIjoyLCJzdHlsZSI6eyJoZWFkIjp7Im5hbWUiOiJub25lIn19fV0sWzgsOSwiVF97MTN9VF97MzJ9WF97MTJ9OiIsMix7ImxldmVsIjoyLCJzdHlsZSI6eyJoZWFkIjp7Im5hbWUiOiJub25lIn19fV1d
\begin{tikzcd}
  {T_1} & {T_2} & {T_3} & {T_1} \\
  {X_1} & {X_2} & {X_3} & {X_1} \\
  & {T_{21}X_1} & {T_{32}X_2} & {T_{13}X_3} \\
  && {T_{32}T_{21}X_1} & {T_{13}T_{32}X_2} \\
  &&& {T_{13}T_{32}T_{21}X_1}
  \arrow["{T_{21}}", from=1-1, to=1-2]
  \arrow["{T_{32}}", from=1-2, to=1-3]
  \arrow["{T_{13}}", from=1-3, to=1-4]
  \arrow["{X_{12}:}"', equals, from=2-2, to=3-2]
  \arrow["{X_{23}:}"', equals, from=2-3, to=3-3]
  \arrow["{X_{31}:}"', equals, from=2-4, to=3-4]
  \arrow["{T_{32}X_{12}:}"', equals, from=3-3, to=4-3]
  \arrow["{T_{13}X_{23}:}"', equals, from=3-4, to=4-4]
  \arrow["{T_{13}T_{32}X_{12}:}"', equals, from=4-4, to=5-4]
\end{tikzcd}
\caption{The data in each fiber as we move around a triangle with vertices indexed 1, 2, and 3. Double-lines indicate identity types between two types, and their labels are terms of this type. Items with one index are terms of some type at a vertex, and items with two indices are terms of a type on an edge.}
\label{fig:vector_transport}
\end{figure}

As we traverse an edge, say \( e_{12} \), we get a path in \( T_2 \) which is the image of \( e_{12} \) under \( X \), denoted \( X_{12} \). As we traverse an additional edge, \( X_{12} \) is simply mapped to the next vertex by transport. The image is carried first to \( T_{32}(X_{12}) \) then to \( T_{13}\circ T_{32}(X_{12}) \).

\begin{figure}[h]
\centering
\begin{tikzpicture}
  [arrow/.style={-{Stealth[scale=1.1]}}, vec/.style={ultra thick, color=black}, vectr/.style={thick, color=black}, vectrtr/.style={thick, dashed, color=black}, vectrtrtr/.style={thick, dotted, color=black}]
  \tikzset{oo/.style={circle, scale=0.6, fill=black}}
  \tikzset{ii/.style={circle, scale=0.3, fill=gray}}
\newlength{\mylen}
  \setlength{\mylen}{3cm}
  \newlength{\mylin}
  \setlength{\mylin}{1.2cm}
    \node[oo, label=below right:\( v_1 \)] (V1) at (0, 0) {};
    \node[oo, label=below:\( v_2 \)] (V2) at (2*\mylen, 0) {};
    \node[oo, label=above:\( v_3 \)] (V3) at (\mylen, 1.732*\mylen) {};

    \draw[arrow] (V2) edge[very thick, color=teal, "\( e_{23} \)"] (V3);
    \draw[arrow] (V1) edge[very thick, color=magenta, "\( e_{12} \)"] (V2);
    \draw[arrow] (V3) edge[very thick, color=blue, "\( e_{31} \)"] (V1);
    
    \node [ii, above=\mylin of V1.north, label=above:\( v_{11} \)] (V11) {};
    \node [ii, right=\mylin of V1.east,  label=below right:\( v_{12} \)] (V12) {};
    \node [ii, below=\mylin of V1.south, label=below:\( v_{13} \)] (V13) {};
    \node [ii, left= \mylin of V1.west,  label=left: \( v_{14} \)] (V14) {};

    \node [ii, above=\mylin of V2.north, label=above:\( v_{21} \)] (V21) {};
    \node [ii, right=\mylin of V2.east,  label=right:\( v_{22} \)] (V22) {};
    \node [ii, below=\mylin of V2.south, label=below:\( v_{23} \)] (V23) {};
    \node [ii, left= \mylin of V2.west,  label=below left: \( v_{24} \)] (V24) {};

    \node [ii, above=\mylin of V3.north, label=above:\( v_{31} \)] (V31) {};
    \node [ii, right=\mylin of V3.east,  label=right:\( v_{32} \)] (V32) {};
    \node [ii, below=\mylin of V3.south, label=below:\( v_{33} \)] (V33) {};
    \node [ii, left= \mylin of V3.west,  label=left: \( v_{34} \)] (V34) {};

    \draw[dashed] (V11) -- (V12);
    \draw[dashed] (V12) -- (V13);
    \draw[dashed] (V13) -- (V14);
    \draw[dashed] (V14) -- (V11);

    \draw[dashed] (V21) -- (V22);
    \draw[dashed] (V22) -- (V23);
    \draw[dashed] (V23) -- (V24);
    \draw[dashed] (V24) -- (V21);
    
    \draw[dashed] (V31) -- (V32);
    \draw[dashed] (V32) -- (V33);
    \draw[dashed] (V33) -- (V34);
    \draw[dashed] (V34) -- (V31);
    
    \draw[arrow] (V1) edge[vec] (V11);
    \draw[arrow] (V2) edge[vectr] (V21);
    \draw[arrow] (V3) edge[vectrtr] (V34);
    \draw[arrow] (V1) edge[vectrtrtr] (V14);

    \draw[arrow] (V2) edge[vec] (V24);
    \draw[arrow] (V3) edge[vectr] (V33);
    \draw[arrow] (V1) edge[vectrtr] (V13);

    \draw[arrow] (V21) edge[thick, color=magenta] (V24);
    \draw[arrow] (V34) edge[thick, color=magenta] (V33);
    \draw[arrow] (V14) edge[thick, color=magenta] (V13);
    \draw[arrow] (V33) edge[thick, color=teal] (V32);
    \draw[arrow] (V13) edge[thick, color=teal] (V12);
    \draw[arrow] (V12) edge[thick, color=blue] (V11);

    \draw[arrow] (V3) edge[vec] (V32);
    \draw[arrow] (V1) edge[vectr] (V12);

\end{tikzpicture}
\caption{A vector field swirling around a face, in a bundle of squares. Imagine that transport along \( e_{12} \) does not rotate along the page, that transport along \( e_{23} \) rotates counterclockwise by 90 degrees, and that transport along \( e_{31} \) again does not rotate along the page. Thick black vectors are the vector field at a point. Thin vectors are transported once, dashed twice, and dotted three times. The vertices \( v_{ij} \) are in the tangent fibers. If you see color, the colors of the arrows correspond: the red edge produced the red edge in the fibers.}
\label{fig:swirling}
\end{figure}

We wish to simplify expressions such as \( T_{13}T_{32}X_{12}\cdot T_{13}X_{23}\cdot X_{31} \), which take place in a particular fiber (\( T_1 \) in this case), and which depend on arranging for the endpoint of one segment to agree with the start of another. The simplification will empower us to easily perform calculations over the whole manifold, and to prove our main theorem~\ref{thm:total_index_total_curvature}.

First we will choose a specific group that acts on all the fibers of \( T \). Fix a point \( m:\mm \), and recall the map \( \tau:\pit{x:M}((Tx=Tx)\xrightarrow[]{\simeq}(Tm=Tm)) \) from Lemma~\ref{lem:gauge_triv}. This map is a trivialization of the automorphism bundle, so in particular provides an action of \( Tm=Tm \) on each \( Tx \) (or in our case \( T_i \)). Denote the action of \( Tm=Tm \) on the torsor \( T_i \) by \( \alpha:(Tm=Tm)\times T_i\to T_i \). This induces an equivalence \( (\alpha, \pr_2):(Tm=Tm)\times T_i\simto T_i\times T_i \).

\begin{mydef}
The map \( \pr_1\circ (\alpha, \pr_2)^{-1}:T_i\times T_i\to (Tm=Tm) \) is called \defemph{subtraction}. It maps \( (x,y) \) to the unique term \( \delta:Tm=Tm \) such that \( \alpha(\delta, x)=y \). For brevity we denote \( \pr_1\circ (\alpha, \pr_2)^{-1}(x,y) \) by \( y-x \).
\end{mydef}

Each fiber \( T_i \) is pointed by \( X_i \), so we can define the map \( T_i\xrightarrow[]{-X_i}(Tm=Tm) \), and then give a name to the special term \( \rho_{ji}\defeq T_{ji}X_j - X_i \) which is the image in \( Tm=Tm \) of the vector \( X_j \) in a neighboring vertex after transporting it to \( T_i \).

\begin{mylemma} The following diagram is well-typed and commutes, and therefore \( \rho_{ij}=-\rho_{ji} \).
\label{lem:subtraction}\end{mylemma}
\begin{center}
% https://q.uiver.app/#q=WzAsNCxbMCwwLCIoWF9qPV97VF9qfVRfe2ppfVhfaSlcXHRpbWVzKFhfaT1fe1RfaX1UX3tpan1YX2opIl0sWzAsMSwiKFxcbWF0aHJte2lkfT1fe1RtPVRtfVxccmhvX3tqaX0pXFx0aW1lcyhcXG1hdGhybXtpZH09X3tUbT1UbX1cXHJob197aWp9KSJdLFsxLDEsIlxcbWF0aHJte2lkfT1fe1RtPVRtfVxccmhvX3tpan0rXFxyaG9fe2ppfSJdLFsxLDAsIlhfaj1fe1Rfan1YX2oiXSxbMSwyLCIoXFx0ZXh0ey19KSsoXFxyaG9fe2ppfSsoXFx0ZXh0ey19KSkiLDJdLFswLDMsIihcXHRleHR7LX0pXFxjZG90IFRfe2ppfShcXHRleHR7LX0pIl0sWzMsMiwiLVhfaiJdLFswLDEsIi1YX2pcXHRpbWVzIC1YX2kiLDJdXQ==
\begin{tikzcd}
  {(X_j=_{T_j}T_{ji}X_i)\times(X_i=_{T_i}T_{ij}X_j)} & {X_j=_{T_j}X_j} \\
  {(\mathrm{id}=_{Tm=Tm}\rho_{ji})\times(\mathrm{id}=_{Tm=Tm}\rho_{ij})} & {\mathrm{id}=_{Tm=Tm}\rho_{ij}+\rho_{ji}}
  \arrow["{(\text{-})\cdot T_{ji}(\text{-})}", from=1-1, to=1-2]
  \arrow["{-X_j\times -X_i}"', from=1-1, to=2-1]
  \arrow["{-X_j}", from=1-2, to=2-2]
  \arrow["{(\text{-})+(\rho_{ji}+(\text{-}))}"', from=2-1, to=2-2]
\end{tikzcd}
\end{center}
\begin{proof}
The top map has the given codomain because \( T_{ji}T_{ij}=\id \). The upper-right composition is \( ((\text{-})\cdot(T_{ji}(\text{-}))-X_j \), and the lower-left composition is \( ((\text{-})-X_j)+(\rho_{ji}+((\text{-})-X_i)) \), so commutativity follows from the definition of \( \rho_{ji} \). The final statement follows from comparing the types on the right.
\end{proof}

We now know that the type in the bottom-right is \( \id=_{Tm=Tm}\id \), so the vertical map \( -X_j \) maps the path \( X_{ij}\cdot X_{ji} \) to a path in \( \id=_{Tm=Tm}\id \) and we can ask if the image is \( \refl \).

\begin{mylemma}
\( (X_{ij}\cdot X_{ji})-X_j=\refl \).
\end{mylemma}
\begin{proof}
We have
\begin{alignat*}{2}
-X_j(X_{ij}\cdot X_{ji}) &= -X_j(X(e_{ji}\cdot e_{ij}))&&\quad\text{(by definition of dependent paths)}\\
&= -X_j(X(\refl))&&\quad\text{}\\
&= -X_j(\refl)&&\quad\text{(by path induction)}\\
&= \refl&&\quad\text{}
\end{alignat*}
\end{proof}

\begin{mylemma}The group operation along the bottom of Lemma~\ref{lem:subtraction} is abelian, and the vertical maps form a homomorphism from the groupoid of dependent paths to the commutative group of paths in \( Tm=Tm \).
\end{mylemma}
\begin{proof}
Not entirely sure what remains here.\todo{By induction from 5.4, building up a long concat of dep. paths.}
\end{proof}

\subsection{An example vector field on the sphere}
\todo{define the example I worked out, compute swirling}
