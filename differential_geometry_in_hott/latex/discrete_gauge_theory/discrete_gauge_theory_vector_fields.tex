\section{Vector fields}

\begin{mydef}
A \defemph{partial function} \( f:A\to B \) is a function \( f:A\to B+\star \), the disjoint union of \( B \) with the 1-element type.
\end{mydef}

If \( T:\mm\to\EMzo \) is a bundle of mere circles, then a vector field should be a partial function \( T_\bullet:\mm\to\EMpzo \) that lifts \( T \). In other words, a pointing of \emph{some} of the fibers. This aligns with the classical picture of a choice of nonzero vector at each point, except for some points where the vector field vanishes. So instead of having a notion corresponding to the full tangent vector space (one candidate for which would be the disk at each point, i.e. \( \link \) plus its spokes and filler triangles) we are mapping some vertices to their circular fibers, and others to \( \star \). This lets us continue to work with \( \EMzo \) instead of a type of tangent spaces.

Figure~\ref{fig:flattorus_zero} illustrates what removing the preimage of \( \star \) looks like. The resulting type is no longer a combinatorial manifold, since it fails the condition about every point having a circular link. 

\begin{figure}[htbp]
\centering
% https://q.uiver.app/#q=WzAsMzUsWzEsNCwiXFxidWxsZXQiXSxbMSw2LCJcXGJ1bGxldCJdLFsyLDcsIlxcYnVsbGV0Il0sWzIsNSwiXFxidWxsZXQiXSxbMiwzLCJcXGJ1bGxldCJdLFsxLDgsIlxcYnVsbGV0Il0sWzEsMiwiXFxidWxsZXQiXSxbMywyLCJcXGJ1bGxldCJdLFszLDQsIlxcYnVsbGV0Il0sWzMsOCwiXFxidWxsZXQiXSxbMiwxLCJcXGJ1bGxldCJdLFs0LDEsIlxcYnVsbGV0Il0sWzUsMiwiXFxidWxsZXQiXSxbNCwzLCJcXGJ1bGxldCJdLFs1LDQsIlxcYnVsbGV0Il0sWzQsNSwiXFxidWxsZXQiXSxbNSw2LCJcXGJ1bGxldCJdLFs0LDcsIlxcYnVsbGV0Il0sWzUsOCwiXFxidWxsZXQiXSxbMiwwLCJcXG1hdGhybXtiYWNrfSJdLFs0LDAsIlxcbWF0aHJte2Zyb250fSJdLFsxLDAsIlxcbWF0aHJte291dGVyfSJdLFszLDAsIlxcbWF0aHJte2hvbGV9Il0sWzUsMCwiXFxtYXRocm17b3V0ZXJ9Il0sWzEsMTAsIlxcYnVsbGV0Il0sWzIsOSwiXFxidWxsZXQiXSxbMywxMCwiXFxidWxsZXQiXSxbNCw5LCJcXGJ1bGxldCJdLFs1LDEwLCJcXGJ1bGxldCJdLFswLDIsIlxcbWF0aHJte3RvcFxcIG9mXFwgZGlhbW9uZHN9Il0sWzAsNiwiXFxtYXRocm17Ym90dG9tXFwgb2ZcXCBkaWFtb25kc30iXSxbMSwxLCJcXCAiXSxbNSwxLCJcXCAiXSxbMywxLCJcXCAiXSxbMCwxMCwiXFxtYXRocm17dG9wXFwgb2ZcXCBkaWFtb25kc30iXSxbNCwzLCIiLDIseyJzdHlsZSI6eyJoZWFkIjp7Im5hbWUiOiJub25lIn19fV0sWzMsMiwiIiwyLHsic3R5bGUiOnsiaGVhZCI6eyJuYW1lIjoibm9uZSJ9fX1dLFswLDEsIiIsMix7InN0eWxlIjp7ImhlYWQiOnsibmFtZSI6Im5vbmUifX19XSxbNCwwLCIiLDEseyJjb2xvdXIiOlsyMzYsOTEsNjBdLCJzdHlsZSI6eyJoZWFkIjp7Im5hbWUiOiJub25lIn19fV0sWzAsMywiIiwxLHsiY29sb3VyIjpbMjM2LDkxLDYwXSwic3R5bGUiOnsiaGVhZCI6eyJuYW1lIjoibm9uZSJ9fX1dLFszLDEsIiIsMSx7ImNvbG91ciI6WzIzNiw5MSw2MF0sInN0eWxlIjp7ImhlYWQiOnsibmFtZSI6Im5vbmUifX19XSxbMSwyLCIiLDEseyJjb2xvdXIiOlsyMzYsOTEsNjBdLCJzdHlsZSI6eyJoZWFkIjp7Im5hbWUiOiJub25lIn19fV0sWzIsNSwiIiwxLHsiY29sb3VyIjpbMjM2LDkxLDYwXSwic3R5bGUiOnsiaGVhZCI6eyJuYW1lIjoibm9uZSJ9fX1dLFsxLDUsIiIsMSx7InN0eWxlIjp7ImhlYWQiOnsibmFtZSI6Im5vbmUifX19XSxbNiw0LCIiLDEseyJjb2xvdXIiOlsyMzYsOTEsNjBdLCJzdHlsZSI6eyJoZWFkIjp7Im5hbWUiOiJub25lIn19fV0sWzYsMCwiIiwxLHsic3R5bGUiOnsiaGVhZCI6eyJuYW1lIjoibm9uZSJ9fX1dLFs3LDgsIiIsMCx7InN0eWxlIjp7ImhlYWQiOnsibmFtZSI6Im5vbmUifX19XSxbNyw0LCIiLDAseyJjb2xvdXIiOlsyMzYsOTEsNjBdLCJzdHlsZSI6eyJoZWFkIjp7Im5hbWUiOiJub25lIn19fV0sWzQsOCwiIiwwLHsiY29sb3VyIjpbMjM2LDkxLDYwXSwic3R5bGUiOnsiaGVhZCI6eyJuYW1lIjoibm9uZSJ9fX1dLFs4LDMsIiIsMCx7ImNvbG91ciI6WzIzNiw5MSw2MF0sInN0eWxlIjp7ImhlYWQiOnsibmFtZSI6Im5vbmUifX19XSxbMiw5LCIiLDAseyJjb2xvdXIiOlsyMzYsOTEsNjBdLCJzdHlsZSI6eyJoZWFkIjp7Im5hbWUiOiJub25lIn19fV0sWzEwLDQsIiIsMCx7InN0eWxlIjp7ImhlYWQiOnsibmFtZSI6Im5vbmUifX19XSxbMTAsNiwiIiwwLHsiY29sb3VyIjpbMjM2LDkxLDYwXSwic3R5bGUiOnsiaGVhZCI6eyJuYW1lIjoibm9uZSJ9fX1dLFsxMCw3LCIiLDAseyJjb2xvdXIiOlsyMzYsOTEsNjBdLCJzdHlsZSI6eyJoZWFkIjp7Im5hbWUiOiJub25lIn19fV0sWzE4LDE2LCIiLDAseyJzdHlsZSI6eyJoZWFkIjp7Im5hbWUiOiJub25lIn19fV0sWzE3LDE1LCIiLDAseyJzdHlsZSI6eyJoZWFkIjp7Im5hbWUiOiJub25lIn19fV0sWzE1LDEzLCIiLDAseyJzdHlsZSI6eyJoZWFkIjp7Im5hbWUiOiJub25lIn19fV0sWzE2LDE0LCIiLDAseyJzdHlsZSI6eyJoZWFkIjp7Im5hbWUiOiJub25lIn19fV0sWzE0LDEyLCIiLDAseyJzdHlsZSI6eyJoZWFkIjp7Im5hbWUiOiJub25lIn19fV0sWzEzLDExLCIiLDAseyJzdHlsZSI6eyJoZWFkIjp7Im5hbWUiOiJub25lIn19fV0sWzExLDcsIiIsMCx7ImNvbG91ciI6WzEsOTUsNjBdLCJzdHlsZSI6eyJoZWFkIjp7Im5hbWUiOiJub25lIn19fV0sWzExLDEyLCIiLDAseyJjb2xvdXIiOlsxLDk1LDYwXSwic3R5bGUiOnsiaGVhZCI6eyJuYW1lIjoibm9uZSJ9fX1dLFsxMiwxMywiIiwwLHsiY29sb3VyIjpbMSw5NSw2MF0sInN0eWxlIjp7ImhlYWQiOnsibmFtZSI6Im5vbmUifX19XSxbMTMsOCwiIiwwLHsiY29sb3VyIjpbMSw5NSw2MF0sInN0eWxlIjp7ImhlYWQiOnsibmFtZSI6Im5vbmUifX19XSxbMTMsMTQsIiIsMCx7ImNvbG91ciI6WzEsOTUsNjBdLCJzdHlsZSI6eyJoZWFkIjp7Im5hbWUiOiJub25lIn19fV0sWzE0LDE1LCIiLDAseyJjb2xvdXIiOlsxLDk1LDYwXSwic3R5bGUiOnsiaGVhZCI6eyJuYW1lIjoibm9uZSJ9fX1dLFs4LDE1LCIiLDAseyJjb2xvdXIiOlsxLDk1LDYwXSwic3R5bGUiOnsiaGVhZCI6eyJuYW1lIjoibm9uZSJ9fX1dLFsxNSwxNiwiIiwwLHsiY29sb3VyIjpbMSw5NSw2MF0sInN0eWxlIjp7ImhlYWQiOnsibmFtZSI6Im5vbmUifX19XSxbMTYsMTcsIiIsMCx7ImNvbG91ciI6WzEsOTUsNjBdLCJzdHlsZSI6eyJoZWFkIjp7Im5hbWUiOiJub25lIn19fV0sWzE3LDE4LCIiLDAseyJjb2xvdXIiOlsxLDk1LDYwXSwic3R5bGUiOnsiaGVhZCI6eyJuYW1lIjoibm9uZSJ9fX1dLFsxNyw5LCIiLDAseyJjb2xvdXIiOlsxLDk1LDYwXSwic3R5bGUiOnsiaGVhZCI6eyJuYW1lIjoibm9uZSJ9fX1dLFs3LDEzLCIiLDAseyJjb2xvdXIiOlsxLDk1LDYwXSwic3R5bGUiOnsiaGVhZCI6eyJuYW1lIjoibm9uZSJ9fX1dLFs1LDI1LCIiLDEseyJjb2xvdXIiOlsyMzYsOTEsNjBdLCJzdHlsZSI6eyJoZWFkIjp7Im5hbWUiOiJub25lIn19fV0sWzksMjUsIiIsMSx7ImNvbG91ciI6WzIzNiw5MSw2MF0sInN0eWxlIjp7ImhlYWQiOnsibmFtZSI6Im5vbmUifX19XSxbMiwyNSwiIiwxLHsic3R5bGUiOnsiaGVhZCI6eyJuYW1lIjoibm9uZSJ9fX1dLFsxNywyNywiIiwxLHsic3R5bGUiOnsiaGVhZCI6eyJuYW1lIjoibm9uZSJ9fX1dLFsyNSwyNCwiIiwxLHsiY29sb3VyIjpbMjM2LDkxLDYwXSwic3R5bGUiOnsiaGVhZCI6eyJuYW1lIjoibm9uZSJ9fX1dLFsyNSwyNiwiIiwxLHsiY29sb3VyIjpbMjM2LDkxLDYwXSwic3R5bGUiOnsiaGVhZCI6eyJuYW1lIjoibm9uZSJ9fX1dLFsyNywyOCwiIiwxLHsiY29sb3VyIjpbMSw5NSw2MF0sInN0eWxlIjp7ImhlYWQiOnsibmFtZSI6Im5vbmUifX19XSxbMTgsMjgsIiIsMSx7InN0eWxlIjp7ImhlYWQiOnsibmFtZSI6Im5vbmUifX19XSxbOSwyNiwiIiwxLHsic3R5bGUiOnsiaGVhZCI6eyJuYW1lIjoibm9uZSJ9fX1dLFs1LDI0LCIiLDEseyJzdHlsZSI6eyJoZWFkIjp7Im5hbWUiOiJub25lIn19fV0sWzI3LDI2LCIiLDEseyJjb2xvdXIiOlsxLDk1LDYwXSwic3R5bGUiOnsiaGVhZCI6eyJuYW1lIjoibm9uZSJ9fX1dLFsyNywxOCwiIiwxLHsiY29sb3VyIjpbMSw5NSw2MF0sInN0eWxlIjp7ImhlYWQiOnsibmFtZSI6Im5vbmUifX19XSxbMjcsOSwiIiwxLHsiY29sb3VyIjpbMSw5NSw2MF0sInN0eWxlIjp7ImhlYWQiOnsibmFtZSI6Im5vbmUifX19XSxbMjksNiwiIiwxLHsic3R5bGUiOnsiYm9keSI6eyJuYW1lIjoiZG90dGVkIn19fV0sWzMwLDEsIiIsMSx7InN0eWxlIjp7ImJvZHkiOnsibmFtZSI6ImRvdHRlZCJ9fX1dLFsxOSwxMCwiIiwwLHsic3R5bGUiOnsiYm9keSI6eyJuYW1lIjoiZG90dGVkIn19fV0sWzIwLDExLCIiLDAseyJzdHlsZSI6eyJib2R5Ijp7Im5hbWUiOiJkb3R0ZWQifX19XSxbMjEsMzEsIiIsMCx7InN0eWxlIjp7ImJvZHkiOnsibmFtZSI6ImRvdHRlZCJ9fX1dLFsyMywzMiwiIiwwLHsic3R5bGUiOnsiYm9keSI6eyJuYW1lIjoiZG90dGVkIn19fV0sWzIyLDMzLCIiLDAseyJzdHlsZSI6eyJib2R5Ijp7Im5hbWUiOiJkb3R0ZWQifX19XSxbMzQsMjQsIiIsMSx7InN0eWxlIjp7ImJvZHkiOnsibmFtZSI6ImRvdHRlZCJ9fX1dXQ==
\begin{tikzcd}[column sep=1ex,row sep=1ex]
  & {\mathrm{outer}} & {\mathrm{back}} & {\mathrm{hole}} & {\mathrm{front}} & {\mathrm{outer}} \\
  & {\ } & \bullet & {\ } & \bullet & {\ } \\
  {\mathrm{top\ of\ diamonds}} & \bullet && \bullet && \bullet \\
  && \bullet && \bullet \\
  & \bullet && \bullet && \bullet \\
  && \bullet && \bullet \\
  {\mathrm{bottom\ of\ diamonds}} & \bullet &&&& \bullet \\
  && \bullet && \bullet \\
  & \bullet && \bullet && \bullet \\
  && \bullet && \bullet \\
  {\mathrm{top\ of\ diamonds}} & \bullet && \bullet && \bullet
  \arrow[dotted, from=1-2, to=2-2]
  \arrow[dotted, from=1-3, to=2-3]
  \arrow[dotted, from=1-4, to=2-4]
  \arrow[dotted, from=1-5, to=2-5]
  \arrow[dotted, from=1-6, to=2-6]
  \arrow[draw={rgb,255:red,60;green,73;blue,246}, no head, from=2-3, to=3-2]
  \arrow[draw={rgb,255:red,60;green,73;blue,246}, no head, from=2-3, to=3-4]
  \arrow[no head, from=2-3, to=4-3]
  \arrow[draw={rgb,255:red,250;green,59;blue,56}, no head, from=2-5, to=3-4]
  \arrow[draw={rgb,255:red,250;green,59;blue,56}, no head, from=2-5, to=3-6]
  \arrow[dotted, from=3-1, to=3-2]
  \arrow[draw={rgb,255:red,60;green,73;blue,246}, no head, from=3-2, to=4-3]
  \arrow[no head, from=3-2, to=5-2]
  \arrow[draw={rgb,255:red,60;green,73;blue,246}, no head, from=3-4, to=4-3]
  \arrow[draw={rgb,255:red,250;green,59;blue,56}, no head, from=3-4, to=4-5]
  \arrow[no head, from=3-4, to=5-4]
  \arrow[draw={rgb,255:red,250;green,59;blue,56}, no head, from=3-6, to=4-5]
  \arrow[draw={rgb,255:red,60;green,73;blue,246}, no head, from=4-3, to=5-2]
  \arrow[draw={rgb,255:red,60;green,73;blue,246}, no head, from=4-3, to=5-4]
  \arrow[no head, from=4-3, to=6-3]
  \arrow[no head, from=4-5, to=2-5]
  \arrow[draw={rgb,255:red,250;green,59;blue,56}, no head, from=4-5, to=5-4]
  \arrow[draw={rgb,255:red,250;green,59;blue,56}, no head, from=4-5, to=5-6]
  \arrow[draw={rgb,255:red,60;green,73;blue,246}, no head, from=5-2, to=6-3]
  \arrow[no head, from=5-2, to=7-2]
  \arrow[draw={rgb,255:red,60;green,73;blue,246}, no head, from=5-4, to=6-3]
  \arrow[draw={rgb,255:red,250;green,59;blue,56}, no head, from=5-4, to=6-5]
  \arrow[no head, from=5-6, to=3-6]
  \arrow[draw={rgb,255:red,250;green,59;blue,56}, no head, from=5-6, to=6-5]
  \arrow[draw={rgb,255:red,60;green,73;blue,246}, no head, from=6-3, to=7-2]
  \arrow[no head, from=6-3, to=8-3]
  \arrow[no head, from=6-5, to=4-5]
  \arrow[draw={rgb,255:red,250;green,59;blue,56}, no head, from=6-5, to=7-6]
  \arrow[dotted, from=7-1, to=7-2]
  \arrow[draw={rgb,255:red,60;green,73;blue,246}, no head, from=7-2, to=8-3]
  \arrow[no head, from=7-2, to=9-2]
  \arrow[no head, from=7-6, to=5-6]
  \arrow[draw={rgb,255:red,250;green,59;blue,56}, no head, from=7-6, to=8-5]
  \arrow[draw={rgb,255:red,60;green,73;blue,246}, no head, from=8-3, to=9-2]
  \arrow[draw={rgb,255:red,60;green,73;blue,246}, no head, from=8-3, to=9-4]
  \arrow[no head, from=8-3, to=10-3]
  \arrow[no head, from=8-5, to=6-5]
  \arrow[draw={rgb,255:red,250;green,59;blue,56}, no head, from=8-5, to=9-4]
  \arrow[draw={rgb,255:red,250;green,59;blue,56}, no head, from=8-5, to=9-6]
  \arrow[no head, from=8-5, to=10-5]
  \arrow[draw={rgb,255:red,60;green,73;blue,246}, no head, from=9-2, to=10-3]
  \arrow[no head, from=9-2, to=11-2]
  \arrow[draw={rgb,255:red,60;green,73;blue,246}, no head, from=9-4, to=10-3]
  \arrow[no head, from=9-4, to=11-4]
  \arrow[no head, from=9-6, to=7-6]
  \arrow[no head, from=9-6, to=11-6]
  \arrow[draw={rgb,255:red,60;green,73;blue,246}, no head, from=10-3, to=11-2]
  \arrow[draw={rgb,255:red,60;green,73;blue,246}, no head, from=10-3, to=11-4]
  \arrow[draw={rgb,255:red,250;green,59;blue,56}, no head, from=10-5, to=9-4]
  \arrow[draw={rgb,255:red,250;green,59;blue,56}, no head, from=10-5, to=9-6]
  \arrow[draw={rgb,255:red,250;green,59;blue,56}, no head, from=10-5, to=11-4]
  \arrow[draw={rgb,255:red,250;green,59;blue,56}, no head, from=10-5, to=11-6]
  \arrow[dotted, from=11-1, to=11-2]
\end{tikzcd}
\caption{The flat torus with one vertex removed. This also removes the edges and faces containing that vertex.}
\label{fig:flattorus_zero}
\end{figure}

\begin{mydef}
Let \( \mm:\combmfdt \) be a combinatorial manifold and \( Z \) an isolated set of vertices. A \defemph{vector field \( X \) on \( \mm \) with zero set \( Z \)} is a partial section of \( P \), i.e. a term \( X:\pit{x:\mm\setminus Z}T(x) \) (and eliding the unique term of \( Z\to\star \)). The \defemph{exponential map} \( \exp:P\to \mm \) is the map sending points in a fiber to the corresponding point in the link of the base point: \( \exp(x, y:\link(x))=y \). In commutative diagram form we have:
\end{mydef}
% https://q.uiver.app/#q=WzAsNCxbMCwyLCJNXFxzZXRtaW51cyBaIl0sWzIsMiwiXFxtYXRocm17RU19KFxcbWF0aGJie1p9LDEpIl0sWzIsMCwiXFxtYXRocm17RU19X1xcYnVsbGV0KFxcbWF0aGJie1p9LDEpIl0sWzAsMCwiUFxcc3RhY2tyZWx7XFxtYXRocm17ZGVmfX09e1xcc3VtX3tDOlRNfUN9Il0sWzAsMSwiVCJdLFsyLDEsIlxcbWF0aHJte3ByfV8xIl0sWzMsMCwiXFxtYXRocm17cHJ9XzEiXSxbMywyLCJcXG92ZXJsaW5le1R9Il0sWzAsMywie1g6XFxwcm9kX3t4Ok1cXHNldG1pbnVzIFp9VHh9IiwwLHsiY3VydmUiOi0zfV0sWzMsMCwiXFxtYXRocm17ZXhwfSIsMCx7ImN1cnZlIjotM31dLFswLDIsIlRfXFxidWxsZXQiXSxbMywxLCIiLDAseyJzdHlsZSI6eyJuYW1lIjoiY29ybmVyIn19XV0=
\begin{center}
\begin{tikzcd}
  {P\stackrel{\mathrm{def}}={\sum_{C:T(\mm\setminus Z)}C}} && {\mathrm{EM}_\bullet(\mathbb{Z},1)} \\
  \\
  {\mm\setminus Z} && {\mathrm{EM}(\mathbb{Z},1)}
  \arrow["{\overline{T}}", from=1-1, to=1-3]
  \arrow["{\mathrm{pr}_1}", from=1-1, to=3-1]
  \arrow["{\mathrm{exp}}", curve={height=-18pt}, from=1-1, to=3-1]
  \arrow["\lrcorner"{anchor=center, pos=0.125}, draw=none, from=1-1, to=3-3]
  \arrow["{\mathrm{pr}_1}", from=1-3, to=3-3]
  \arrow["{{X:\prod_{x:\mm\setminus Z}Tx}}", curve={height=-18pt}, from=3-1, to=1-1]
  \arrow["{T_\bullet}", from=3-1, to=1-3]
  \arrow["T", from=3-1, to=3-3]
\end{tikzcd}
\end{center}
where \( T_\bullet=\overline{T}\circ X \). Note that \( \exp \) is different from \( \pr_1 \) since it spreads a fiber out onto the manifold. The composition \( \exp\circ X \) is a map \( \mm\setminus Z\to \mm \), and can be thought of as the flow of the vector field.

Let's see a few examples.

\begin{mydef}
The \defemph{spinning vector field} \( \Xspin \) on \( \oo\setminus\{w, y\} \) is given by the following data. We compose with \( \exp \) to keep the notation directly in \( \oo \). See Figure~\ref{fig:sphere_spin}
\begin{align*}
\exp\circ\Xspin(b)  &=r   \\
\exp\circ\Xspin(r)  &=g   \\
\exp\circ\Xspin(g)  &=o   \\
\exp\circ\Xspin(o)  &=b   \\
\end{align*}
We must also define pathovers and faceovers, i.e. \( \apd(\Xspin \). For example, \( \Xspin(b) \) is the point \( r \) in the link \( woyr \). Transport along \( br \) takes the link of \( b \) to the link of \( r \), mapping \( r:Tb \) to \( g:Tr \). This agrees with \( \Xspin(r) \) and so we will choose \( \Xspin(br)=\refl_g \) in \( Tr \). We similarly choose \( \refl \) pathovers for the other equatorial edges. And since we have deleted all the faces when removing the zeros, there are no faceovers.
\end{mydef}

\begin{figure}[htbp]
\centering
\begin{tikzpicture}%
  [x={(-0.860769cm, -0.121512cm)},
  y={(0.508996cm, -0.205391cm)},
  z={(-0.000053cm, 0.971107cm)},
  scale=1,
  eqback/.style={-Latex, very thick},
  back/.style={loosely dotted, thin},
  eqedge/.style={-Latex, very thick},
  edge/.style={black, thin},
  facet/.style={fill=blue!95!black,fill opacity=0.0},
  vertex/.style={inner sep=1pt,circle,draw=green!25!black,fill=black,thick}]
\coordinate (-1, -1, 0) at (-1, -1, 0);
\coordinate (-1, 1, 0) at (-1, 1, 0);
\coordinate (0, 0, -1) at (0, 0, -1);
\coordinate (0, 0, 1) at (0, 0, 1);
\coordinate (1, -1, 0) at (1, -1, 0);
\coordinate (1, 1, 0) at (1, 1, 0);
%% Drawing edges in the back
%%
\draw[edge,eqback] (-1, -1, 0) -- (-1, 1, 0);
\draw[edge,back] (-1, -1, 0) -- (0, 0, -1.4);
\draw[edge,back] (-1, -1, 0) -- (0, 0, 1.4);
\draw[edge,eqback] (1, -1, 0) -- (-1, -1, 0);
%% Drawing vertices in the back
%%
\node[vertex] at (-1, -1, 0)     {};
%% Drawing the facets
%%
\fill[facet] (1, 1, 0) -- (0, 0, -1.4) -- (1, -1, 0) -- cycle {};
\fill[facet] (1, 1, 0) -- (0, 0, 1.4) -- (1, -1, 0) -- cycle {};
\fill[facet] (1, 1, 0) -- (-1, 1, 0) -- (0, 0, 1.4) -- cycle {};
\fill[facet] (1, 1, 0) -- (-1, 1, 0) -- (0, 0, -1.4) -- cycle {};
%% Drawing edges in the front
%%
\draw[edge] (-1, 1, 0) -- (0, 0, -1.4);
\draw[edge] (-1, 1, 0) -- (0, 0, 1.4);
\draw[eqedge] (-1, 1, 0) -- (1, 1, 0);
\draw[edge] (0, 0, -1.4) -- (1, -1, 0);
\draw[edge] (0, 0, -1.4) -- (1, 1, 0);
\draw[edge] (0, 0, 1.4) -- (1, -1, 0);
\draw[edge] (0, 0, 1.4) -- (1, 1, 0);
\draw[eqedge] (1, 1, 0) -- (1, -1, 0);
%% Drawing the vertices in the front
%%
\begin{scope}[nodes=vertex]
\node[label=above right:\( b \)] at (-1, 1, 0)     {};
\node[label=below:\( y \)] at (0, 0, -1.4)     {};
\node[label=above:\( w \)] at (0, 0, 1.4)     {};
\node[label=above left:\( g \)] at (1, -1, 0)     {};
\node[label=above left:\( r \)] at (1, 1, 0)     {};
\node[label=above right:\( o \)] at (-1, -1, 0)     {};
\end{scope}
\end{tikzpicture}
\caption{The vector field \( \Xspin \) on \( \oo \), which circulates around the equator. \( w \) and \( y \) are zeros.}
\label{fig:sphere_spin}
\end{figure}%fig:sphere_spin

\begin{mydef}
The \defemph{downward vector field} \( \Xdown \) on \( \oo\setminus\{w, y\} \) is given by the following data, where again we compose with \( \exp \) to keep the notation directly in \( \oo \). See Figure~\ref{fig:sphere_downward}
\begin{align*}
\exp\circ\Xspin(b)  &=y   \\
\exp\circ\Xspin(r)  &=y   \\
\exp\circ\Xspin(g)  &=y   \\
\exp\circ\Xspin(o)  &=y   \\
\end{align*}
We also need to select a pathover for each edge on the equator. Transport on all these edges takes \( y \) in one fiber to \( y \) in the next, so we choose the path \( \refl_y\) in all four of these fibers. Again there are no faceovers to map.
\end{mydef}

\begin{figure}[htbp]
\centering
\begin{tikzpicture}%
  [x={(-0.860769cm, -0.121512cm)},
  y={(0.508996cm, -0.205391cm)},
  z={(-0.000053cm, 0.971107cm)},
  scale=1,
  eqback/.style={-Latex, very thick},
  back/.style={loosely dotted, thin},
  eqedge/.style={-Latex, very thick},
  edge/.style={black, thin},
  facet/.style={fill=blue!95!black,fill opacity=0.0},
  vertex/.style={inner sep=1pt,circle,draw=green!25!black,fill=black,thick}]
\coordinate (-1, -1, 0) at (-1, -1, 0);
\coordinate (-1, 1, 0) at (-1, 1, 0);
\coordinate (0, 0, -1) at (0, 0, -1);
\coordinate (0, 0, 1) at (0, 0, 1);
\coordinate (1, -1, 0) at (1, -1, 0);
\coordinate (1, 1, 0) at (1, 1, 0);
%% Drawing edges in the back
%%
\draw[edge,back] (-1, -1, 0) -- (-1, 1, 0);
\draw[eqedge] (-1, -1, 0) -- (0, 0, -1.4);
\draw[edge,back] (-1, -1, 0) -- (0, 0, 1.4);
\draw[edge,back] (1, -1, 0) -- (-1, -1, 0);
%% Drawing vertices in the back
%%
\node[vertex] at (-1, -1, 0)     {};
%% Drawing the facets
%%
\fill[facet] (1, 1, 0) -- (0, 0, -1.4) -- (1, -1, 0) -- cycle {};
\fill[facet] (1, 1, 0) -- (0, 0, 1.4) -- (1, -1, 0) -- cycle {};
\fill[facet] (1, 1, 0) -- (-1, 1, 0) -- (0, 0, 1.4) -- cycle {};
\fill[facet] (1, 1, 0) -- (-1, 1, 0) -- (0, 0, -1.4) -- cycle {};
%% Drawing edges in the front
%%
\draw[eqedge] (-1, 1, 0) -- (0, 0, -1.4);
\draw[edge] (-1, 1, 0) -- (0, 0, 1.4);
\draw[edge] (-1, 1, 0) -- (1, 1, 0);
\draw[eqedge] (1, -1, 0) -- (0, 0, -1.4);
\draw[eqedge] (1, 1, 0) -- (0, 0, -1.4);
\draw[edge] (0, 0, 1.4) -- (1, -1, 0);
\draw[edge] (0, 0, 1.4) -- (1, 1, 0);
\draw[edge] (1, 1, 0) -- (1, -1, 0);
%% Drawing the vertices in the front
%%
\begin{scope}[nodes=vertex]
\node[label=above right:\( b \)] at (-1, 1, 0)     {};
\node[label=below:\( y \)] at (0, 0, -1.4)     {};
\node[label=above:\( w \)] at (0, 0, 1.4)     {};
\node[label=above left:\( g \)] at (1, -1, 0)     {};
\node[label=above left:\( r \)] at (1, 1, 0)     {};
\node[label=above right:\( o \)] at (-1, -1, 0)     {};
\end{scope}
\end{tikzpicture}
\caption{The vector field \( \Xdown \) on \( \oo \), which flows downward. \( w \) and \( y \) are zeros.}
\label{fig:sphere_downward}
\end{figure}%fig:sphere_downward

\subsection{Index of a vector field}

Index should be an integer that computes a winding number ``of the vector field'' around a zero. We can compute an integer from a map by taking its \emph{degree}, which is a construction we will assume that we have, for example using \cite{buchholtz_favonia}, where they indeed require that degree agrees with winding number for maps \( S^1\to S^1 \).

\begin{mydef}
Let \( \mm:\combmfdt \) and let \( T:\mm\to\EMzo \) be a bundle of circles given on \( \mm_0 \) by \( \link \). Let \( z:Z \) be an isolated zero and let \( \link z \) be its polygonal link in \( \mm \), with a clockwise orientation, say with ordered vertices \( \{l_{z1},\ldots,l_{zn}\} \). We call the degree of the map \( \tr(\link z):Tl_{z1}=Tl_{z1} \) the \defemph{index of \( X \) at \( z \)}. It does not depend on which vertex we use.
\end{mydef}

% \begin{mydef}
% Let \( \mm:\combmfdt \), let \( T:\mm\to\EMzo \) be the discrete tangent bundle given on \( \mm_0 \) by \( \link \). Let \( P\defeq\sit{x:\mm}Tx \), and \( Z \) be a set of isolated zeros. Let \( X:\mm\setminus Z\to P \) be a nonvanishing vector field. Finally, let \( z:Z \) be a zero and let \( \link z \) be its polygonal link in \( \mm \), with a clockwise orientation, say with ordered vertices \( \{l_{z1},\ldots,l_{zn}\} \). Define a map \( X'(z):\link z\to \link z \), called \defemph{the circle automorphism at \( z \)}, on vertices of \( \link z \) as follows:
% \[ 
% X'(z)(l_{zi}) = \exp\left(\tr(l_{zi}z)(X(l_{zi})\right) 
% \] In words, we transport the vector field along the unique edge joining \( l_{zi} \) to \( z \), then apply \( \exp \). The result will be some point in \( \link z \). In different words, \( X(l_{zi}) \) may point various places: to another point of \( \link z \), to \( z \) itself, or somewhere beyond \( \link z \). Transporting all these fibers to the central fiber at \( z \), then exponentiating, gives a self-map of the link.
% 
% \( X'(z) \) extends to the edges of \( \link z \), e.g. \( l_{z1}l_{z2} \), by the action on paths of \( \tr \) along \( l_{zi}z \). To visualize this we can imagine transporting \( X \) along an edge of \( \link z \) from \( l_{z1} \) to \( l_{z2} \) and then transporting the result along \( l_{z2}z \).
% 
% Finally, we call the degree of \( X'(z) \) on the clockwise-oriented link the \defemph{index of \( X \) at \( z \)}.\qed
% \end{mydef}

\begin{mylemma}
The index of \( \Xspin \) at both \( y \) and at \( w \) is 1, and the same is true for \( \Xdown \).
\end{mylemma}
\begin{proof}
\( \apd(\Xspin)(br) = \refl_{\Xspin(r)} \) and similarly for the other edges and for \( \Xdown \). So \( \apd \) on the loop around the equator is the identity, which has index 1.
\end{proof}

If these vector fields both have index +1, what does index --1 look like? We can see an example on the torus with its downward vector field.

On the torus we can also consider both a spinning vector field and a downward vector field. Figure~\ref{fig:spin_torus} shows one way to spin the torus, and in this case there are no zeros so the index is the degree of a map from the empty set, which is 0 (as it factors through a constant map).

Figure~\ref{fig:morse_torus} shows a downward flow with four zeros. Although this is a picture of the flat torus, the vector field is derived from the shape of Figure~\ref{fig:torus} where we actually have a notion of up and down. We see at the position labeled (outer, top of diamonds), i.e. the top of the torus, an everywhere outward pointing vector field. At (outer, bottom of diamonds) we see an inward pointing vector field. But at (hole, top of diamonds), i.e. the top of the hole, we see something else. Imagine transport from the neighbor to the lower right to the neighbor below, then continuing clockwise around the link. If we assume that we define \( \apd \) on these edges to be \emph{counterclockwise} rotation, then transport around the whole link has degree --1. Similarly for the zero at (hole, bottom of diamonds). Adding these four indexes we again get 0.

Summarizing what we've seen in our examples, vector fields with isolated zeros have an index, and this index tracks with the total curvature, and the Euler characteristic.

\begin{figure}[htbp]
\centering
\begin{tikzcd}[column sep=1ex,row sep=1ex]
  & {\mathrm{outer}} & {\mathrm{back}} & {\mathrm{hole}} & {\mathrm{front}} & {\mathrm{outer}} \\
  & {\ } & \bullet & {\ } & \bullet & {\ } \\
  {\mathrm{top\ of\ diamonds}} & {\bullet} && {\bullet} && {\bullet} \\
  && \bullet && \bullet \\
  & \bullet && \bullet && \bullet \\
  && \bullet && \bullet \\
  {\mathrm{bottom\ of\ diamonds}} & {\bullet} && {\bullet} && {\bullet} \\
  && \bullet && \bullet \\
  & \bullet && \bullet && \bullet \\
  && \bullet && \bullet \\
  {\mathrm{top\ of\ diamonds}} & {\bullet} && {\bullet} && {\bullet}
  \arrow[dotted, from=1-2, to=2-2]
  \arrow[dotted, from=1-3, to=2-3]
  \arrow[dotted, from=1-4, to=2-4]
  \arrow[dotted, from=1-5, to=2-5]
  \arrow[dotted, from=1-6, to=2-6]
  \arrow[dotted, from=3-1, to=3-2]
  \arrow[dotted, from=7-1, to=7-2]
  \arrow[dotted, from=11-1, to=11-2]
  \arrow[draw={rgb,255:red,60;green,73;blue,246}, no head, from=2-3, to=3-2]
  \arrow[draw={rgb,255:red,60;green,73;blue,246}, no head, from=2-3, to=3-4]
  \arrow[line width=.5mm, Latex-, from=2-3, to=4-3]
  \arrow[draw={rgb,255:red,250;green,59;blue,56}, no head, from=2-5, to=3-4]
  \arrow[draw={rgb,255:red,250;green,59;blue,56}, no head, from=2-5, to=3-6]
  \arrow[draw={rgb,255:red,60;green,73;blue,246}, no head, from=3-2, to=4-3]
  \arrow[line width=.5mm, Latex-, from=3-2, to=5-2]
  \arrow[draw={rgb,255:red,60;green,73;blue,246}, no head, from=3-4, to=4-3]
  \arrow[draw={rgb,255:red,250;green,59;blue,56}, no head, from=3-4, to=4-5]
  \arrow[line width=.5mm, Latex-, from=3-4, to=5-4]
  \arrow[draw={rgb,255:red,250;green,59;blue,56}, no head, from=3-6, to=4-5]
  \arrow[draw={rgb,255:red,60;green,73;blue,246}, no head, from=4-3, to=5-2]
  \arrow[draw={rgb,255:red,60;green,73;blue,246}, no head, from=4-3, to=5-4]
  \arrow[line width=.5mm, Latex-, from=4-3, to=6-3]
  \arrow[line width=.5mm, -Latex, from=4-5, to=2-5]
  \arrow[draw={rgb,255:red,250;green,59;blue,56}, no head, from=4-5, to=5-4]
  \arrow[draw={rgb,255:red,250;green,59;blue,56}, no head, from=4-5, to=5-6]
  \arrow[draw={rgb,255:red,60;green,73;blue,246}, no head, from=5-2, to=6-3]
  \arrow[line width=.5mm, Latex-, from=5-2, to=7-2]
  \arrow[draw={rgb,255:red,60;green,73;blue,246}, no head, from=5-4, to=6-3]
  \arrow[draw={rgb,255:red,250;green,59;blue,56}, no head, from=5-4, to=6-5]
  \arrow[line width=.5mm, Latex-, from=5-4, to=7-4]
  \arrow[line width=.5mm, -Latex, from=5-6, to=3-6]
  \arrow[draw={rgb,255:red,250;green,59;blue,56}, no head, from=5-6, to=6-5]
  \arrow[draw={rgb,255:red,60;green,73;blue,246}, no head, from=6-3, to=7-2]
  \arrow[draw={rgb,255:red,60;green,73;blue,246}, no head, from=6-3, to=7-4]
  \arrow[line width=.5mm, Latex-, from=6-3, to=8-3]
  \arrow[line width=.5mm, -Latex, from=6-5, to=4-5]
  \arrow[draw={rgb,255:red,250;green,59;blue,56}, no head, from=6-5, to=7-4]
  \arrow[draw={rgb,255:red,250;green,59;blue,56}, no head, from=6-5, to=7-6]
  \arrow[draw={rgb,255:red,60;green,73;blue,246}, no head, from=7-2, to=8-3]
  \arrow[line width=.5mm, Latex-, from=7-2, to=9-2]
  \arrow[draw={rgb,255:red,60;green,73;blue,246}, no head, from=7-4, to=8-3]
  \arrow[draw={rgb,255:red,250;green,59;blue,56}, no head, from=7-4, to=8-5]
  \arrow[line width=.5mm, Latex-, from=7-4, to=9-4]
  \arrow[line width=.5mm, -Latex, from=7-6, to=5-6]
  \arrow[draw={rgb,255:red,250;green,59;blue,56}, no head, from=7-6, to=8-5]
  \arrow[draw={rgb,255:red,60;green,73;blue,246}, no head, from=8-3, to=9-2]
  \arrow[draw={rgb,255:red,60;green,73;blue,246}, no head, from=8-3, to=9-4]
  \arrow[line width=.5mm, Latex-, from=8-3, to=10-3]
  \arrow[line width=.5mm, -Latex, from=8-5, to=6-5]
  \arrow[draw={rgb,255:red,250;green,59;blue,56}, no head, from=8-5, to=9-4]
  \arrow[draw={rgb,255:red,250;green,59;blue,56}, no head, from=8-5, to=9-6]
  \arrow[line width=.5mm, Latex-, from=8-5, to=10-5]
  \arrow[draw={rgb,255:red,60;green,73;blue,246}, no head, from=9-2, to=10-3]
  \arrow[line width=.5mm, Latex-, from=9-2, to=11-2]
  \arrow[draw={rgb,255:red,60;green,73;blue,246}, no head, from=9-4, to=10-3]
  \arrow[line width=.5mm, Latex-, from=9-4, to=11-4]
  \arrow[line width=.5mm, -Latex, from=9-6, to=7-6]
  \arrow[line width=.5mm, Latex-, from=9-6, to=11-6]
  \arrow[draw={rgb,255:red,60;green,73;blue,246}, no head, from=10-3, to=11-2]
  \arrow[draw={rgb,255:red,60;green,73;blue,246}, no head, from=10-3, to=11-4]
  \arrow[draw={rgb,255:red,250;green,59;blue,56}, no head, from=10-5, to=9-4]
  \arrow[draw={rgb,255:red,250;green,59;blue,56}, no head, from=10-5, to=9-6]
  \arrow[draw={rgb,255:red,250;green,59;blue,56}, no head, from=10-5, to=11-4]
  \arrow[draw={rgb,255:red,250;green,59;blue,56}, no head, from=10-5, to=11-6]
\end{tikzcd}

\caption{A vector field on the torus that spins and has no zeros.}
\label{fig:spin_torus}
\end{figure}%fig:spin_torus

\begin{figure}[htbp]
\centering
\begin{tikzcd}[column sep=1ex,row sep=1ex]
  & {\mathrm{outer}} & {\mathrm{back}} & {\mathrm{hole}} & {\mathrm{front}} & {\mathrm{outer}} \\
  & {\ } & \bullet & {\ } & \bullet & {\ } \\
  {\mathrm{top\ of\ diamonds}} & {} && {} && {} \\
  && \bullet && \bullet \\
  & \bullet && \bullet && \bullet \\
  && \bullet && \bullet \\
  {\mathrm{bottom\ of\ diamonds}} & {} && {} && {} \\
  && \bullet && \bullet \\
  & \bullet && \bullet && \bullet \\
  && \bullet && \bullet \\
  {\mathrm{top\ of\ diamonds}} & {} && {} && {}
  \arrow[dotted, from=1-2, to=2-2]
  \arrow[dotted, from=1-3, to=2-3]
  \arrow[dotted, from=1-4, to=2-4]
  \arrow[dotted, from=1-5, to=2-5]
  \arrow[dotted, from=1-6, to=2-6]
  \arrow[draw={rgb,255:red,60;green,73;blue,246}, no head, from=2-3, to=3-2]
  \arrow[draw={rgb,255:red,60;green,73;blue,246}, from=2-3, to=3-4, line width=.5mm, -Latex]
  \arrow[no head, from=2-3, to=4-3]
  \arrow[draw={rgb,255:red,250;green,59;blue,56}, from=2-5, to=3-4, line width=.5mm, -Latex]
  \arrow[draw={rgb,255:red,250;green,59;blue,56}, no head, from=2-5, to=3-6]
  \arrow[dotted, from=3-1, to=3-2]
  \arrow[draw={rgb,255:red,60;green,73;blue,246}, no head, from=3-2, to=4-3]
  \arrow[no head, from=3-2, to=5-2]
  \arrow[draw={rgb,255:red,60;green,73;blue,246}, tail reversed, no head, from=3-4, to=4-3, line width=.5mm, Latex-]
  \arrow[draw={rgb,255:red,250;green,59;blue,56}, tail reversed, no head, from=3-4, to=4-5, line width=.5mm, Latex-]
  \arrow[no head, from=3-4, to=5-4]
  \arrow[draw={rgb,255:red,250;green,59;blue,56}, no head, from=3-6, to=4-5]
  \arrow[draw={rgb,255:red,60;green,73;blue,246}, no head, from=4-3, to=5-2]
  \arrow[draw={rgb,255:red,60;green,73;blue,246}, no head, from=4-3, to=5-4]
  \arrow[no head, from=4-3, to=6-3]
  \arrow[no head, from=4-5, to=2-5]
  \arrow[draw={rgb,255:red,250;green,59;blue,56}, no head, from=4-5, to=5-4]
  \arrow[draw={rgb,255:red,250;green,59;blue,56}, no head, from=4-5, to=5-6]
  \arrow[draw={rgb,255:red,60;green,73;blue,246}, no head, from=5-2, to=6-3]
  \arrow[from=5-2, to=7-2, line width=.5mm, -Latex]
  \arrow[draw={rgb,255:red,60;green,73;blue,246}, no head, from=5-4, to=6-3]
  \arrow[draw={rgb,255:red,250;green,59;blue,56}, no head, from=5-4, to=6-5]
  \arrow[from=5-4, to=7-4, line width=.5mm, -Latex]
  \arrow[no head, from=5-6, to=3-6]
  \arrow[draw={rgb,255:red,250;green,59;blue,56}, no head, from=5-6, to=6-5]
  \arrow[draw={rgb,255:red,60;green,73;blue,246}, from=6-3, to=7-2, line width=.5mm, -Latex]
  \arrow[draw={rgb,255:red,60;green,73;blue,246}, no head, from=6-3, to=7-4]
  \arrow[no head, from=6-3, to=8-3]
  \arrow[no head, from=6-5, to=4-5]
  \arrow[draw={rgb,255:red,250;green,59;blue,56}, no head, from=6-5, to=7-4]
  \arrow[draw={rgb,255:red,250;green,59;blue,56}, from=6-5, to=7-6, line width=.5mm, -Latex]
  \arrow[dotted, from=7-1, to=7-2]
  \arrow[draw={rgb,255:red,60;green,73;blue,246}, tail reversed, no head, from=7-2, to=8-3, line width=.5mm, Latex-]
  \arrow[tail reversed, no head, from=7-2, to=9-2, line width=.5mm, Latex-]
  \arrow[draw={rgb,255:red,60;green,73;blue,246}, no head, from=7-4, to=8-3]
  \arrow[draw={rgb,255:red,250;green,59;blue,56}, no head, from=7-4, to=8-5]
  \arrow[tail reversed, no head, from=7-4, to=9-4, line width=.5mm, Latex-]
  \arrow[tail reversed, no head, from=7-6, to=5-6, line width=.5mm, Latex-]
  \arrow[draw={rgb,255:red,250;green,59;blue,56}, tail reversed, no head, from=7-6, to=8-5, line width=.5mm, Latex-]
  \arrow[draw={rgb,255:red,60;green,73;blue,246}, no head, from=8-3, to=9-2]
  \arrow[draw={rgb,255:red,60;green,73;blue,246}, no head, from=8-3, to=9-4]
  \arrow[no head, from=8-3, to=10-3]
  \arrow[no head, from=8-5, to=6-5]
  \arrow[draw={rgb,255:red,250;green,59;blue,56}, no head, from=8-5, to=9-4]
  \arrow[draw={rgb,255:red,250;green,59;blue,56}, no head, from=8-5, to=9-6]
  \arrow[no head, from=8-5, to=10-5]
  \arrow[draw={rgb,255:red,60;green,73;blue,246}, no head, from=9-2, to=10-3]
  \arrow[no head, from=9-2, to=11-2]
  \arrow[draw={rgb,255:red,60;green,73;blue,246}, no head, from=9-4, to=10-3]
  \arrow[no head, from=9-4, to=11-4]
  \arrow[from=9-6, to=7-6, line width=.5mm, -Latex]
  \arrow[no head, from=9-6, to=11-6]
  \arrow[draw={rgb,255:red,60;green,73;blue,246}, no head, from=10-3, to=11-2]
  \arrow[draw={rgb,255:red,60;green,73;blue,246}, from=10-3, to=11-4, line width=.5mm, -Latex]
  \arrow[draw={rgb,255:red,250;green,59;blue,56}, no head, from=10-5, to=9-4]
  \arrow[draw={rgb,255:red,250;green,59;blue,56}, no head, from=10-5, to=9-6]
  \arrow[draw={rgb,255:red,250;green,59;blue,56}, from=10-5, to=11-4, line width=.5mm, -Latex]
  \arrow[draw={rgb,255:red,250;green,59;blue,56}, no head, from=10-5, to=11-6]
  \arrow[dotted, from=11-1, to=11-2]
\end{tikzcd}

\caption{A vector field on the torus that flows downward, when viewed as a donut. The zeros are represented as missing dots. Every dot has one outgoing vector.}
\label{fig:morse_torus}
\end{figure}%fig:morse_torus

\subsection{Equality of total index and total curvature}

Here we are inspired by the classical proof of Hopf\cite{hopf}, presented in detail in Needham\cite{needham}.

\begin{mydef}
\label{def:enumeration}
An \defemph{enumeration} of a principal bundle with connection and vector field on a higher combinatorial manifold consists of the following data:
\begin{itemize}
\item a family \( T:\mm\to\Kzt \) on some higher combinatorial manifold
\item a nonvanishing vector field \( X:\mm\setminus Z\to P \) with isolated zeros \( Z \)
\item a total face of the replacement \( \mm_Z \) (Definition~\ref{def:totalface}), that is 
\begin{itemize}
\item a basepoint \( a:\mm_Z \) 
\item a term \( f_{\mm_Z}:\refl_a=\refl_a \) given by
\begin{itemize}
\item an ordering of the face constructors \( \{f_i\} \), with the sub-list of faces denoted \( \{f_{zk}\} \) refers to the replacement faces at the zeros
\item a vertex \( v_i \) in each face
\item terms \( a=v_i \) for each face
\end{itemize}
\end{itemize}
\end{itemize}
\end{mydef}

\begin{mynote}
An enumeration let us work both with \( \mm\setminus Z \) where the vector field is nonvanishing, while also having access to the disks that are missing from \( \mm\setminus Z \).
\end{mynote}

\begin{mylemma}
The sub-list of faces \( \{f_i\}-\{f_{zk}\} \) obtained by skipping the replacement faces at the zeros, is an ordering of face constructors for \( \mm\setminus Z \).
\end{mylemma}\begin{proof}
The algorithm that visits each face in order always starts and ends at \( a \) and so we can skip any faces we wish, to obtain an ordering of face constructors for the remaining union of faces.
\end{proof}

Note that on \( \mm\setminus Z \) the vector field \( X \) trivializes the bundle by mapping into the contractible type of pointed types over \( \Kzt \). So \( X\simeq Y:\mm\setminus Z\to (Ta,a) \), the fixed pointed circle in the fiber of the basepoint \( a \). 

\begin{mylemma}
The degree of \( Y \) is minus the total index of \( X \).
\end{mylemma}\begin{proof}
The ordering of faces \( \{f_i\}-\{f_{zk}\} \) provides an ordering of all the edges, say \( \{e_l\} \). Each edge appears an even number of times in this list, traversed in opposite directions, except those bounding a replacement face. These replacement-bounding edges are traversed an odd number of times and can be concatenated to traverse the boundary counterclockwise. Concatenation of paths in \( S^1 \) is abelian, so  \( Y(\{e_l\}) \) cancels except on the boundary of the replacement faces, which gives a map from the disjoint union of boundary circles to \( Ta \), with each boundary circle traversed in the counterclockwise direction. The orientation gives the minus sign.
\end{proof}

Consider now the total face \( f_{\mm_Z} \) of \( \mm_Z \) and its ordering of faces \( \{f_i\} \). \( Y \) is only defined on some of these faces. We will define a new function on all the \( \{f_i\} \).

\begin{mydef}
The \defemph{coupling map} \( C:\mm_Z\to Ta \) is defined to be \( Y \) on \( \mm\setminus Z \) and on the remaining faces it is defined by \( C(f_i)\defeq\apd(X)(\partial f_i) \) where \( \partial f_i:v_i=v_i \) is the clockwise path around the face starting from the vertex supplied by the data of the total face.
\end{mydef}

Because \( \apd \) uses both transport and the value of the vector field, it couples the connection with the vector field, hence the name. Of course in HoTT this coupling is built into the definition of \( \apd \), so that's another reminder that we aren't straying far from the foundations to find these geometrical concepts. 

The fact that \( C \) is defined on all the faces, by using the value of the vector field only on the 1-skeleton of \( \mm_Z \) where it was always defined, lets us make the final part of the argument.

\begin{mylemma}
\( C:\cup_i\{f_i\}\to Ta \) is equal to a constant map.
\end{mylemma}
\begin{proof}
The data of the total face provides an ordering of all the edges. Each edge appears an even number of times, traversed in opposite directions, including the edges in the replacement faces. Concatenation of paths in the polygon \( Ta \) is abelian, so the paths all cancel.
\end{proof}

\( C \) is similar to \( X \) and \( Y \) except that it is a total function. On any given edge it computes a path, that is, a homotopy from the function \( \tr \) to the function \( X \), which we can call ``the difference between transport and the vector field on that edge.'' We have found a way to add all these homotopies together to get 0. We can call this total ``the difference between the total index and the total curvature.''

Recall now that we made some arbitrary choices in Definition~\ref{def:enumeration} of an enumeration, and hence the function \( C \). But since \( C \) is unconditionally constant, the space of extra data is contractible.

\begin{mycor}
The total index of a vector field with isolated zeros is independent of the vector field.
\end{mycor}

\begin{mycor}
The total curvature is an integer.
\end{mycor}

The last step is to link this value to the Euler characteristic.

\subsection{Identification with Euler characteristic}

Here we only point the way. Combinatorial manifolds are intuitive objects that connect directly to the classical definition of Euler characteristic. We can argue using Morse theory, the study of smooth real-valued functions on smooth manifolds and their singularities. Classically the gradient of a Morse function is a vector field that can be used to decompose the manifold into its \emph{handlebody decomposition}. This would be an excellent story to pursue in future work.

Imagine starting with a classical 2-manifold of genus \( g \) that has been triangulated. Stand it upright with the holes forming a vertical sequence. Now install a vector field that points downward whenever possible. This vector field will have a zero at the top and bottom, and one at the top and bottom of each hole. The top and bottom will have zeros of (classical) index 1, and zeros in the holes will have index --1. We include some sketches in the case of a torus (Figure~\ref{fig:torus_morse_skel}). This illustrates how we obtain the formula for genus \( g \): \( \chi(M)=2-2g \). If we choose the triangulation so that the zeros are at vertices, we should be able to import that data into \( \combmfdt \) and reproduce the computation using the tools in this note.

\begin{figure}[htbp]
\centering
% https://q.uiver.app/#q=WzAsOCxbMCwwLCJcXGJ1bGxldCJdLFswLDEsIlxcYnVsbGV0Il0sWzAsMiwiXFxidWxsZXQiXSxbMCwzLCJcXGJ1bGxldCJdLFsxLDAsIlxcbWF0aHJte3RvcH0iXSxbMSwxLCJcXG1hdGhybXt1cHBlclxcIGhvbGV9Il0sWzEsMiwiXFxtYXRocm17bG93ZXJcXCBob2xlfSJdLFsxLDMsIlxcbWF0aHJte2JvdHRvbX0iXSxbMCwxLCIiLDIseyJjdXJ2ZSI6LTEsInN0eWxlIjp7ImJvZHkiOnsibmFtZSI6ImRhc2hlZCJ9fX1dLFswLDEsIiIsMCx7ImN1cnZlIjoxLCJzdHlsZSI6eyJib2R5Ijp7Im5hbWUiOiJkYXNoZWQifX19XSxbMiwzLCIiLDAseyJjdXJ2ZSI6LTEsInN0eWxlIjp7ImJvZHkiOnsibmFtZSI6ImRhc2hlZCJ9fX1dLFsyLDMsIiIsMix7ImN1cnZlIjoxLCJzdHlsZSI6eyJib2R5Ijp7Im5hbWUiOiJkYXNoZWQifX19XSxbMCwzLCIiLDEseyJjdXJ2ZSI6LTV9XSxbMCwzLCIiLDEseyJjdXJ2ZSI6NX1dLFsxLDIsIiIsMSx7ImN1cnZlIjoyfV0sWzEsMiwiIiwxLHsiY3VydmUiOi0yfV1d
\begin{tikzcd}[cramped]
  \bullet & {\mathrm{top}} \\
  \bullet & {\mathrm{upper\ hole}} \\
  \bullet & {\mathrm{lower\ hole}} \\
  \bullet & {\mathrm{bottom}}
  \arrow[curve={height=-6pt}, dashed, from=1-1, to=2-1]
  \arrow[curve={height=6pt}, dashed, from=1-1, to=2-1]
  \arrow[curve={height=-30pt}, from=1-1, to=4-1]
  \arrow[curve={height=30pt}, from=1-1, to=4-1]
  \arrow[curve={height=12pt}, from=2-1, to=3-1]
  \arrow[curve={height=-12pt}, from=2-1, to=3-1]
  \arrow[curve={height=-6pt}, dashed, from=3-1, to=4-1]
  \arrow[curve={height=6pt}, dashed, from=3-1, to=4-1]
\end{tikzcd}
\caption{Schematic of the zeros of the downward flow of a torus.}
\label{fig:torus_morse_skel}
\end{figure}
