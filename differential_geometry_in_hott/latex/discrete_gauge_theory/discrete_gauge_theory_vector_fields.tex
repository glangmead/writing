\section{Vector fields}
Vector fields are sections of the tangent bundle of a manifold. We do not have a general theory of tangent bundles, even for 2-dimensional higher combinatorial manifolds, since we cannot yet prove that connections always exist on the 1-skeleton. But \emph{given} an extension \( T_1 \) of the \( \link \) function to the 1-skeleton, we can consider the type of sections \( \pit{x:\mm_1}T_1(x) \).

\begin{mydef}
A \defemph{nonvanishing vector field} on a higher combinatorial 2-manifold \( \mm \) equipped with type family \( T_1:\mm_1\to\Kzt \) that extends \( \link \) is a term \( X:\pit{x:\mm_1}T_1(x) \). 
\end{mydef}

\begin{mynote}
We do not lose any classical generality by assuming we have such a nonvanishing field on the entire 1-skeleton. This corresponds to the fact that the zeros of a vector field can be perturbed to be isolated, and to live inside the faces of a triangulation.
\end{mynote}

\begin{mynote}
A section \( \pit{x:\mm}f(x) \) for \( f:\mm\to\Kzt \) is a trivialization of the bundle. The fact that an orientation suffices to factor the tangent bundle of a 2-manifold (which \emph{a priori} maps to \( \EMzo \)) through a principal bundle classifier is special to dimension 2. For higher dimensional manifolds the tangent bundle is a bundle of spheres, and even if the bundle is oriented it is not necessarily a principal bundle. On the other hand, the \( n \)-truncation modal operator maps the type of \( n \)-spheres to the classifying space \( \K(\zz,n) \), and so another way to phrase this remark is that \( S^1 \) is the only \( n \)-sphere which is \( n \)-truncated.
\end{mynote}

On the 0-skeleton \( X \) picks a point in each link, i.e. a neighbor of each vertex. On a path \( p:x=_\mm y \), \( X \) assigns a dependent path over \( p \), which as we know is a term \( \pi:\tr(p)(X(x))=_{T_1 y} X(y) \).

\begin{mynote}Concrete pathover terms such as \( \pi \) couple together the connection (the transport) and \( X \). The transport is needed to provide a single type in which to compare the two values of \( X \). The path \( \pi \) therefore reflects both the movement of \( X \) along the path \( p \), and the moving lens that allows us to examine in a single fiber the two points \( X(x) \) and \( X(y) \) as well as \( X(p) \).\end{mynote}

\subsection{Swirling}

Classically, vector fields are seen to swirl around between points. We can see this in HoTT as well, though in contrast to the classical explanations we always have a connection that interacts with the vector field. 

Consider a vertex \( x:\mm \), a face \( F \) containing vertices \( x, y, z \), and the boundary path \( \ell\defeq e_{xy}\cdot e_{yz} \cdot e_{zx} \). We can calculate \( X(\ell) \) by concatenating the data along the path. As we visit each point we accumulate more information:

\begin{enumerate}
\item In \( T_1(x) \) we have 
\begin{enumerate}
\item the point \( X(x) \).
\end{enumerate}
\item In \( T_1(y) \) we have
\begin{enumerate}
\item point \( X(y) \)
\item point \( \tr(e_{xy})(X(x)) \)
\item the path \( X(e_{xy}):\tr(e_{xy})(X(x))=X(y) \) from (b) to (a).
\end{enumerate}
\item In \( T_1(z) \) we have 
\begin{enumerate}
\item point \( X(z) \)
\item point \( \tr(e_{yz})(X(y)) \)
\item point \( \tr(e_{yz})\circ\tr(e_{xy})(X(x)) \)
\item path \( X(e_{yz}):\tr(e_{yz})(X(y))=X(z) \) from (b) to (a)
\item path \( \tr(e_{yz})(X(e_{xy})) \) from (c) to (b)
\end{enumerate}
\item Then back at \( T_1(x) \) we have
\begin{enumerate}
\item point \( X(x) \)
\item point \( \tr(e_{zx})(X(z)) \)
\item point \( \tr(e_{zx})\circ\tr(e_{yz})(X(y)) \)
\item point \( \tr(e_{zx})\circ\tr(e_{yz})\circ\tr(e_{xy})(X(x)) \), i.e. \( \tr(\ell)(X(x)) \)
\item path \( X(e_{zx}): \tr(e_{zx})(X(z))=X(x) \) from (b) to (a)
\item path \( \tr(e_{zx})(X(e_{yz})) \) from (c) to (b)
\item path \( \tr(e_{zx})\circ\tr(e_{yz})(X(e_{xy})) \) from (d) to (c).
\end{enumerate}
\end{enumerate}

As we traverse an edge, say \( e_{xy} \), we get a path in \( T_1(y) \) which is the image of \( e_{xy} \) under \( X \). Although this path is then transported along further edges of the triangle, it will help us to think of it as maintaining its relationship to \( e_{xy} \):

\begin{mydef}
If \( e_{xy}:x=_{\mm_1} y, e_{yz}:y=_{\mm_1}z \) are paths in \( \mm_1 \) and \( X \) is a nonvanishing vector field on \( \mm_1 \), then we call \( \tr(e_{yz})(X(e_{xy})) \) \defemph{the contribution on \( e_{xy} \) of \( X \)}, even though this is a path in \( T_1(z) \).
\end{mydef}

As we traverse an additional edge this image is simply mapped to the next vertex through the lens of transport, which acts similarly on all the points in a fiber. The image is carried first to \( \tr(e_{yz})(X(e_{xy})) \) then to \( \tr(e_{zx})\circ\tr(e_{yz})(X(x)) \).

At stage 4 we have three paths, each consisting of \( X \) acting on a single edge of the triangle, mapped with some transport to \( T_1(x) \). We see that these three can be concatenated to form a path \( X(\ell):\tr(\ell)(X(x))=X(x) \).

\begin{mynote}
In Hopf \cite{hopf} and in Needham \cite{needham}, the value \( X(e_{xy}) \) is called ``the change in angle between \( X \) and \( X(x)_{||} \) along the edge \( e_{xy} \),'' where by \( X(x)_{||} \) we mean the transport of the fixed single vector \( X(x) \) to the point \( X(y) \), i.e. \( \tr(e_{xy})(X(x)) \). The concatenation \( X(\ell) \) is called ``the sum of the changes in angle along each edge.'' This remark is meant to help anyone intending to make further comparisons with classical results.
\end{mynote}

When \( T_1 \) has an extension to \( F \) we can obtain a second path in \( \tr(\ell)(X(x))=X(x) \)! Namely given a filler path \( \ell=_{v=_\mm v} \refl(v) \) that fills the loop, an extension of \( T_1 \) to \( F \) provides by functoriality a corresponding path \( T_1(\ell)=_{T_1(v)=T_1(v)}\id(T_1(v)) \), which is a homotopy between these two automorphisms of \( T_1(v) \) that we called the flatness structure on \( F \). Evaluating this homotopy at \( X(x) \) provides the term \( \flat(\ell)(X(x)):X(x)=\tr(\ell)(X(x)) \) (and its inverse is hence the promised second path).

Concatenating \( \flat(\ell)(X(x))\cdot X(\ell)(X(x)) \) gives a loop in the pointed type \( (T_1(x), X(x)) \). This is a term in \( \zz \):

\begin{mydef}
\label{def:index}
The \defemph{index} of \( X \) at \( x \) on the face \( F \) is \( \flat(\ell)(X(x))\cdot X(\ell)(X(x)):\zz \).
\end{mydef}

\begin{mynote}
Intuitively, \( X(\ell) \) combines the swirling of \( X \) with the twisting of the connection, and prepending with the flatness structure erases the effect of the connection, leaving just the swirling of \( X \). This should line up with the classical definition of index, which is also an integer.
\end{mynote}
