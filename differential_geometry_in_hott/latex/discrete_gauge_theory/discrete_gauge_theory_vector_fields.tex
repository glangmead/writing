\section{Vector fields}

We will again use the dimension structure of a marked presented manifold(\ref{def:marked_presented_manifold}) to construct vector fields on each skeleton. Let \( M:\simcomp \) be a combinatorial manifold and \( \mm\defeq \mrlzn(M) \) the corresponding marked presented manifold.

\begin{mydef}
\label{def:vector_field}
A \defemph{nonvanishing flow} on a higher combinatorial manifold \( \mm \) is a section of the tangent bundle over the 0-skeleton, i.e. a term of \( X_0:\pit{x:\mm_0}Tx \).
\end{mydef}

\begin{mydef}
\label{def:vector_pathover}
A \defemph{nonvanishing vector field} on \( \mm \) is an extension \( X \) of \( X_0 \) from the 0-skeleton to the 1-skeleton: \( X:\pit{x:\mm_1}Tx \).
\end{mydef}

An extension of \( X \) to the full 2-skeleton of \( \mm \) would be a trivialization of the tangent bundle, which we do not stipulate. But we do not lose any classical generality by assuming we have such a section/trivialization on the 1-skeleton. This corresponds to the fact that the zeros of a vector field can be perturbed to be isolated, and to live inside the faces of a triangulation.

On 1-paths \( p:x=_{\mm_1}y \), \( X \) assigns a choice of pathover \( \pi:\tr(p)(x)=_{Ty}y \), which can also be thought of as a proof of pointedness of the transport map \( \tr(p) \). All told, given a loop \( \ell:m=_\mm m \) around a single face, we have the following constructions:

\begin{enumerate}
\item \( \tr(\ell):Tm=Tm \), the holonomy around \( \ell \).
\item \( \flat(\ell):\tr(\ell)=_{Tm=Tm}\id_{Tm} \), the flatness structure.
\item \( X(\ell):\tr(\ell)(Xm)=_{Tm} Xm \), the path over \( \ell \) selected by \( X \), i.e. a pointing of \( \tr(\ell) \).
\end{enumerate}

\subsection{Index of a vector field: subtracting curvature}

\subsection{Equality of total index and total curvature}

\subsection{Identification with Euler characteristic}


