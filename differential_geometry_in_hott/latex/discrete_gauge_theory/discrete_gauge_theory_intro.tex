\section{Overview}

The outline is that we will define 
\begin{itemize}
\item principal bundles in Section~\ref{sec:torsors},
\item simplicial complexes, and homotopical realizations of these in Section~\ref{sec:discrete_man},
\item vector fields in Section~\ref{sec:vector_fields},
\end{itemize}
and observe emerging from those definitions the presence of
\begin{itemize}
\item connections and curvature in Section~\ref{sec:connections},
\item the index of a vector field in Section~\ref{sec:totals},
\end{itemize}
and then define in Section~\ref{sec:totals}
\begin{itemize}
\item the total curvature, as in the Gauss-Bonnet theorem
\item the total index of a vector field, as in Poincaré-Hopf theorem,
\item and prove the equality of these to each other.
\end{itemize}

We will build up an example of all of these structures on an octahedron model of the sphere, and compute its Euler characteristic of 2. We will not, however, be supplying a separate definition of Euler characteristic so as to truly reproduce the Gauss-Bonnet and Poincaré-Hopf theorems.

We will consider functions \( \mm\to \EM(\zz,1) \) where \( \EM(\zz,1) \) is the connected component in the universe of the Eilenberg-MacLane space \( \K(\zz,1) \) which we will take to be \( \so \), and where \( \mm \) is a combinatorial manifold of dimension 2, which is a simplicial complex encoded in a higher inductive type, such that each vertex has a neighborhood that looks like a disk with a discrete circle boundary (i.e. a polygon). We can call terms \( C:\EM(\zz,1) \) ``mere circles.''

We will see in Section~\ref{sec:polygons} that \( \EM(\zz,1) \) contains all the polygons. We will construct a map \( \link:\mm\to\EM(\zz,1) \) that maps each vertex to the polygon consisting of its neighbors. Then we can consider the type of pointed mere circles \( \EMp(\zz,1)\defeq \sit{Y:\EM(\zz,1)}Y \) as well as the first projection that forgets the point. This is a univalent fibration (univalent fibrations are always equivalent to a projection of a type of pointed types to some connected component of the universe\cite{christensen_univalence}). If we form the pullback
% https://q.uiver.app/#q=WzAsNCxbMCwwLCJQXFxzdGFja3JlbHtcXG1hdGhybXtkZWZ9fXs9fVxcc3VtX3tDOlRNfUMiXSxbMSwwLCJcXG1hdGhybXtFTX1fXFxidWxsZXQoXFxtYXRoYmJ7Wn0sMSkiXSxbMSwxLCJcXG1hdGhybXtFTX0oXFxtYXRoYmJ7Wn0sMSkiXSxbMCwxLCJNIl0sWzAsMywiXFxtYXRocm17cHJ9XzEiLDJdLFsxLDIsIlxcbWF0aHJte3ByfV8xIiwyXSxbMCwxLCJcXG92ZXJsaW5le1R9Il0sWzMsMiwiVCJdLFswLDIsIiIsMSx7InN0eWxlIjp7Im5hbWUiOiJjb3JuZXIifX1dXQ==
\begin{center}
\begin{tikzcd}[cramped]
  {P} & {\mathrm{EM}_\bullet(\mathbb{Z},1)} \\
  \mm & {\mathrm{EM}(\mathbb{Z},1)}
  \arrow["", from=1-1, to=1-2]
  \arrow["{\mathrm{pr}_1}"', from=1-1, to=2-1]
  \arrow["\lrcorner"{anchor=center, pos=0.125}, draw=none, from=1-1, to=2-2]
  \arrow["{\mathrm{pr}_1}"', from=1-2, to=2-2]
  \arrow["{\mathsf{link}}", from=2-1, to=2-2]
\end{tikzcd}
\end{center}
then we have a bundle of mere circles, with total space given by the \( \sit{} \)-type construction. We will show that this is not quite a principal bundle, i.e. a bundle of torsors. Torsors are types with the additional structure of a free and transitive group action. But if \( \link  \) satisfies an additional property (amounting to an orientation) then the pullback is a principal fibration, i.e. \( \link \) factors through a map \( \K(\zz,2)\to\EMzo \), where \( \K(\zz,2) \) is an Eilenberg-Mac Lane space. 

We will argue that extending \( \link \) to a function \( T \) on paths can be thought of as constructing a connection, together with a ``flatness structure,'' i.e. a proof of flatness. Moreover, lifting \( T \) to \( X:\mm\to\EMp(\zz,1) \) can be thought of as a nonvanishing vector field. There will in general not be a total lift, just a lift on the 1-skeleton of \( \mm \). We will define the ``index'' of \( X \) at a face.

We will then define a method for visiting all the faces of a manifold in order to form ``totals'' of local objects. We will examine the total curvature and the total index and prove that they are equal, and argue that they are equal to the usual Euler characteristic. This will simultaneously prove the Poincaré-Hopf theorem and Gauss-Bonnet theorem in 2 dimensions, for combinatorial manifolds. This is similar to the classical proof of Hopf\cite{hopf}, presented in detail in Needham\cite{needham}.

