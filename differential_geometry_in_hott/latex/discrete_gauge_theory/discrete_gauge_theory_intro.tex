\section{Overview}

The outline is that we will define 
\begin{itemize}
\item principal bundles in Section~\ref{sec:torsors},
\item simplicial complexes, and homotopical realizations of these in Section~\ref{sec:discrete_man},
\item vector fields in Section~\ref{sec:vector_fields},
\end{itemize}
and observe emerging from those definitions the presence of
\begin{itemize}
\item connections and curvature in Section~\ref{sec:connections},
\item the index of a vector field in Section~\ref{sec:totals},
\end{itemize}
and then define in Section~\ref{sec:totals}
\begin{itemize}
\item the total curvature, as in the Gauss-Bonnet theorem
\item the total index of a vector field, as in Poincaré-Hopf theorem,
\item and prove the equality of these to each other.
\end{itemize}

We will build up an example of all of these structures on an octahedron model of the sphere, and compute its Euler characteristic of 2. We will not, however, be supplying a separate definition of Euler characteristic so as to truly reproduce the Gauss-Bonnet and Poincaré-Hopf theorems.

Once we have defined homotopical realizations of simplicial complexes in Section~\ref{sec:discrete_man}, we will focus in most of this note on dimensions 1 and 2. In dimension 1 we obtain polygons, which we prove are equivalent to \( S^1 \), and so give terms in the type \( \EMzo\defeq \sit{A:\uni}||A\simeq S^1||_1 \). We can call this component of the universe ``mere circles.'' In dimension 2 we will focus on a subset of complexes where the neighboring vertices and edges of each vertex (the vertex's ``link'') form a polygon. The homotopical realization \( \mm \) of such a complex then has a map \( \link \) from each vertex to a homotopical polygon, i.e. a map to \( \EMzo \). We do not know under what conditions this map necessarily extends to the higher cells of the realization.

Given a map \( \mm\to\EMzo \) we can form the pullback
% https://q.uiver.app/#q=WzAsNCxbMCwwLCJQXFxzdGFja3JlbHtcXG1hdGhybXtkZWZ9fXs9fVxcc3VtX3tDOlRNfUMiXSxbMSwwLCJcXG1hdGhybXtFTX1fXFxidWxsZXQoXFxtYXRoYmJ7Wn0sMSkiXSxbMSwxLCJcXG1hdGhybXtFTX0oXFxtYXRoYmJ7Wn0sMSkiXSxbMCwxLCJNIl0sWzAsMywiXFxtYXRocm17cHJ9XzEiLDJdLFsxLDIsIlxcbWF0aHJte3ByfV8xIiwyXSxbMCwxLCJcXG92ZXJsaW5le1R9Il0sWzMsMiwiVCJdLFswLDIsIiIsMSx7InN0eWxlIjp7Im5hbWUiOiJjb3JuZXIifX1dXQ==
\begin{center}
\begin{tikzcd}[cramped]
  {P} & {\mathrm{EM}_\bullet(\mathbb{Z},1)} \\
  \mm & {\mathrm{EM}(\mathbb{Z},1)}
  \arrow["", from=1-1, to=1-2]
  \arrow["{\mathrm{pr}_1}"', from=1-1, to=2-1]
  \arrow["\lrcorner"{anchor=center, pos=0.125}, draw=none, from=1-1, to=2-2]
  \arrow["{\mathrm{pr}_1}"', from=1-2, to=2-2]
  \arrow["{\mathsf{link}}", from=2-1, to=2-2]
\end{tikzcd}
\end{center}
to obtain a bundle of mere circles. We will discuss how, if \( \K(\zz,2) \) is an Eilenberg-Mac Lane space and if \( \link \) factors through a map \( \K(\zz,2)\to\EMzo \) then the pullback is a principal fibration.

Then in Section~\ref{sec:connections} we will name various elements of the above construction, indicating their relationship to classical definitions.

In Section~\ref{sec:vector_fields} we will define vector fields, which require a tangent bundle. We will introduce a method for computing vector fields along concatenations of paths.

Finally, in Section~\ref{sec:totals} we will define a method for visiting all the faces of a manifold in order to form ``totals'' of local objects. We will examine the total curvature and the total index and prove that they are equal. Our proof tracks very closely with the classical proof of Hopf\cite{hopf}, presented in detail in Needham\cite{needham}. In their case they can go on to prove that these values are both equal to the Euler characteristic, but we would need an independent definition to prove agreement with, which we do not currently have.

