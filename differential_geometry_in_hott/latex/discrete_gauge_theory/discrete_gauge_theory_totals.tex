\section{The total construction}
\label{sec:totals}
We will place holonomy, flatness, and vector fields on the same footing, and combine them. We will prove the equivalance of the total curvature of a tangent bundle, and total index of a vector field. This is the key relationship in proving both the Gauss-Bonnet theorem and the Poincaré-Hopf theorem.

\subsection{Index of the vector field on a face}
The index of a vector field is derived from other data that we have. Consider a 2-dimensional simplicial complex \( M \) and its 2-dimensional realization \( \mm=\mm_0\to\mm_1\to\mm_2 \). Consider a face \( F:M_2 \), with \( m:\mm \) a vertex of \( F \) and \( \partial F:\partial\Delta^2\to\mm \) the boundary loop of the face. We have accumulated the following constructions:
\[\begin{aligned}
\tr_F&\defeq \tr(\partial F)&&:Tm=Tm&&\text{holonomy/curvature}\\
\flat_F&\defeq \flat(\partial F)&&:\id=_{Tm=Tm}\tr(\partial F)&&\text{flatness}\\
X_F&\defeq X(\partial F)&&:\tr(\partial F)(X(m))=_{Tm}X(m)&&\text{swirling.}\\
\end{aligned}\]
The flatness is a path in automorphisms of \( Tm \) whereas the swirling is a path in \( Tm \), but we can view the latter it as a path in automorphisms as well:
\begin{myprop}
\label{prop:eveq}
Given a polygon \( \ccc(n):\Kzt \) and a point \( b:\ccc(n) \) the evaluation map \( \ev(b):(\ccc(n)=_{\Kzt}\ccc(n))\to \ccc(n) \) is an equivalence.
\end{myprop}
\begin{myproof}
See \cite{buchholtz2023central}.
\end{myproof}

Now \( \sigma^X_F\defeq\ev(X(m))^{-1}(X_F) \) is the automorphism of \( Tm \) that corresponds to the swirling. We now have:
\[\begin{aligned}
\tr_F&:Tm=Tm&&\text{curvature on }F\\
\flat_F&:\id=_{Tm=Tm}\tr_F&&\text{flatness on }F\\
\sigma^X_F&:\tr_F=_{Tm=Tm}\id&&\text{swirling on }F.\\
\end{aligned}\]
These last two can be concatenated to make a loop. We can then obtain an integer by using the fact that the automorphisms are a circle \( \loopy_{Tm}(\Kzt)\simeq S^1 \), plus the well known formula that loops in the circle are integers: \( \loopy(S^1,\base)\simeq\zz \) (e.g. \cite{hottbook} Corollary 8.1.10).
\begin{mydef}
The \defemph{index of the vector field \( X \) on the face \( F \)} is the integer \( I^X_F\defeq\loopy(\flat_F\cdot \sigma^X_F):\loopy(\id=_{Tm=Tm}\id) \).
\end{mydef}
We now have the following final list of ingredients at a single face:
\begin{equation}
\label{eq:face_elements}
\begin{aligned}
\tr_F&:Tm=Tm&&\text{curvature on }F\\
\flat_F&:\id=_{Tm=Tm}\tr_F&&\text{flatness on }F\\
\sigma^X_F&:\tr_F=_{Tm=Tm}\id&&\text{swirling on }F\\
I^X_F&:\zz&&\text{index on }F.
\end{aligned}
\end{equation}

\subsection{Total curvature and index on the sphere}
Now we wish to compute a sum over all faces of the data. To do this we need the following.

\begin{mydef}
An \defemph{oriented set of pointed faces} of a realization \( \mm \) of an oriented 2-dimensional simplicial complex \( M=[M_0,M_1,M_2] \), is a choice \( v_F:F \) of a vertex in each face \( F:M_2 \), and a choice of boundary path \( \ell_F:v_F=v_F \) on each face, satisfying the \defemph{cancellation property}: for each face the boundary path is a concatenation of three edge paths, and for each edge \( e=\{x,y\} \) in \( M_1 \), whose image under realization is denoted \( p_e:x=y \), the path \( p_e \) appears as a concatenation term in exactly one loop, say \( \ell_F \), and \( p_e^{-1} \) appears in exactly one loop \( \ell_{F'} \).
\end{mydef}

We have seen in Lemma~\ref{lem:multpath} how to perform the required sum.
\begin{mydef}
The \defemph{total flatness}, \defemph{total swirling}, and \defemph{total index} on an oriented set of pointed faces is
\begin{equation}
\label{eq:tot_elements}
\begin{aligned}
\tr_\tot      & \defeq \sum_F\tr_F &&:Tm=Tm\\
\flat_\tot    & \defeq \sum_F\flat_F &&:\id=_{Tm=Tm}\tr_\tot\\
\sigma^X_\tot & \defeq \sum_F\sigma^X_F &&:\tr_\tot=_{Tm=Tm}\id\\
I^X_\tot      & \defeq \sum_F I^X_F &&:\zz.
\end{aligned}
\end{equation}
\end{mydef}

The assumption about edges appearing twice, once in each direction, together with Lemma \ref{lem:subtraction} proves the following
\begin{myprop}The total transport is \( 0 \) in \( Tm=Tm \).\label{prop:total_rotation}
\end{myprop}
\begin{myproof}
With a vector field \( X \) chosen, we can compute \[ \sum_F\tr_F=\sum_{(\mathrm{edges\ }e_{ij})} \rho_{ij}+\rho_{ji}=_{Tm=Tm}0 \] and note that the result does not depend on \( X \).
\end{myproof}

And making use of Lemma \ref{lem:cancellation} we obtain
\begin{myprop}The total swirling is \( \refl_0 \) in \( Tm=Tm \).\label{prop:total_swirling}
\end{myprop}
\begin{myproof}
\(\sum_F\sigma^X_F=\sum_{(\mathrm{edges\ }e_{ij})}\sigma_{ij}+\sigma_{ji}=_{Tm=Tm}\refl_0\)
\end{myproof}

\begin{mycor}
Total flatness is a loop: \( \flat_\tot:\id=_{Tm=Tm}\id \).
\end{mycor}\begin{myproof}Follows from Corollary~\ref{prop:total_rotation}.\end{myproof}

And finally, if we give the loop map \( \loopy \) the nickname ``winding number'':

\begin{mythm}\label{thm:total_index_total_curvature}The winding number of total flatness equals the total index.
\end{mythm}
\begin{myproof}
Follows directly from \( I^X_\tot\defeq\loopy(\flat_\tot\cdot \sigma^X_\tot) \) and Proposition~\ref{prop:total_swirling}.
\end{myproof}
