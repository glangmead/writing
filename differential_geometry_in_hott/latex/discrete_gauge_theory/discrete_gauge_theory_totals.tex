\section{Total constructions}
\chcomment[id=Greg]{Moved all ``totaling'' stuff into a section}
We wish to make computations that combine contributions from every face of a combinatorial manifold. To do this we will add some structure and then prove that the definitions are independent of that structure.

\begin{mydef}
A \defemph{total enumeration of faces} for a combinatorial 2-manifold \( \mm \) with underlying simplicial complex \( M=[M_0, M_1, M_2] \) consists of
\begin{itemize}
\item A ``master basepoint'' \( m:M_0 \).
\item For each face \( F:M_2 \) with vertices \( \{v_1, v_2, v_3\} \) an enumeration of its vertices \( [v_1, v_2, v_3] \), including the choice of the first vertex in the enumeration as the basepoint of \( F \), that is \defemph{globally compatible} with the choices for the other faces, meaning that when two faces \( F_1,F_2 \) share an edge \( \{v, w\} \), then one of the faces includes the sublist \( [v, w] \) and the other includes \( [w, v] \).
\item An ordering of the faces \( [F_1,\ldots,F_n] \).
\item For each face \( F \) a path \( p_F:m=_\mm F_v \). This is the only ingredient that uses the type \( \mm \). 
\end{itemize}
\end{mydef}

For example, for the octahedron we might choose \( b \) as the master basepoint, as well as the basepoint for four of the faces. For the other four faces we could choose \( g \) as the basepoint. We could choose \( br\cdot rg \) as the path between the basepoints, and we could order the faces like this: \( [bwo, brw, boy, byr, gow, gwr, gry, gyo] \).

For a single face \( F \), the concatenation \( \ell_F\defeq p_F\cdot e_{12}\cdot e_{23}\cdot e_{31}\cdot p_F^{-1} \) is a loop around \( F \) based at \( m \). The concatenation \( \ell_{F_1}\cdot\cdots\cdot\ell_{F_n} \) is a loop that visits each face once.

\begin{mydef}
Given a total enumeration of faces, the concatenation \( \elltot\defeq\ell_{F_1}\cdot\cdots\cdot\ell_{F_n} \) is called \defemph{the total loop} of the enumeration.
\end{mydef}

\begin{figure}[h]
\centering
\begin{tikzpicture}[arrow/.style={-{Stealth[scale=1.1]}}]
  \tikzset{internal/.style={inner sep=1, outer sep=1, line cap=rect}}
    % Define coordinates for the equilateral triangle
    \node[internal, label=above:\( v_2 \)] (A) at (1, 1.732) {}; % Top vertex
    \node[internal, label=below:\( v_1 \)] (B) at (0, 0) {};     % Left vertex
    \node[internal, label=below:\( v_3 \)] (C) at (2, 0) {};     % Right vertex
    \node (Face) at (1, 0.85) {\( F \)}; % Top vertex

    % Draw the equilateral triangle with black vertices
    \draw[arrow] (A) edge["\( e_{23} \)"] (C);
    \draw[arrow] (B) edge["\( e_{12} \)"] (A);
    \draw[arrow] (C) edge["\( e_{31} \)"] (B);
    
    \node[internal, label=left:\( m \)] (F) at (-3, 0) {};

    % Draw the line segment from B through D, E, F
    \draw[arrow] (F) edge["\( p\to \)"] (B);
    \draw[arrow] (B) edge["\( \leftarrow p^{-1} \)"] (F);
    

    % Add black circles at each vertex of the triangle
    \fill [black] (A) circle (2pt);
    \fill [black] (B) circle (2pt);
    \fill [black] (C) circle (2pt);
    
    \fill [black] (F) circle (2pt);
\end{tikzpicture}
\caption{A loop \( \ell_F:m=_\mm m \) around the face \( F \).}
\label{fig:lasso}
\end{figure}


\subsection{Total holonomy}
Given a total enumeration of faces we can form the compositions \( T\circ\ell_{F_i} \) with the associated loop to obtain transport isomorphisms \( T\circ\ell_{F_i}:Tm=Tm \). We can concatenate all the loops \( \ell_{F_1}\cdot\cdots\cdot\ell_{F_n} \) and obtain a \emph{total holonomy} \( T\circ(\ell_{F_1}\cdot\cdots\cdot\ell_{F_n}):Tm=Tm  \).

\begin{myprop}
The total holonomy is conjugated if we change the paths \( p_{F_i} \) of the face enumerator, or the basepoint \( m \). In particular, if the automorphism groups of the fibers are abelian then the total holonomy is invariant. The total holonomy becomes the inverse automorphism if we change the enumeration of all the faces.
\end{myprop}
\begin{proof}
We will prove that changing a point in some fixed \( p_F \) to an adjacent vertex conjugates the holonomy. We will then prove that moving the master basepoint \( m \) to an adjacent vertex does not affect the result. The full result follows by concatenating such proofs. Figure~\ref{fig:tail_move} shows the image in \( \mm \) of two functions \( f, g \) on a segment of the tail. If \( f \) and \( g \) agree on the rest of the lasso, how is total holonomy affected? We will traverse this segment twice: left to right, then later from right to left. When moving to the right we must compare \( g(e_i)\cdot g(e_{i+1}) \) to \( f(e_i)\cdot f(e_{i+1}) \). There is a filler \( H:f(e_i)\cdot f(e_{i+1})=g(e_i)\cdot g(e_{i+1}) \) and so if the holonomy around the lasso \( f(L) \) is \[ f(e_1)\cdot\cdots\cdot f(e_i)\cdot f(e_{i+1})\cdot\cdots\cdot f(e_{i+1})^{-1}\cdot f(e_i)^{-1}\cdot\cdots f(e_1)^{-1} \] then there is a path \[ f(e_1)\cdot\cdots\cdot H\cdot\cdots\cdot H^{-1}\cdot\cdots f(e_1)^{-1} \] from \( f(L) \) to \( g(L) \).
\end{proof}

\begin{figure}[h]
\centering
\begin{tikzpicture}[arrow/.style={-{Stealth[scale=1.1]}}]
  \tikzset{internal/.style={inner sep=1, outer sep=1, line cap=rect}}
    % Define coordinates for the three additional vertices on the line segment
    \node[internal, label=left:\( \begin{array}{c} \scriptstyle f(p_i)\\\scriptstyle g(p_i)\end{array} \)]  (P1)  at (-7, 0) {};
    \node[internal, label=below:\(\scriptstyle f(p_{i+1}) \)]  (P21) at (-4, 0) {};
    \node[internal, label=above:\( \scriptstyle g(p_{i+1}) \)] (P22) at (-4, 2) {};
    \node[internal, label=right:\( \begin{array}{c} \scriptstyle f(p_{i+2})\\\scriptstyle g(p_{i+2})\end{array} \)]  (P3)  at (-1, 0) {};

    % f
    \draw[arrow] (P1)  edge["\( f(e_i) \)", swap] (P21);
    \draw[arrow] (P21) edge["\( f(e_{i+1})\)", swap]  (P3);

    \draw[arrow] (P1)  edge["\( g(e_i) \)"] (P22);
    \draw[arrow] (P22) edge["\( g(e_{i+1})\)"]  (P3);
    
    % Add black circles at each additional vertex
    \fill [black] (P1) circle (2pt);
    \fill [black] (P21) circle (2pt);
    \fill [black] (P3) circle (2pt);
    \fill [black] (P22) circle (2pt);
\end{tikzpicture}
\caption{Two images of a part of the lasso tail.}
\label{fig:tail_move}
\end{figure}

\begin{myex}
The total holonomy of \( T_1 \) on \( \oo \) (Definition~\ref{def:octahedron_holonomy}) is \( R^8=\) which is two full rotations of \( C_4 \).
\end{myex}



\subsection{Total index of a vector field: subtracting curvature}

\subsection{Equality of total index and total curvature}

\subsection{Identification with Euler characteristic}


