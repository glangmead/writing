\section{Total constructions}
We wish to make computations that combine contributions from every face of a combinatorial 2-manifold. To do this we will add some structure and then prove that the definitions are independent of that structure.

\begin{mydef}
A \defemph{total enumeration of faces} for a combinatorial 2-manifold \( \mm \) with underlying simplicial complex \( M=[M_0, M_1, M_2] \) consists of
\begin{itemize}
\item A ``master basepoint'' \( m:M_0 \).
\item For each face \( F:M_2 \) with vertices \( \{v_1, v_2, v_3\} \) an enumeration of its vertices \( [v_1, v_2, v_3] \), including the choice of the first vertex in the enumeration as the basepoint of \( F \), that is \defemph{globally compatible} with the choices for the other faces, meaning that when two faces \( F_1,F_2 \) share an edge \( \{v, w\} \), then one of the faces includes the sublist \( [v, w] \) and the other includes \( [w, v] \).
\item An ordering of the faces \( [F_1,\ldots,F_n] \).
\item For each face \( F \) a path \( p_F:m=_\mm F_v \) and a loop \( \ell_F\defeq p_F\cdot e_{12}\cdot e_{23}\cdot e_{31}\cdot p_F^{-1} \) where \( e_{ij}:v_i=v_j \) are the paths corresponding to the edges. 
\end{itemize}
\end{mydef}

Note that the last bullet of the definition is the only one that uses the realization \( \mm \).

\begin{mynote}
For such an enumeration to exist the underlying simplicial complex must be \emph{orientable} in a classical sense. We are not going to explore this requirement internally in HoTT, nor prove any relationship between orientability of the set-based complex and orientation in the sense of factoring classifying maps through \( \Kzt\to\EMzo \).
\end{mynote}

\begin{myex}
The octahedron: for \( \oo \) we might choose \( b \) as the master basepoint, as well as the basepoint for four of the faces. For the other four faces we could choose \( g \) as the basepoint. We could choose \( br\cdot rg \) as the path between the basepoints, and we could order the faces like this: \( [bwo, brw, boy, byr, gow, gwr, gry, gyo] \).
\end{myex}

\begin{mydef}
Given a total enumeration of faces, the concatenation \( \elltot\defeq\ell_{F_1}\cdot\cdots\cdot\ell_{F_n} \) is called \defemph{the total loop} of the enumeration.
\end{mydef}

\begin{figure}[h]
\centering
\begin{tikzpicture}[arrow/.style={-{Stealth[scale=1.1]}}]
  \tikzset{internal/.style={inner sep=1, outer sep=1, line cap=rect}}
    \node[internal, label=above:\( v_2 \)] (V2) at (1, 1.732) {};
    \node[internal, label=below:\( v_1 \)] (V1) at (0, 0) {};
    \node[internal, label=below:\( v_3 \)] (V3) at (2, 0) {};
    \node (Face) at (1, 0.85) {\( F \)};

    \draw[arrow] (V2) edge["\( e_{23} \)"] (V3);
    \draw[arrow] (V1) edge["\( e_{12} \)"] (V2);
    \draw[arrow] (V3) edge["\( e_{31} \)"] (V1);
    
    \node[internal, label=left:\( m \)] (m) at (-3, 0) {};

    \draw[arrow] (m) edge["\( p\to \)"] (V1);
    \draw[arrow] (V1) edge["\( \leftarrow p^{-1} \)"] (m);
    
    \fill [black] (V1) circle (2pt);
    \fill [black] (V2) circle (2pt);
    \fill [black] (V3) circle (2pt);
    \fill [black] (m) circle (2pt);
\end{tikzpicture}
\caption{A loop \( \ell_F:m=_\mm m \) around the face \( F \).}
\label{fig:lasso}
\end{figure}


\subsection{Total holonomy}
Given a total enumeration of faces we can form the compositions \( T\circ\ell_{F_i} \) with the associated loop to obtain holonomy isomorphisms \( T\circ\ell_{F_i}:Tm=Tm \). We can concatenate all the loops \( \ell_{F_1}\cdot\cdots\cdot\ell_{F_n} \) and obtain a \emph{total holonomy} \( T\circ(\ell_{F_1}\cdot\cdots\cdot\ell_{F_n}):Tm=Tm  \).

\begin{mydef}
The \defemph{total holonomy} of a total enumeration of faces is the map \( T\circ(\ell_{F_1}\cdot\cdots\cdot\ell_{F_n}):Tm=Tm  \).
\end{mydef}

\begin{myprop}
The holonomy \( \ell_F \) around a face \( F \) is conjugated if we change the path \( p_F \) or the basepoint \( m \). The holonomy becomes the inverse automorphism if we change the enumeration of \( F \) by an odd permutation of its vertices (but keep the basepoint fixed).
\end{myprop}
\begin{proof}
If we change the path \( p_F \) to a homotopic path \( p'_F \), then the change is made up of a finite sequence of moves like the one in Figure~\ref{fig:tail_move}, where the triangle \( xyz \) bounds a face. Then \( p_{F} \) is given by 
\[ (p\cdot e_{xy})\cdot (q\cdot e_{12}\cdot e_{23}\cdot e_{31}\cdot q^{-1})\cdot (p\cdot e_{xy})^{-1}\] 
where we grouped the parts that involve the change and do not involve it. So replacing \( e_{xy} \) by \( e_{xz}\cdot e_{zy} \) is definitionally equal to 
\[ A\cdot (p\cdot e_{xy})\cdot (q\cdot e_{12}\cdot e_{23}\cdot e_{31}\cdot q^{-1})\cdot (p\cdot e_{xy})^{-1}\cdot A^{-1}\] 
where \( A\defeq p\cdot e_{xz}\cdot e_{zy}\cdot e_{xy}^{-1}\cdot p^{-1} \) is the loop that circulates the difference between the paths. Said another way, we are making use of the observation that \( p_F \) is related to \( p'_F \) by conjugating with the loop, based at \( m \), that goes around the discrepancy. If we add more such basic changes then we add additional conjugations.

If we change the basepoint \( m \) to \( m' \) then we need to make a new choice, which is a path \( \pi:m'=m \). This lengthens the path \( p_F \) to the path \( \pi\cdot p_F\cdot \pi^{-1} \). This conjugation relates two different groups, the automorphism groups \( Tm=Tm \) and \( Tm'=Tm' \). The two groups are not canonically identified because the identification relies on the choice of \( \pi \). If we change \( \pi \) by a sequence of moves like we did with \( p_F \) then again the holonomy at \( m' \) will change by a conjugation.

Finally we observe that for the triangle \( F \), an odd permutation that keeps the basepoint fixed is simply a traversal of the loop in the opposite direction, sending \( \ell_F \) to \( \ell_{F^{-1}} \).
\end{proof}

\begin{figure}[h]
\centering
\begin{tikzpicture}[arrow/.style={-{Stealth[scale=1.1]}}]
  \tikzset{internal/.style={inner sep=1, outer sep=1, line cap=rect}}
    \node[internal, label=above:\( v_2 \)] (V2) at (1, 1.732) {};
    \node[internal, label=below:\( v_1 \)] (V1) at (0, 0) {};
    \node[internal, label=below:\( v_3 \)] (V3) at (2, 0) {};
    \node (Face) at (1, 0.85) {\( F \)};

    \draw[arrow] (V2) edge["\( e_{23} \)"] (V3);
    \draw[arrow] (V1) edge["\( e_{12} \)"] (V2);
    \draw[arrow] (V3) edge[swap, "\( e_{31} \)"] (V1);
    
    \node[internal, label=left:\( m \)] (m) at (-4, 0) {};
    \node[internal, label=below:\( x \)] (x) at (-3, 0) {};
    \node[internal, label=below:\( y \)] (y) at (-1, 0) {};
    \node[internal, label=above:\( z \)] (z) at (-2, 1.732) {};

    \draw[arrow] (m) edge["\( p \)"] (x);
    \draw[arrow] (x) edge["\( e_{xy} \)"] (y);
    \draw[arrow, dashed] (x) edge["\( e_{xz} \)"] (z);
    \draw[arrow, dashed] (z) edge["\( e_{zy} \)"] (y);
    \draw[arrow] (y) edge["\( q \)"] (V1);

    \fill [black] (V1) circle (2pt);
    \fill [black] (V2) circle (2pt);
    \fill [black] (V3) circle (2pt);
    \fill [black] (m) circle (2pt);
    \fill [black] (x) circle (2pt);
    \fill [black] (y) circle (2pt);
    \fill [black] (z) circle (2pt);
\end{tikzpicture}
\caption{Two paths to the face, from \( x \) to \( y \) directly, or through \( z \).}
\label{fig:tail_move}
\end{figure}

\begin{mycor}
In a bundle of mere circles \( T:\mm\to\EMzo \) the total holonomy is independent of the choice of paths \( p_{F_i} \) and the basepoint \( m \). It is inverted if we switch orientations.
\end{mycor}

\begin{myex}
The total holonomy of \( T_1 \) on \( \oo \) (Definition~\ref{def:octahedron_holonomy}) is \( R^8=\) which is two full rotations of \( C_4 \).
\end{myex}

\subsection{Total index of a vector field}
We defined the index on a face in Definition~\ref{def:index}. We will extend this to a total index defined with a total enumeration of faces.

\subsection{The main theorem}
\begin{mythm}
The total index is constant on \( \pit{x:\mm_1}T_1(x) \).
\end{mythm}

This appears to be a little less than what we have classically, which is:
\begin{enumerate}
\item A definition of total curvature of a connection on the tangent bundle.
\item A definition of total index, without using the connection.
\item A definition of Euler characteristic.
\item A proof that total index is independent of the vector field.
\item A proof that total curvature is equal to total index.
\end{enumerate}
This collection of results provides enough redundancy to separately produce corollaries such as: total curvature is an integer; total curvature does not depend on the connection; a second proof that total index is independent of the vector field (since it is equal to total curvature, which does not so depend).

How much of this do we have?
