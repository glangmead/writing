\section{The total construction}
\chcomment[id=Greg]{All new}
We will place holonomy, flatness, and vector fields on the same footing, and combine them. We will prove the equivalance of the total curvature of a tangent bundle, and total index of a vector field. This is the key relationship in proving both the Gauss-Bonnet theorem and the Poincaré-Hopf theorem.

\subsection{Index of the vector field}
Given a polygon cellular type \( C_{n,0}\to C_{n,1}=C_n \), and map \( *_{C_n}:\unit\to\Kzt \) giving it the structure of an \( S^1 \)-torsor, we have the following general fact:
\begin{myprop}
\label{prop:eveq}
Given a pointing \( b:C_n \) the evaluation map \( \ev(b):(C_n=C_n)\to C_n \) is an equivalence.
\end{myprop}

Consider the cellular type \( \mm_0\to\mm_1\to\mm \) with point \( m:\mm \). Consider a loop \( \ell:m=_\mm m \) with proof of contractibility \( f:\ell=\refl_m \). For example, we may have a face \( F \) in a combinatorial realization, with \( m \) a vertex of \( F \) and \( \ell=\partial F \) the boundary loop. We have accumulated the following constructions:
\[\begin{aligned}
\tr(\ell)&:Tm=Tm\\
\flat(\ell)&:\tr(\ell)=_{Tm=Tm}\id\\
X(\ell)&:\tr(\ell)(X(m))=_{Tm=Tm}X(m)\\
\end{aligned}\]
which invites us to make use of the equivalence \ref{prop:eveq} to define \( X_E(\ell)\defeq \ev(X(m))^{-1}\circ X(\ell):\tr(\ell)=\id \) (where the subscript \( E \) stands for the extension to all of \( Tm \)) to obtain
\[
\begin{aligned}
\tr(\ell)&:Tm=Tm\\
\flat(\ell)&:\tr(\ell)=_{Tm=Tm}\id\\
X_E(\ell)&:\tr(\ell)=_{Tm=Tm}\id\\
\end{aligned}
\]
These last two can be concatenated to make a loop.
\begin{mydef}
The \defemph{index of the vector field \( X \) around the loop \( \ell \)} is the integer \( I_X(\ell):\flat(\ell)^{-1}\cdot X_E:\id=_{Tm=Tm}\id \).
\end{mydef}
\begin{mynote}
Classically the index around a loop makes use of a trivialization of the tangent bundle, and does not take place in the presence of a connection. We know from section~\ref{sec:localtriv} that the connection can serve as such a trivialization. Taking that point of view, the formula constitutes the difference between the swirling of the vector field inside the chart, and the twisting of the chart itself. 
\end{mynote}
We now have the following list of ingredients given the loop \( \ell \):
\begin{equation}
\label{eq:face_elements}
\begin{aligned}
\tr(\ell)&:Tm=Tm\\
\flat(\ell)&:\tr(\ell)=_{Tm=Tm}\id\\
X_E(\ell)&:\tr(\ell)=_{Tm=Tm}\id\\
I_X(\ell)&:\id=_{Tm=Tm}\id\\
\end{aligned}
\end{equation}

\subsection{Cancellations}
The next observation is that we can relate the quantities in \ref{eq:face_elements} at different points around a cellular surface by making use of the automorphism bundle \( \sit{x:\mm}(Tx=Tx) \) and its trivialization \( \tau:\pit{x:M}(Tx=Tx)\xrightarrow[]{\simeq}(Tm=Tm) \) as in Lemma~\ref{lem:gauge_triv}.

\begin{mydef}
Consider the contractible type \( \sit{\alpha:Tm=Tm}\alpha=\id \) with the addition law \( (\alpha, p)+(\beta, q) = (\alpha\cdot\beta, p\cdot \alpha(q)) \) where \( \alpha(q):\alpha=\alpha\cdot\beta \) (using concatenation notation instead of function composition). This is a commutative group with identity \( \refl_\id \) which we call the \defemph{type of angles of \( Tm \)}.
\end{mydef}

\begin{mylemma}
\( \tau\circ X \) maps concatenation of paths to the commutative operation \( + \) in the type of angles \( \sit{\alpha:Tm=Tm}\alpha=\id \).
\end{mylemma}
\begin{proof}
Suppose we have \( p:x=_\mm y \) and \( q:y=_\mm z \). Then \( X(p\cdot q)=\tr(q)(X(p))\cdot X(q) \) which is exactly the operation \( + \) but in the group \( \sit{\alpha:Tz=Tz}\alpha=\id \). We can then transport the result to \( m \) by an arbitrary path.
\end{proof}

\begin{mycor}
\label{cor:x_cancel}
With \( e, p, \ell \) as in Figure~\ref{fig:lasso}, we have \(\tau(X_E(e\cdot\ell\cdot e^{-1})) = \tau(X_E(\ell))\).
\end{mycor}

So under \( \tau \) we can cancel the contributions to \( X \) from the edge \( e \) after traversing it once in each direction. Meanwhile we have something similar for the transport automorphisms:

\begin{mylemma}
With \( e, p, \ell \) as in Figure~\ref{fig:lasso}, we have \( \tau(\tr(e\cdot\ell\cdot e^{-1}))=\tau(\tr(\ell)) \).
\end{mylemma}
\begin{proof}
Recall that \( \tau \) conjugates automorphisms by transport along some arbitrary path to \( m \). To transport automorphisms from \( x \) to \( p \), choose a path \( p:m=_\mm x \). Then make the specific choice \( p\cdot e:m=_\mm y \) to transport automorphisms from \( y \) to \( p \). Then \( \tau(\tr(e\cdot\ell\cdot e^{-1})) \) and \( \tau(\tr(\ell)) \) both compute to \( \tr((p\cdot e)\cdot\ell\cdot(p\cdot e)^{-1}) \). 
\end{proof}

\begin{figure}[h]
\centering
\begin{tikzpicture}[arrow/.style={-{Stealth[scale=1.1]}}]
  \tikzstyle{internal}=[inner sep=0, outer sep=0, line cap=rect]
    \coordinate (V2) at (1, 1.732) {};
    \node[label=below:\( y \)] (y) at (0, 0) {};
    \coordinate (V3) at (2, 0) {};

    \draw (V2) edge["\( \ell \)"] (V3);
    \draw (y) -- (V2);
    \draw[arrow] (V3) -- (y);
    
    \node[label=below:\( m \)] (m) at (-3, 0) {};
    \node[label=below:\( x \)] (x) at (-1.5, 0) {};

    \draw[arrow] (m) edge["\( p\)"] (x);
    \draw[arrow] (x) edge["\( e\)"] (y);
    
    \fill [black] (y) circle (2pt);
    \fill [black] (m) circle (2pt);
    \fill [black] (x) circle (2pt);
\end{tikzpicture}
\caption{Comparing maps on the edge \( e \) at the master point \( m \) with and without traversing \( \ell \).}
\label{fig:lasso}
\end{figure}

\subsection{Total enumeration of faces}
\begin{mydef}
A \defemph{total enumeration of faces} for a combinatorial 2-manifold \( \mm \) with underlying simplicial complex \( M=[M_0, M_1, M_2] \) consists of
\begin{enumerate}
\item A ``master basepoint'' \( m:M_0 \).
\item For each face \( F:M_2 \) with vertices \( \{v_{F,1}, v_{F,2}, v_{F,3}\} \) an enumeration of its vertices \( [v_{F,1}, v_{F,2}, v_{F,3}] \), including the choice of the first vertex in the enumeration as the basepoint of \( F \), which is \defemph{globally compatible} with the choices for the other faces, meaning that when two faces \( F_1,F_2 \) share an edge \( \{v, w\} \), then one of the faces includes the sublist \( [v, w] \) and the other includes \( [w, v] \).
\item An ordering of the faces \( [F_1,\ldots,F_n] \).
\end{enumerate}
Two enumerations that differ only in the ordering of vertices (2.) are said to have the \defemph{same orientation} if it is true for every face that the two orderings of vertices differ by an even orientation.
\end{mydef}

\begin{mynote}
For such an enumeration to exist the underlying simplicial complex must be \emph{orientable} in a classical sense. We are not going to explore this requirement internally in HoTT, nor prove any relationship between orientability of the set-based complex and orientation in the sense of factoring classifying maps through \( \Kzt\to\EMzo \).
\end{mynote}

\begin{myex}
The octahedron: for \( \oo \) we might choose \( b \) as the master basepoint, as well as the basepoint for four of the faces. For the other four faces we could choose \( g \) as the basepoint. We could choose \( br\cdot rg \) as the path between the basepoints, and we could order the faces like this: \( [bwo, brw, boy, byr, gow, gwr, gry, gyo] \).
\end{myex}

\begin{mydef}
The \defemph{total path} of an enumeration is the list \( \ell_{F_1},\ldots,\ell_{F_n} \) of loops where \( \ell_{F_i}:v_{F_i,1}=v_{F_i,1} \) is the loop around \( F_i \) connecting vertices \( v_{F,1}, v_{F,2}, v_{F,3} \).
\end{mydef}

\begin{mydef}
Using \ref{eq:face_elements} we can define the \defemph{total curvature}, \defemph{total swirling} and \defemph{total index} of a total path as follows:
\begin{equation}
\label{eq:total_elements}
\begin{aligned}
\flat_\mathrm{tot}&=\tau(\flat(\ell_{F_1}))+\cdots+\tau(\flat(\ell_{F_n}))\\
X_\mathrm{tot}&=\tau(X(\ell_{F_1}))+\cdots+\tau(X(\ell_{F_n}))\\
I_\mathrm{tot}&=\flat_\mathrm{tot}^{-1}\cdot X_\mathrm{tot}
\end{aligned}
\end{equation}
where the sums are taken in the group of angles \( \sit{\alpha:Tm=Tm}\alpha=\id \).
\end{mydef}

\begin{mylemma}
\label{lem:edges_cancel}
The total path of an enumeration visits each edge an even number of times, equally in each direction.
\end{mylemma}

\begin{mythm}
\( \tau(X(\ell_{F_1}))\cdots\tau(X(\ell_{F_n}))=\refl \) in \( \sit{\alpha:Tm=Tm}\alpha=\id \). 
\end{mythm}
\begin{proof}
Immediate from Lemma~\ref{lem:edges_cancel} and Corollary~\ref{cor:x_cancel}.
\end{proof}

\begin{mycor}
The total index is equal to the opposite of the total curvature. The total curvature is an integer.
\end{mycor}
