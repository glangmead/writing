\documentclass[13pt,aspectratio=169,hyperref={hypertexnames = false},handout]{beamer}
\usetheme{default}

% something tries to renewcommand these but beamer must break the assumption that they are defined so I am defining them
\newcommand{\labelitemi}{}
\newcommand{\labelitemii}{}
\newcommand{\labelitemiii}{}
\newcommand{\labelitemiv}{}

\renewcommand{\mathrm}[1]{\mathsf{#1}}
\usepackage{greg}
\usepackage{array}
\renewcommand{\defemph}[1]{\alert{#1}}

\useinnertheme{circles}
\setbeamertemplate{navigation symbols}{}
%\setbeamercovered{transparent}
\definecolor{cmu_red}{RGB}{176,28,51} % #990000
\definecolor{cmu_dark_grey}{RGB}{70, 70, 70} % #464646
\definecolor{mygrey}{RGB}{100, 100, 100}
\definecolor{steveblue}{RGB}{51, 51, 172}
\usecolortheme[named=cmu_red]{structure}
\setbeamercolor{palette secondary}{bg=steveblue,fg=white}
\setbeamercolor{palette tertiary}{bg=steveblue,fg=white}
\setbeamercolor{section in head/foot}{bg=mygrey,fg=white}

\AtBeginSection[]{
  \begin{frame}
  \vfill
  \centering
  \begin{beamercolorbox}[sep=8pt,center,shadow=true,rounded=true]{title}
    \usebeamerfont{title}\insertsectionhead\par%
  \end{beamercolorbox}
  \vfill
  \end{frame}
}

% \setbeamercolor{corollary title}{use=structure,fg=white,bg=structure.fg!75!black}
% \setbeamercolor{corollary body}{parent=normal text,use=block title,bg=blue}
%\setlength{\parskip}{\baselineskip} 

%%
%% FOR PRINTING, COMMENT THESE
%%
% \logo{\vspace{-4pt}\includegraphics[width=100pt]{figs/cmu-wordmark-horizontal-r.pdf}\quad}
% \useoutertheme[subsection=false]{miniframes}
% \setbeamercolor{alerted text}{fg=cmu_red}

%%
%% FOR PRINTING, UNCOMMENT THESE AND ADD "handout" to beamer options in line 1
%%
\setbeamercolor{alerted text}{fg=black}
\setbeamerfont{alerted text}{series=\ifmmode\else\bfseries\fi}

\newcommand{\nologo}{\setbeamertemplate{logo}{}}
\title[Geometry in HoTT]{Discrete differential geometry in homotopy type theory}
\author{Greg Langmead}
\institute[CMU]{Carnegie Mellon University}
\date{April 2025}

\begin{document}
\begin{frame}
\titlepage
\end{frame}

\newlength{\mylen}
\newlength{\mylin}

% \section{Summary}

\begin{frame}{Summary}
This work brings to HoTT
\begin{itemize}
\item connections, curvature, and vector fields
\item the index of a vector field
\item a theorem in dimension 2 that total curvature = total index
\end{itemize}
\end{frame}

\begin{frame}{Classical \( \to \) HoTT}
Let \( M \) be a smooth, oriented 2-manifold without boundary, \( F_A \) the curvature of a connection \( A \) on the tangent bundle, and \( X \) a vector field with isolated zeroes \( x_1,\ldots,x_n \).
% https://q.uiver.app/#q=WzAsNSxbMCwwLCJcXGRpc3BsYXlzdHlsZXtcXGZyYWN7MX17MlxccGl9XFxpbnRfTSBGX0F9Il0sWzEsMCwiXFxkaXNwbGF5c3R5bGV7XFxzdW1fe2k9MX1ebiBcXG1hdGhybXtpbmRleH1fWCh4X2kpfSJdLFswLDEsIlxcZGlzcGxheXN0eWxle1xcT21lZ2FcXGxlZnQoXFxzdW1fe1xcbWF0aHJte2ZhY2VzfVxcIGZ9XFxmbGF0X2ZcXHJpZ2h0KX0iXSxbMSwxLCJcXGRpc3BsYXlzdHlsZXtcXHN1bV97XFxtYXRocm17ZmFjZXN9fUleWF9mfSJdLFsyLDAsIlxcY2hpKE0pIl0sWzAsMSwiIiwwLHsibGV2ZWwiOjIsInN0eWxlIjp7ImhlYWQiOnsibmFtZSI6Im5vbmUifX19XSxbMCwyXSxbMSwzXSxbMiwzLCIiLDIseyJsZXZlbCI6Miwic3R5bGUiOnsiaGVhZCI6eyJuYW1lIjoibm9uZSJ9fX1dLFsxLDQsIiIsMix7ImxldmVsIjoyLCJzdHlsZSI6eyJoZWFkIjp7Im5hbWUiOiJub25lIn19fV1d
\[\begin{tikzcd}[ampersand replacement=\&,column sep=tiny]
  {\displaystyle{\frac{1}{2\pi}\int_M F_A}} \& {\displaystyle{\sum_{i=1}^n \mathrm{index}_X(x_i)}} \& {\chi(M)} \\
  {\displaystyle{\sum_{\mathrm{faces}\ F}\flat_F}} \& {\displaystyle{\sum_{\mathrm{faces}\ F}L^X_F}}
  \arrow[equals, from=1-1, to=1-2]
  \arrow[from=1-1, to=2-1]
  \arrow[equals, from=1-2, to=1-3]
  \arrow[from=1-2, to=2-2]
  \arrow[equals, from=2-1, to=2-2]
\end{tikzcd}\]
\end{frame}

\begin{frame}{Classical index}
Near an isolated zero there are only three possibilities: index 0, 1, --1.

Index is the winding number of the field as you move clockwise around the zero.
\begingroup
\colorlet{veccol}{cmu_red}
\colorlet{myblue}{blue!60!black}
\tikzstyle{vector}=[->,thick,veccol,style={-{Stealth[scale=0.9]}}]
\pgfmathsetmacro{\R}{1}%
\pgfmathsetmacro{\r}{0.03}%
\pgfmathsetmacro{\N}{8}%
\pgfmathsetmacro{\ang}{60}%
\pgfmathsetmacro{\RR}{0.5}%
\[
% zero
\begin{tikzpicture}
  \fill[myblue] (0,0) circle (\r);
  \foreach \x/\y in {-1/0,-1/1,0/1,1/1,1/0,-1/-1,0/-1,1/-1}{
    \draw[vector] (\x*0.5*\R,\y*0.5*\R) ++ (\ang-180:\R/2) --++ (\ang:\R);
  }
  \node at (0,-1.35*\R) {index 0};
\end{tikzpicture}\,\,
% outward
\begin{tikzpicture}
  \fill[myblue] (0,0) circle (\r);
  \foreach \i [evaluate={\ang=\i*360/\N;}] in {0,...,\N}{
    \draw[vector] (\ang:0.1*\R) --++ (\ang:\R);
  }
  \node at (0,-1.35*\R) {index +1};
\end{tikzpicture}\,\,
% inward
\begin{tikzpicture}
  \fill[myblue] (0,0) circle (\r);
  \foreach \i [evaluate={\ang=\i*360/\N;}] in {0,...,\N}{
    \draw[vector] (\ang:1.1*\R) -- (\ang:0.1*\R);
  }
  \node at (0,-1.35*\R) {index +1};
\end{tikzpicture}\,\,
% counterclock
\begin{tikzpicture}
  \fill[myblue] (0,0) circle (\r);
  \foreach \R in {0.44,0.88}{
    \foreach \i [evaluate={\ang=\i*360/\N;}] in {1,...,\N}{
      \draw[vector] (\ang:\R) --++ (\ang+90:\RR);
    }
  }
  \node at (0,-1.3*\R) {index +1};
\end{tikzpicture}\,\,
% clockwise
\begin{tikzpicture}
  \fill[myblue] (0,0) circle (\r);
  \foreach \R in {0.44,0.88}{
    \foreach \i [evaluate={\ang=\i*360/\N;}] in {1,...,\N}{
      \draw[vector] (\ang:\R) --++ (\ang-90:\RR);
    }
  }
  \node at (0,-1.3*\R) {index +1};
\end{tikzpicture}\,\,
% minus one
\pgfmathsetmacro{\r}{0.03}%
\pgfmathsetmacro{\N}{20}%
\pgfmathsetmacro{\ang}{10}%
\begin{tikzpicture}
  \fill[myblue] (0,0) circle (\r);
  \foreach \R in {0.88}{
    \foreach \i [evaluate={\ang=\i*360/\N;}] in {1,...,\N}{
      \draw[vector] (\ang:\R) --++ (-\ang-90:\RR);
    }
  }
  \node at (0,-1.3*\R) {index --1};
\end{tikzpicture}

\]
\endgroup
% \includegraphics[width=420pt]{figs/needham_indexes.pdf}
\end{frame}

\begin{frame}{Poincaré-Hopf theorem}
The total index of a vector field is the Euler characteristic.

Examples:
\vspace{3ex}
\begin{columns}
\begin{column}{0.5\textwidth}
\centering
\includegraphics[width=20ex]{figs/sphere_vf_rotate.pdf}

Rotation: index +1 at each pole = \textbf{2}
\end{column}

\begin{column}{0.5\textwidth}
\centering
\includegraphics[width=20ex]{figs/sphere_vf_morse.pdf}

Height: index +1 at each pole = \textbf{2}
\end{column}
\end{columns}
\end{frame}

\begin{frame}{Gauss-Bonnet theorem}
Total curvature divided by \( 2\pi \) is the Euler characteristic.

Curvature in 2D is a function \( F_A:M\to \rr \).

\( \int_M F_A \) sums the values at every point. 
\begin{columns}
\begin{column}{0.58\textwidth}
\centering
\includegraphics[height=20ex]{figs/torus_gauss_bonnet.pdf}

Positive and negative curvature cancel: \textbf{0}
\end{column}
\begin{column}{0.42\textwidth}
\centering
\includegraphics[height=20ex]{figs/sphere_curvature.pdf}

Constant curvature 1, area \( 4\pi \): \textbf{2}
\end{column}
\end{columns}
\end{frame}

\begin{frame}{Plan}
\begin{itemize}
\item Combinatorial manifolds
\item Torsors and classifying maps
\item Connections and curvature
\item Vector fields
\item Main theorem
\end{itemize}
\end{frame}

\begin{frame}{HoTT background}
\begin{enumerate}
\item \alert{Symmetry},\\
Bezem,~M., Buchholtz,~U., Cagne,~P., Dundas,~B.~I., and Grayson,~D.~R., (2021-) \\
\url{https://github.com/UniMath/SymmetryBook}.

\item \alert{Central H-spaces and banded types},\\
Buchholtz,~U., Christensen,~J.~D. , Flaten,~J.~G.~T., and Rijke,~E. (2023) \\
arXiv:2301.02636

\item \alert{Nilpotent types and fracture squares in homotopy type theory},\\
Scoccola,~L. (2020) \\
MSCS 30(5). arXiv:1903.03245
\end{enumerate}
\end{frame}


% \section{Combinatorial manifolds}

\begin{frame}{Manifolds in HoTT}
\begin{itemize}
\item Recall the classical theory of \alert{simplicial complexes}
\item Define a \alert{realization} procedure to construct types
\end{itemize}
\end{frame}

\begin{frame}{Simplicial complexes}
\begin{columns}
\begin{column}{0.5\textwidth}
\begin{mydef}
An \defemph{abstract simplicial complex \( M \) of dimension \( n \)} is an ordered list of sets \( M\defeq[M_0,\ldots,M_n] \) consisting of 
\begin{itemize}
\item a set \( M_0 \) of vertices
\item sets \( M_k \) of subsets of \( M_0 \) of cardinality \( k+1 \)
\item downward closed: if \( F\in M_k \) and \( G\subseteq F \), \( |G|=j+1 \) then \( G\in M_j \)
\end{itemize}
We call the truncated list \( M_{\leq k}\defeq [M_0,\ldots,M_k] \) \alert{the \( k \)-skeleton of \( M \)}.
\end{mydef}
\end{column}
\begin{column}{0.5\textwidth}
\resizebox{220pt}{!}{
\onslide<2->{
\begin{tikzpicture}
  \matrix (A) [matrix of math nodes, row sep=1cm, column sep=-.2cm]
  { M_2\\ M_1\\ M_0\\ };
\end{tikzpicture}
\begin{tikzpicture}
    \matrix (A) [matrix of math nodes, row sep=1cm, column sep=-.2cm]
    { 
  ~ &  ~ & \scriptstyle\{w, b, r\} & \scriptstyle\{w, r, g\}  & \scriptstyle\{w, g, o\} & \scriptstyle\{w, o, b\} & \scriptstyle\{y, b, r\} & \scriptstyle\{y, r, g\}  & \scriptstyle\{y, g, o\} & \scriptstyle\{y, o, b\}\\
  \scriptstyle\{w, b\} & \scriptstyle\{w,r\}  & \scriptstyle\{w,g\} & \scriptstyle\{w,o\} & \scriptstyle\{b, r\} & \scriptstyle\{r, g\}  & \scriptstyle\{g, o\} & \scriptstyle\{o, b\} & \scriptstyle\{y, b\} & \scriptstyle\{y,r\}  & \scriptstyle\{y,g\} & \scriptstyle\{y,o\}\\
  ~ & ~ & ~ &  \scriptstyle w  & \scriptstyle b & \scriptstyle r & \scriptstyle g & \scriptstyle o & \scriptstyle y\\
    };

    \draw (A-1-3.south)--(A-2-1.north);
    \draw (A-1-3.south)--(A-2-2.north);
    \draw (A-1-3.south)--(A-2-5.north);

    \draw (A-1-4.south)--(A-2-2.north);
    \draw (A-1-4.south)--(A-2-3.north);
    \draw (A-1-4.south)--(A-2-6.north);

    \draw (A-1-5.south)--(A-2-3.north);
    \draw (A-1-5.south)--(A-2-4.north);
    \draw (A-1-5.south)--(A-2-7.north);

    \draw (A-1-6.south)--(A-2-4.north);
    \draw (A-1-6.south)--(A-2-1.north);
    \draw (A-1-6.south)--(A-2-8.north);

    \draw (A-1-7.south)--(A-2-5.north);
    \draw (A-1-7.south)--(A-2-10.north);
    \draw (A-1-7.south)--(A-2-9.north);

    \draw (A-1-8.south)--(A-2-6.north);
    \draw (A-1-8.south)--(A-2-11.north);
    \draw (A-1-8.south)--(A-2-10.north);

    \draw (A-1-9.south)--(A-2-7.north);
    \draw (A-1-9.south)--(A-2-12.north);
    \draw (A-1-9.south)--(A-2-11.north);

    \draw (A-1-10.south)--(A-2-8.north);
    \draw (A-1-10.south)--(A-2-9.north);
    \draw (A-1-10.south)--(A-2-12.north);

    \draw (A-2-1.south)--(A-3-4.north);
    \draw (A-2-1.south)--(A-3-5.north);

    \draw (A-2-2.south)--(A-3-4.north);
    \draw (A-2-2.south)--(A-3-6.north);

    \draw (A-2-3.south)--(A-3-4.north);
    \draw (A-2-3.south)--(A-3-7.north);

    \draw (A-2-4.south)--(A-3-4.north);
    \draw (A-2-4.south)--(A-3-8.north);

    \draw (A-2-5.south)--(A-3-5.north);
    \draw (A-2-5.south)--(A-3-6.north);

    \draw (A-2-6.south)--(A-3-6.north);
    \draw (A-2-6.south)--(A-3-7.north);

    \draw (A-2-7.south)--(A-3-7.north);
    \draw (A-2-7.south)--(A-3-8.north);

    \draw (A-2-8.south)--(A-3-8.north);
    \draw (A-2-8.south)--(A-3-5.north);

    \draw (A-2-9.south)--(A-3-9.north);
    \draw (A-2-9.south)--(A-3-5.north);

    \draw (A-2-10.south)--(A-3-9.north);
    \draw (A-2-10.south)--(A-3-6.north);

    \draw (A-2-11.south)--(A-3-9.north);
    \draw (A-2-11.south)--(A-3-7.north);

    \draw (A-2-12.south)--(A-3-9.north);
    \draw (A-2-12.south)--(A-3-8.north);
\end{tikzpicture}
}
}
\onslide<2->{\input{figs/discrete_gauge_theory_oo_tikz}}
\onslide<2->{\includegraphics[width=60pt]{figs/hungarian_cube.pdf}}
\end{column}
\end{columns}
\end{frame}

\begin{frame}{Simplicial complexes}
\begin{example}
The \defemph{complete simplex of dimension \( n \)}, denoted \alert{\( \Delta(n), \)} is the set \( \{0,\ldots,n\} \) and its power set. The \( (n-1) \)-skeleton \( \Delta(n)_{\leq (n-1)} \) is denoted \alert{\( \partial \Delta(n) \)} and will serve as a combinatorial \( (n-1) \)-sphere.
\end{example}
\( \Delta(1) \) is visually 
\begin{tikzpicture}[baseline=-1.2mm]
\tikzset{oo/.style={circle, scale=0.25, fill=black}}
\node[oo, label=left:\( 0 \)] at (-0.7, 0) (a) {};
\node[oo, label=right:\( 1 \)] at  (0.7, 0) (c) {};
\draw[fill=blue!95!black,fill opacity=0.1] (-0.7, 0)--(0.7, 0);
\end{tikzpicture}, 
\( \partial\Delta(1) \) is visually
\begin{tikzpicture}[baseline=-1.2mm]
\tikzset{oo/.style={circle, scale=0.25, fill=black}}
\node[oo, label=left:\( 0 \)] at (-0.7, 0) (a) {};
\node[oo, label=right:\( 1 \)] at  (0.7, 0) (c) {};
\end{tikzpicture}, 

\( \Delta(2) \) is visually 
\begin{tikzpicture}[baseline=3mm]
\tikzset{oo/.style={circle, scale=0.25, fill=black}}
\node[oo, label=left:\( 0 \)] at (-0.7, 0) (a) {};
\node[oo, label=right:\( 1 \)] at  (0, 1) (b) {};
\node[oo, label=right:\( 2 \)] at  (0.7, 0) (c) {};
\draw[fill=blue!95!black,fill opacity=0.1] (-0.7, 0)--(0, 1)--(0.7, 0)--(a);
\end{tikzpicture}, 
\( \partial \Delta(2) \) is visually 
\begin{tikzpicture}[baseline=3mm]
\tikzset{oo/.style={circle, scale=0.25, fill=black}}
\node[oo, label=left:\( 0 \)] at (-0.7, 0) (a) {};
\node[oo, label=right:\( 1 \)] at  (0, 1) (b) {};
\node[oo, label=right:\( 2 \)] at  (0.7, 0) (c) {};
\draw (a) -- (b);
\draw (b) -- (c);
\draw (c) -- (a);
\end{tikzpicture}
\end{frame}

\begin{frame}{Homotopy realization: dimension 0}
We will \alert{realize} simplicial complexes by means of \alert{a sequence of pushouts}.

Base case: the realization \( \mm \) of a 0-dimensional complex \( M \) is \( M_0 \).

In particular the 0-sphere \( \partial\Dd(1)\defeq \partial \Delta(1)_0 \).
\end{frame}

\begin{frame}{Homotopy realization: dimension 1}
For a 1-dim complex \( M\defeq [M_0,M_1] \) the realization is given by
\[% https://q.uiver.app/#q=WzAsNCxbMCwxLCJNXzA9XFxtYXRoYmJ7TX1fMCJdLFsxLDEsIlxcbWF0aGJie019XzEiXSxbMSwwLCJNXzEiXSxbMCwwLCJNXzFcXHRpbWVzIFxcYmRzaW1wbGV4bnsxfSJdLFswLDFdLFszLDAsIlxcbWF0aGJie0F9XzAiLDJdLFszLDIsIlxcbWF0aHJte3ByfV8xIl0sWzIsMSwiKl97XFxtYXRoYmJ7TX1fMX0iXSxbMSwzLCIiLDIseyJvZmZzZXQiOjMsInN0eWxlIjp7Im5hbWUiOiJjb3JuZXItaW52ZXJzZSJ9fV0sWzAsMiwiaF8xIiwyLHsic2hvcnRlbiI6eyJzb3VyY2UiOjQwLCJ0YXJnZXQiOjQwfSwibGV2ZWwiOjJ9XV0=
\begin{tikzcd}
  {M_1\times \bdsimplexn{1}} & {M_1} \\
  {M_0=\mathbb{M}_0} & {\mathbb{M}_1}
  \arrow["{\mathrm{pr}_1}", from=1-1, to=1-2]
  \arrow["{\mathbb{A}_0}"', from=1-1, to=2-1]
  \arrow["{*_{\mathbb{M}_1}}", from=1-2, to=2-2]
  \arrow["{h_1}"', shorten <=14pt, shorten >=14pt, Rightarrow, from=2-1, to=1-2]
  \arrow[from=2-1, to=2-2]
  \arrow["\ulcorner"{anchor=center, pos=0.125, rotate=180}, shift right=3, draw=none, from=2-2, to=1-1]
\end{tikzcd}

\]
\end{frame}

\begin{frame}{Homotopy realization: dimension 1}
For example the simplicial 1-sphere \( \partial\Dd(2)\defeq \)
\begin{tikzpicture}[baseline=3mm, scale=0.7]
\tikzset{oo/.style={circle, scale=0.25, fill=black}}
\node[oo, label=left:\( 0 \)] at (-0.7, 0) (a) {};
\node[oo, label=right:\( 1 \)] at  (0, 1) (b) {};
\node[oo, label=right:\( 2 \)] at  (0.7, 0) (c) {};
\draw (a) -- (b);
\draw (b) -- (c);
\draw (c) -- (a);
\end{tikzpicture} is given by
\vspace{3ex}
\begin{columns}
\begin{column}{0.4\textwidth}
% https://q.uiver.app/#q=WzAsNCxbMSwwLCJcXHBhcnRpYWwgXFxEZWx0YSgyKV8xIl0sWzEsMSwiXFxwYXJ0aWFsXFxEZCgyKSJdLFswLDAsIlxccGFydGlhbCBcXERlbHRhKDIpXzFcXHRpbWVzIFxccGFydGlhbFxcRGQoMSkiXSxbMCwxLCJcXHBhcnRpYWwgXFxEZWx0YSgyKV8wIl0sWzIsMF0sWzIsM10sWzMsMV0sWzAsMV0sWzEsMiwiIiwxLHsic3R5bGUiOnsibmFtZSI6ImNvcm5lci1pbnZlcnNlIn19XSxbMywwLCJoXzEiLDIseyJzaG9ydGVuIjp7InNvdXJjZSI6MzAsInRhcmdldCI6MzB9LCJsZXZlbCI6Mn1dXQ==
\begin{tikzcd}[ampersand replacement=\&]
  {\partial \Delta(2)_1\times \partial\Dd(1)} \& {\partial \Delta(2)_1} \\
  {\partial \Delta(2)_0} \& {\partial\Dd(2)}
  \arrow[from=1-1, to=1-2]
  \arrow[from=1-1, to=2-1]
  \arrow[from=1-2, to=2-2]
  \arrow["{h_1}"', shorten <=11pt, shorten >=11pt, Rightarrow, from=2-1, to=1-2]
  \arrow[from=2-1, to=2-2]
  \arrow["\ulcorner"{pos=0, rotate=180}, draw=none, from=2-2, to=1-1]
\end{tikzcd}
\end{column}
\begin{column}{0.05\textwidth}
i.e.
\end{column}
\begin{column}{0.6\textwidth}
% https://q.uiver.app/#q=WzAsNCxbMSwwLCJcXHNjcmlwdHN0eWxlXFx7XFx7MCwgMVxcfSwgXFx7MSwgMlxcfSwgXFx7MiwgMFxcfVxcfSJdLFsxLDEsIlxccGFydGlhbFxcRGQoMikiXSxbMCwwLCJcXHNjcmlwdHN0eWxlXFx7XFx7MCwgMVxcfSwgXFx7MSwgMlxcfSwgXFx7MiwgMFxcfVxcfVxcdGltZXMgXFx7MCwgMVxcfSJdLFswLDEsIlxcc2NyaXB0c3R5bGVcXHswLCAxLCAyXFx9Il0sWzIsMF0sWzIsM10sWzMsMV0sWzAsMV0sWzEsMiwiIiwxLHsic3R5bGUiOnsibmFtZSI6ImNvcm5lci1pbnZlcnNlIn19XSxbMywwLCJoXzEiLDIseyJzaG9ydGVuIjp7InNvdXJjZSI6MzAsInRhcmdldCI6MzB9LCJsZXZlbCI6Mn1dXQ==
\begin{tikzcd}[ampersand replacement=\&]
  {\scriptstyle\{\{0, 1\}, \{1, 2\}, \{2, 0\}\}\times \{0, 1\}} \& {\scriptstyle\{\{0, 1\}, \{1, 2\}, \{2, 0\}\}} \\
  {\scriptstyle\{0, 1, 2\}} \& {\partial\Dd(2)}
  \arrow[from=1-1, to=1-2]
  \arrow[from=1-1, to=2-1]
  \arrow[from=1-2, to=2-2]
  \arrow["{h_1}"', shorten <=13pt, shorten >=13pt, Rightarrow, from=2-1, to=1-2]
  \arrow[from=2-1, to=2-2]
  \arrow["\ulcorner"{pos=0, rotate=180}, draw=none, from=2-2, to=1-1]
\end{tikzcd}
\end{column}
\end{columns}
\end{frame}

\begin{frame}{Homotopy realization: dimension 1}
Or the 1-skeleton of the octahedron \( \oo \):
% https://q.uiver.app/#q=WzAsNCxbMCwxLCJcXHt3LCBnLFxcbGRvdHNcXH0iXSxbMCwwLCJcXHtcXHt3LCBnXFx9LFxcbGRvdHNcXH1cXHRpbWVzXFx7MCwxXFx9Il0sWzEsMCwiXFx7XFx7dyxnXFx9LFxcbGRvdHNcXH0iXSxbMSwxLCJcXG9vXzEiXSxbMSwyXSxbMSwwXSxbMCwzXSxbMiwzXSxbMywxLCIiLDEseyJzdHlsZSI6eyJuYW1lIjoiY29ybmVyLWludmVyc2UifX1dLFswLDIsImhfMSIsMix7InNob3J0ZW4iOnsic291cmNlIjozMCwidGFyZ2V0IjozMH0sImxldmVsIjoyfV1d
\[\begin{tikzcd}[ampersand replacement=\&]
  {\{\{w, g\},\ldots\}\times\{0,1\}} \& {\{\{w,g\},\ldots\}} \\
  {\{w, g,\ldots\}} \& {\oo_1}
  \arrow[from=1-1, to=1-2]
  \arrow[from=1-1, to=2-1]
  \arrow[from=1-2, to=2-2]
  \arrow["{h_1}"', shorten <=11pt, shorten >=11pt, Rightarrow, from=2-1, to=1-2]
  \arrow[from=2-1, to=2-2]
  \arrow["\ulcorner"{anchor=center, pos=0.125, rotate=180}, draw=none, from=2-2, to=1-1]
\end{tikzcd}\]
\input{figs/discrete_gauge_theory_oo1_tikz}
\end{frame}

\begin{frame}{Homotopy realization: dimension 2}
To realize \( M\defeq[M_0, M_1, M_2] \) use \(  \partial\Dd(1), \partial\Dd(2) \):
\[% https://q.uiver.app/#q=WzAsNyxbMCwwLCJNXzFcXHRpbWVzIFxcYmRzaW1wbGV4bnsxfSJdLFswLDEsIk1fMD1cXG1hdGhiYntNfV8wIl0sWzEsMSwiXFxtYXRoYmJ7TX1fMSJdLFsxLDAsIk1fMSJdLFsxLDIsIk1fMlxcdGltZXMgXFxiZHNpbXBsZXhuezJ9Il0sWzIsMiwiTV8yIl0sWzIsMSwiXFxtYXRoYmJ7TX1fMiJdLFswLDEsIlxcbWF0aGJie0F9XzAiLDJdLFsxLDJdLFswLDMsIlxcbWF0aHJte3ByfV8xIl0sWzMsMiwiKl97XFxtYXRoYmJ7TX1fMX0iXSxbMiwwLCIiLDEseyJzdHlsZSI6eyJuYW1lIjoiY29ybmVyLWludmVyc2UifX1dLFsyLDZdLFs0LDIsIlxcbWF0aGJie0F9XzEiXSxbNSw2LCIqX3tcXG1hdGhiYntNfV8yfSIsMl0sWzQsNSwiXFxtYXRocm17cHJ9XzEiLDJdLFs2LDQsIiIsMSx7InN0eWxlIjp7Im5hbWUiOiJjb3JuZXItaW52ZXJzZSJ9fV0sWzIsNSwiaF8yIiwwLHsic2hvcnRlbiI6eyJzb3VyY2UiOjQwLCJ0YXJnZXQiOjQwfSwibGV2ZWwiOjJ9XSxbMSwzLCJoXzEiLDIseyJzaG9ydGVuIjp7InNvdXJjZSI6NDAsInRhcmdldCI6NDB9LCJsZXZlbCI6Mn1dXQ==
\begin{tikzcd}
  {M_1\times \bdsimplexn{1}} & {M_1} \\
  {M_0=\mathbb{M}_0} & {\mathbb{M}_1} & {\mathbb{M}_2} \\
  & {M_2\times \bdsimplexn{2}} & {M_2}
  \arrow["{\mathrm{pr}_1}", from=1-1, to=1-2]
  \arrow["{\mathbb{A}_0}"', from=1-1, to=2-1]
  \arrow["{*_{\mathbb{M}_1}}", from=1-2, to=2-2]
  \arrow["{h_1}"', shorten <=23pt, shorten >=23pt, Rightarrow, from=2-1, to=1-2]
  \arrow[from=2-1, to=2-2]
  \arrow["\ulcorner"{pos=0.05, rotate=180}, shift left=1, draw=none, from=2-2, to=1-1]
  \arrow[from=2-2, to=2-3]
  \arrow["{h_2}", shorten <=17pt, shorten >=17pt, Rightarrow, from=2-2, to=3-3]
  \arrow["\ulcorner"{pos=-0.2, rotate=-90}, draw=none, from=2-3, to=3-2]
  \arrow["{\mathbb{A}_1}", from=3-2, to=2-2]
  \arrow["{\mathrm{pr}_1}"', from=3-2, to=3-3]
  \arrow["{*_{\mathbb{M}_2}}"', from=3-3, to=2-3]
\end{tikzcd}
\]
\end{frame}

\begin{frame}{Homotopy realization: dimension 2}
The full octahedron \( \oo \):
\begin{columns}
\begin{column}{0.75\textwidth}
% https://q.uiver.app/#q=WzAsNyxbMCwwLCJcXHtcXHt3LCBnXFx9LFxcbGRvdHNcXH1cXHRpbWVzXFx7MCwxXFx9Il0sWzAsMSwiXFx7dywgZyxcXGxkb3RzXFx9Il0sWzEsMSwiXFxvb18xIl0sWzEsMCwiXFx7XFx7dyxnXFx9LFxcbGRvdHNcXH0iXSxbMSwyLCJcXHtcXHt3LGcsclxcfSxcXGxkb3RzXFx9XFx0aW1lcyBcXHBhcnRpYWxcXERkKDIpIl0sWzIsMiwiXFx7XFx7dyxnLHJcXH0sXFxsZG90c1xcfSJdLFsyLDEsIlxcb29fMiJdLFswLDFdLFsxLDJdLFswLDMsIlxcbWF0aHJte3ByfV8xIl0sWzMsMl0sWzIsMCwiIiwxLHsic3R5bGUiOnsibmFtZSI6ImNvcm5lci1pbnZlcnNlIn19XSxbMiw2XSxbNCwyXSxbNSw2XSxbNCw1LCJcXG1hdGhybXtwcn1fMSIsMl0sWzYsNCwiIiwxLHsic3R5bGUiOnsibmFtZSI6ImNvcm5lci1pbnZlcnNlIn19XSxbMiw1LCJoXzIiLDAseyJzaG9ydGVuIjp7InNvdXJjZSI6NDAsInRhcmdldCI6NDB9LCJsZXZlbCI6Mn1dLFsxLDMsImhfMSIsMix7InNob3J0ZW4iOnsic291cmNlIjo0MCwidGFyZ2V0Ijo0MH0sImxldmVsIjoyfV1d
\[\begin{tikzcd}[ampersand replacement=\&, column sep=small]
  {\{\{w, g\},\ldots\}\times\{0,1\}} \& {\{\{w,g\},\ldots\}} \\
  {\{w, g,\ldots\}} \& {\oo_1} \& {\oo_2} \\
  \& {\{\{w,g,r\},\ldots\}\times \partial\Dd(2)} \& {\{\{w,g,r\},\ldots\}}
  \arrow["{\mathrm{pr}_1}", from=1-1, to=1-2]
  \arrow[from=1-1, to=2-1]
  \arrow[from=1-2, to=2-2]
  \arrow["{h_1}"', shorten <=22pt, shorten >=22pt, Rightarrow, from=2-1, to=1-2]
  \arrow[from=2-1, to=2-2]
  \arrow["\ulcorner"{pos=0.05, rotate=180}, draw=none, from=2-2, to=1-1]
  \arrow[from=2-2, to=2-3]
  \arrow["{h_2}", shorten <=21pt, shorten >=21pt, Rightarrow, from=2-2, to=3-3]
  \arrow["\ulcorner"{pos=-0.1, rotate=-90}, draw=none, from=2-3, to=3-2]
  \arrow[from=3-2, to=2-2]
  \arrow["{\mathrm{pr}_1}"', from=3-2, to=3-3]
  \arrow[from=3-3, to=2-3]
\end{tikzcd}\]
\end{column}
\begin{column}{0.25\textwidth}
\input{figs/discrete_gauge_theory_oo_tikz}
\end{column}
\end{columns}
\end{frame}

\begin{frame}{Homotopy realization: dimension 2}
\begin{columns}
\begin{column}{0.25\textwidth}
\input{figs/discrete_gauge_theory_oo_link_tikz.tex}
\end{column}
\begin{column}{0.15\textwidth}
\( \leftarrow\link(w) \)
\end{column}
\begin{column}{0.6\textwidth}
The \defemph{link} of a vertex \( w \) in a 2-complex is: the sets not containing \( w \) but whose union with \( w \) is a face.\\~\\

A \alert{combinatorial manifold} is a simplicial complex all of whose links are\( ^* \) simplicial spheres.\\~\\

This will be our model of the \alert{tangent space}.\\~\\
\end{column}
\end{columns}
\vspace{5ex}
{\scriptsize\( {}^* \)the (classical) geometric realization is homeomorphic to a sphere}
\end{frame}

\begin{frame}{Combinatorial manifolds \( \leftrightarrow \) smooth manifolds}
\begin{theorem}[Whitehead (1940)]
Every smooth \( n \)-manifold has a compatible structure of a \alert{combinatorial manifold}: a simplicial complex of dimension \( n \) such that the link is a combinatorial \( (n-1) \)-sphere, i.e. its geometric realization is an \( (n-1) \)-sphere.
\end{theorem}
\url{https://ncatlab.org/nlab/show/triangulation+theorem}

\onslide<2->{Counterexample: Wikipedia says this is a simplicial complex, but we can see it fails the link condition:

\includegraphics[width=15ex]{figs/simplicial_complex_example.pdf}}
\end{frame}


% \section{Torsors}

\begin{frame}
What type families \( \mm\to\uni \) will we consider? Families of \alert{torsors}, also called \alert{principal bundles}.
\end{frame}

\begin{frame}{Torsors}
Let \( G \) be a (higher) group.
\onslide<2->{\begin{definition}
\begin{itemize}
\item<2-> A \defemph{right \( G \)-object} is a type \( X \) equipped with a homomorphism \( \phi:G^{\mathrm{op}}\to\Aut(X) \).
\item<3-> \( X \) is a \alert{torsor} if it is inhabited and the map \( (\pr_1, \phi):X\times G\to X\times X \) is an equivalence.
\item<4-> The inverse is \( (\pr_1, s) \) where \( s:X\times X\to G \) is called \alert{subtraction} (when \( G \) is commutative).
\item<5-> Let \( \reg{G} \) be the \( G \)-torsor consisting of \( G \) acting on itself on the right.
\end{itemize}
\end{definition}}
\end{frame}

\begin{frame}{Facts}
\begin{enumerate}
\item<2-> \( \loopy(BG,\reg{G}) \simeq G \) and composition of loops corresponds to multiplication in \( G \).
\item<3-> \( BG \) is connected.
\item<4-> 1 \& 2 \( \implies \) \( BG \) is a \( \K(G,1) \).
\item<5-> \( \ev(e):(\reg{G}=_{BG} X)\to X \) is an equivalence (needed when we have vector fields).
\end{enumerate}
\onslide<5->{See the Buchholtz et. al. H-spaces paper for more.}
\end{frame}

\begin{frame}{How to map into \( BS^1 \)}
To construct maps into \( BS^1 \) we \alert{lift} a family of mere circles. (Remark: the lift is a choice of \alert{orientation}.)
% https://q.uiver.app/#q=WzAsNSxbMSwwLCJCU14xIl0sWzIsMCwiXFxCQXV0IChTXjEpIl0sWzMsMCwiXFx1bmkiXSxbMiwyLCJcXG1tIl0sWzAsMSwiXFx0ZXh0e2ZhbWlsaWVzIG9mOn0iXSxbMywwLCJcXG1hdGhybXt0b3Jzb3JzfSJdLFszLDEsIlxcc3Vic3RhY2t7XFxtYXRocm17bWVyZX1cXFxcIFxcbWF0aHJte2NpcmNsZXN9fSIsMV0sWzMsMiwiXFxtYXRocm17dHlwZXN9IiwyXSxbMCwxXSxbMSwyXV0=
\[\begin{tikzcd}[ampersand replacement=\&, row sep=small]
  \& {BS^1} \& {\BAut (S^1)} \& \uni \\
  {\text{families of:}} \\
  \&\& \mm
  \arrow[from=1-2, to=1-3]
  \arrow[from=1-3, to=1-4]
  \arrow["{\mathrm{torsors}}", from=3-3, to=1-2]
  \arrow["\begin{array}{c} \substack{\mathrm{mere}\\ \mathrm{circles}} \end{array}"{description}, from=3-3, to=1-3]
  \arrow["{\mathrm{types}}"', from=3-3, to=1-4]
\end{tikzcd}\]
We will assume we have such a lift when we need it.

\onslide<2->{
Other names:
\begin{itemize}
\item \( \BAut (S^1)=BO(2)=\EMzo \) (where \( \EM(G,n)\defeq \BAut(\K(G,n)) \))
\item \( BS^1=BSO(2)=\Kzt \)
\end{itemize}}
\end{frame}

% \begin{frame}{A connected component of \( \uni \)?}
% \begin{definition}
% The \alert{type of Eilenberg-Mac Lane spaces \( \EM(G,n) \)} is the connected component of \( \K(G,n) \):
% \[ \EM(G,n)\defeq \BAut(\K(G,n))\defeq \sit{Y:\uni}||Y\simeq \K(G,n)||_{-1} \]
% \end{definition}
% It is a property of a map \( f:A\to\EM(G,n) \) to factor through \( \K(G,n+1) \). See the Scoccola paper.
% \end{frame}

% \begin{frame}{Coincidences of 2 dimensions}
% \begin{itemize}
% \item \( S^1 \) is a \( \Kzo \) since \( \loopy(S^1, \base)\simeq \zz \).
% \item So \( \EMzo \) is a type of \alert{mere circles}.
% \item But \( S^1=_{\EMzo}S^1 \) contains an order 2 \alert{flip}, so \( \not\simeq S^1 \).
% \item For a map \( f:A\to\EMzo \) to factor through  \( \Kzt \), it must somehow avoid flips.
% \item This deserves to be called \alert{orientability}.
% \item \( \link:\mm_0\to\EMzo \) is a great starting point.
% \end{itemize}
% \end{frame}
% 


% \section{Connections and curvature}

\begin{frame}{What we hope to capture and explain}
  \( \vcenter{\hbox{\includegraphics[width=30mm]{figs/curved_cube/curved_cube1}}} \!\!\to\!\! \)
  \( \vcenter{\hbox{\includegraphics[width=30mm]{figs/curved_cube/curved_cube2}}}\!\!\to\!\! \)
  \( \vcenter{\hbox{\includegraphics[width=30mm]{figs/curved_cube/curved_cube3}}}\!\!\to\!\! \)
  \( \vcenter{\hbox{\includegraphics[width=30mm]{figs/curved_cube/curved_cube4}}}\!\!\to\!\! \)
  \( \vcenter{\hbox{\includegraphics[width=30mm]{figs/curved_cube/curved_cube5}}}\!\!\to\!\! \)
  \( \vcenter{\hbox{\includegraphics[width=30mm]{figs/curved_cube/curved_cube6}}}\!\!\to\!\! \)
  \( \vcenter{\hbox{\includegraphics[width=30mm]{figs/curved_cube/curved_cube7}}}\!\!\to\!\! \)
  \( \vcenter{\hbox{\includegraphics[width=30mm]{figs/curved_cube/curved_cube8}}} \)
\end{frame}

\begin{frame}{\( T:\mm\to\EMzo \) extending \( \link \)}
We define \( T \) on edges by imaginging \alert{tipping}:

\begingroup
\tikzset{every picture/.style={scale=0.85}}
\begin{tikzpicture}%
  [x={(-0.860769cm, -0.121512cm)},
  y={(0.508996cm, -0.205391cm)},
  z={(-0.000053cm, 0.971107cm)},
  scale=1,
  eqback/.style={very thick},
  back/.style={loosely dotted, thin},
  eqedge/.style={very thick},
  edge/.style={black, thin},
  r/.style={red},
  facet/.style={fill=blue!95!black,fill opacity=0.1},
  vertex/.style={inner sep=1pt,circle,draw=green!25!black,fill=black,thick}]
\coordinate (-1, -1, 0) at (-1, -1, 0);
\coordinate (-1, 1, 0) at (-1, 1, 0);
\coordinate (0, 0, -1) at (0, 0, -1);
\coordinate (0, 0, 1) at (0, 0, 1);
\coordinate (1, -1, 0) at (1, -1, 0);
\coordinate (1, 1, 0) at (1, 1, 0);
%% Drawing edges in the back
%%
\draw[edge,eqback] (-1, -1, 0) -- (-1, 1, 0);
\draw[edge,back] (-1, -1, 0) -- (0, 0, -1.4);
\draw[edge,back] (-1, -1, 0) -- (0, 0, 1.4);
\draw[edge,eqback] (1, -1, 0) -- (-1, -1, 0);
%% Drawing vertices in the back
%%
\node[vertex] at (-1, -1, 0)     {};
%% Drawing the facets
%%
%\fill[facet] (1, 1, 0) -- (0, 0, -1.4) -- (1, -1, 0) -- cycle {};
%\fill[facet] (1, 1, 0) -- (0, 0, 1.4) -- (1, -1, 0) -- cycle {};
\fill[facet] (1, 1, 0) -- (-1, 1, 0) -- (0, 0, 1.4) -- cycle {};
%\fill[facet] (1, 1, 0) -- (-1, 1, 0) -- (0, 0, -1.4) -- cycle {};
%% Drawing edges in the front
%%
\draw[edge] (-1, 1, 0) -- (0, 0, -1.4);
\draw[edge] (-1, 1, 0) -- (0, 0, 1.4);
\draw[eqedge] (-1, 1, 0) -- (1, 1, 0);
\draw[edge] (0, 0, -1.4) -- (1, -1, 0);
\draw[edge] (0, 0, -1.4) -- (1, 1, 0);
\draw[edge] (0, 0, 1.4) -- (1, -1, 0);
\draw[edge] (0, 0, 1.4) -- (1, 1, 0);
\draw[r,eqedge] (1, 1, 0) -- (1, -1, 0);
%% Drawing the vertices in the front
%%
\begin{scope}[nodes=vertex]
\node[label=above right:\( b \)] at (-1, 1, 0)     {};
\node[label=below:\( y \)] at (0, 0, -1.4)     {};
\node[label=above:\( w \)] at (0, 0, 1.4)     {};
\node[label=above left:\( g \)] at (1, -1, 0)     {};
\node[label=above left:\( r \)] at (1, 1, 0)     {};
\node[label=above right:\( o \)] at (-1, -1, 0)     {};
\end{scope}
\end{tikzpicture}
\begin{tikzpicture}%
  [x={(-0.860769cm, -0.121512cm)},
  y={(0.508996cm, -0.205391cm)},
  z={(-0.000053cm, 0.971107cm)},
  scale=1,
  eqback/.style={very thick},
  back/.style={loosely dotted, thin},
  eqedge/.style={very thick},
  r/.style={red},
  edge/.style={black, thin},
  facet/.style={fill=blue!95!black,fill opacity=0.1},
  vertex/.style={inner sep=1pt,circle,draw=green!25!black,fill=black,thick}]
\coordinate (-1, -1, 0) at (-1, -1, 0);
\coordinate (-1, 1, 0) at (-1, 1, 0);
\coordinate (0, 0, -1) at (0, 0, -1);
\coordinate (0, 0, 1) at (0, 0, 1);
\coordinate (1, -1, 0) at (1, -1, 0);
\coordinate (1, 1, 0) at (1, 1, 0);
%% Drawing edges in the back
%%
\draw[edge,back] (-1, -1, 0) -- (-1, 1, 0);
\draw[edge,eqback] (-1, -1, 0) -- (0, 0, -1.4);
\draw[edge,eqback] (0, 0, 1.4) -- (-1, -1, 0);
\draw[edge,back] (1, -1, 0) -- (-1, -1, 0);
%% Drawing vertices in the back
%%
\node[vertex] at (-1, -1, 0)     {};
%% Drawing the facets
%%
% \fill[facet] (1, 1, 0) -- (0, 0, -1.4) -- (1, -1, 0) -- cycle {};
% \fill[facet] (1, 1, 0) -- (0, 0, 1.4) -- (1, -1, 0) -- cycle {};
\fill[facet] (1, 1, 0) -- (-1, 1, 0) -- (0, 0, 1.4) -- cycle {};
% \fill[facet] (1, 1, 0) -- (-1, 1, 0) -- (0, 0, -1.4) -- cycle {};
%% Drawing edges in the front
%%
\draw[edge] (-1, 1, 0) -- (0, 0, -1.4);
\draw[edge] (-1, 1, 0) -- (0, 0, 1.4);
\draw[edge] (-1, 1, 0) -- (1, 1, 0);
\draw[edge] (0, 0, -1.4) -- (1, -1, 0);
\draw[eqedge] (0, 0, -1.4) -- (1, 1, 0);
\draw[edge] (0, 0, 1.4) -- (1, -1, 0);
\draw[r,eqedge] (1, 1, 0) -- (0, 0, 1.4) ;
\draw[edge] (1, 1, 0) -- (1, -1, 0);
%% Drawing the vertices in the front
%%
\begin{scope}[nodes=vertex]
\node[label=above right:\( b \)] at (-1, 1, 0)     {};
\node[label=below:\( y \)] at (0, 0, -1.4)     {};
\node[label=above:\( w \)] at (0, 0, 1.4)     {};
\node[label=above left:\( g \)] at (1, -1, 0)     {};
\node[label=above left:\( r \)] at (1, 1, 0)     {};
\node[label=above right:\( o \)] at (-1, -1, 0)     {};
\end{scope}
\end{tikzpicture}
\begin{tikzpicture}%
  [x={(-0.860769cm, -0.121512cm)},
  y={(0.508996cm, -0.205391cm)},
  z={(-0.000053cm, 0.971107cm)},
  scale=1,
  eqback/.style={very thick},
  back/.style={loosely dotted, thin},
  eqedge/.style={very thick},
  r/.style={red},
  edge/.style={black, thin},
  facet/.style={fill=blue!95!black,fill opacity=0.1},
  vertex/.style={inner sep=1pt,circle,draw=green!25!black,fill=black,thick}]
\coordinate (-1, -1, 0) at (-1, -1, 0);
\coordinate (-1, 1, 0) at (-1, 1, 0);
\coordinate (0, 0, -1) at (0, 0, -1);
\coordinate (0, 0, 1) at (0, 0, 1);
\coordinate (1, -1, 0) at (1, -1, 0);
\coordinate (1, 1, 0) at (1, 1, 0);
%% Drawing edges in the back
%%
\draw[edge,back] (-1, -1, 0) -- (-1, 1, 0);
\draw[edge,back] (-1, -1, 0) -- (0, 0, -1.4);
\draw[edge,back] (-1, -1, 0) -- (0, 0, 1.4);
\draw[edge,back] (1, -1, 0) -- (-1, -1, 0);
%% Drawing vertices in the back
%%
\node[vertex] at (-1, -1, 0)     {};
%% Drawing the facets
%%
% \fill[facet] (1, 1, 0) -- (0, 0, -1.4) -- (1, -1, 0) -- cycle {};
% \fill[facet] (1, 1, 0) -- (0, 0, 1.4) -- (1, -1, 0) -- cycle {};
\fill[facet] (1, 1, 0) -- (-1, 1, 0) -- (0, 0, 1.4) -- cycle {};
% \fill[facet] (1, 1, 0) -- (-1, 1, 0) -- (0, 0, -1.4) -- cycle {};
%% Drawing edges in the front
%%
\draw[eqedge] (-1, 1, 0) -- (0, 0, -1.4);
\draw[eqedge] (0, 0, 1.4) -- (-1, 1, 0);
\draw[edge] (-1, 1, 0) -- (1, 1, 0);
\draw[eqedge] (0, 0, -1.4) -- (1, -1, 0);
\draw[edge] (0, 0, -1.4) -- (1, 1, 0);
\draw[r,eqedge] (1, -1, 0) -- (0, 0, 1.4);
\draw[edge] (0, 0, 1.4) -- (1, 1, 0);
\draw[edge] (1, 1, 0) -- (1, -1, 0);
%% Drawing the vertices in the front
%%
\begin{scope}[nodes=vertex]
\node[label=above right:\( b \)] at (-1, 1, 0)     {};
\node[label=below:\( y \)] at (0, 0, -1.4)     {};
\node[label=above:\( w \)] at (0, 0, 1.4)     {};
\node[label=above left:\( g \)] at (1, -1, 0)     {};
\node[label=above left:\( r \)] at (1, 1, 0)     {};
\node[label=above right:\( o \)] at (-1, -1, 0)     {};
\end{scope}
\end{tikzpicture}
\begin{tikzpicture}%
  [x={(-0.860769cm, -0.121512cm)},
  y={(0.508996cm, -0.205391cm)},
  z={(-0.000053cm, 0.971107cm)},
  scale=1,
  eqback/.style={very thick},
  back/.style={loosely dotted, thin},
  eqedge/.style={very thick},
  edge/.style={black, thin},
  r/.style={red},
  facet/.style={fill=blue!95!black,fill opacity=0.1},
  vertex/.style={inner sep=1pt,circle,draw=green!25!black,fill=black,thick}]
\coordinate (-1, -1, 0) at (-1, -1, 0);
\coordinate (-1, 1, 0) at (-1, 1, 0);
\coordinate (0, 0, -1) at (0, 0, -1);
\coordinate (0, 0, 1) at (0, 0, 1);
\coordinate (1, -1, 0) at (1, -1, 0);
\coordinate (1, 1, 0) at (1, 1, 0);
%% Drawing edges in the back
%%
\draw[edge,eqback] (-1, -1, 0) -- (-1, 1, 0);
\draw[edge,back] (-1, -1, 0) -- (0, 0, -1.4);
\draw[edge,back] (-1, -1, 0) -- (0, 0, 1.4);
\draw[edge,eqback,r] (1, -1, 0) -- (-1, -1, 0);
%% Drawing vertices in the back
%%
\node[vertex] at (-1, -1, 0)     {};
%% Drawing the facets
%%
% \fill[facet] (1, 1, 0) -- (0, 0, -1.4) -- (1, -1, 0) -- cycle {};
% \fill[facet] (1, 1, 0) -- (0, 0, 1.4) -- (1, -1, 0) -- cycle {};
\fill[facet] (1, 1, 0) -- (-1, 1, 0) -- (0, 0, 1.4) -- cycle {};
% \fill[facet] (1, 1, 0) -- (-1, 1, 0) -- (0, 0, -1.4) -- cycle {};
%% Drawing edges in the front
%%
\draw[edge] (-1, 1, 0) -- (0, 0, -1.4);
\draw[edge] (-1, 1, 0) -- (0, 0, 1.4);
\draw[eqedge] (-1, 1, 0) -- (1, 1, 0);
\draw[edge] (0, 0, -1.4) -- (1, -1, 0);
\draw[edge] (0, 0, -1.4) -- (1, 1, 0);
\draw[edge] (0, 0, 1.4) -- (1, -1, 0);
\draw[edge] (0, 0, 1.4) -- (1, 1, 0);
\draw[eqedge] (1, 1, 0) -- (1, -1, 0);
%% Drawing the vertices in the front
%%
\begin{scope}[nodes=vertex]
\node[label=above right:\( b \)] at (-1, 1, 0)     {};
\node[label=below:\( y \)] at (0, 0, -1.4)     {};
\node[label=above:\( w \)] at (0, 0, 1.4)     {};
\node[label=above left:\( g \)] at (1, -1, 0)     {};
\node[label=above left:\( r \)] at (1, 1, 0)     {};
\node[label=above right:\( o \)] at (-1, -1, 0)     {};
\end{scope}
\end{tikzpicture}

\endgroup
\( \tr(\partial(wbr)):Tw=Tw \) is clockwise rotation by one notch.

We define \( T \) on the face \( wbr \) by the shortest homotopy \( T(wbr):\id=\tr(\partial(wbr)) \).
\end{frame}

% \begin{frame}{Rotation}
% Let \( R:\gr{abcd}\to\gr{abcd} \) send \( a\mapsto b , b\mapsto c , c\mapsto d, d\mapsto a \). \\~\\
% 
% Extend \( R \) to edges.
% 
% \begin{lemma}
% \( \hgr{R}:\hgr{abcd}\to\hgr{abcd} \) is homotopic to the identity, i.e. we have \( \pit{x:\hgr{abcd}}x=\hgr{R}(x) \).
% \end{lemma}
% \begin{proof}
% Use edges.
% \end{proof}
% \end{frame}

\begin{frame}
\begin{definition}
If \( \mm\defeq \mm_0\xrightarrow[]{\imath_0}\cdots\xrightarrow[]{\imath_{n-1}}\mm_n \) is a realization and all the triangles commute in the diagram:\vspace{-10pt}
\[\begin{tikzcd}[ampersand replacement=\&, column sep=small]
  {\mm_0} \& {\mm_1} \& {\mm_2} \& \cdots \& {\mm_n} \\
\&\& {\mathcal{U}}
\arrow["{\imath_0}", from=1-1, to=1-2]
\arrow["{f_0}", from=1-1, to=2-3]
\arrow["{\imath_1}", from=1-2, to=1-3]
\arrow["{f_1}", from=1-2, to=2-3]
\arrow["{\imath_2}", from=1-3, to=1-4]
\arrow["{f_2}", from=1-3, to=2-3]
\arrow["{\imath_{n-1}}", from=1-4, to=1-5]
\arrow["f_n"', from=1-5, to=2-3]
\end{tikzcd}\]\vspace{-15pt}
\begin{itemize}
\item The map \( f_k \) is a \defemph{\( k \)-bundle} on \( \mm \).
\item The pair given by the map \( f_k \) and the proof \( f_k\circ \imath_{k-1}=f_{k-1} \), i.e. that \( f_k \) extends \( f_{k-1} \) is called a \defemph{\( k \)-connection on the \( (k-1) \)-bundle \( f_{k-1} \)}.
\end{itemize}
\end{definition}
\end{frame}

\begin{frame}
\begin{mydef}[cont.]
\[\begin{tikzcd}[ampersand replacement=\&, column sep=small]
  {M_k\times \partial\Delta^k} \& {M_k} \\
  {\mathbb{M}_{k-1}} \& {\mathbb{M}_k} \\
  \& {\mathcal{U}}
  \arrow["{\mathrm{pr}_1}", from=1-1, to=1-2]
  \arrow["{\mathbb{A}_{k-1}}"', from=1-1, to=2-1]
  \arrow["{*_{\mathbb{M}_k}}", from=1-2, to=2-2]
  \arrow["{h_k}", shorten <=10pt, shorten >=10pt, Rightarrow, from=2-1, to=1-2]
  \arrow["{\imath_{k-1}}", from=2-1, to=2-2]
  \arrow[""{name=0, anchor=center, inner sep=0}, "{f_{k-1}}"', from=2-1, to=3-2]
  \arrow["\ulcorner"{pos=-0.05, rotate=180}, shift left=1.5, draw=none, from=2-2, to=1-1]
  \arrow["{f_k}", from=2-2, to=3-2]
\end{tikzcd}
\begin{tikzcd}[ampersand replacement=\&]
  {\{F\}\times \partial\Delta^2} \& \unit \\
  {\mathbb{M}_{k-1}} \& {\mathcal{U}}
  \arrow["{!}", from=1-1, to=1-2]
  \arrow["{\mathbb{A}_{k-1}}"', from=1-1, to=2-1]
  \arrow["{*_{\mathbb{M}_k}}", from=1-2, to=2-2]
  \arrow["{\flat_F}", shorten <=11pt, shorten >=11pt, Rightarrow, from=1-2, to=2-1]
  \arrow[from=2-1, to=2-2]
\end{tikzcd}\]
the filler \( \flat_F \) is called a \defemph{flatness structure for the face \( F \)}, and its ending path (the holonomy around the boundary) is called \defemph{the \( k \)-curvature at the face \( F \)}.
\end{mydef}
\end{frame}
\begin{frame}
With these definitions we have now achieved one of our main goals.

Without a definition of Euler characterisitc we can't prove Gauss-Bonnet.

But once we add vector fields there is a lot more to say.
\end{frame}


% \section{Vector fields}
\begin{frame}{Vector fields}
\begin{columns}
\begin{column}{0.7\textwidth}
Let \( T:\mm_2\to\Kzt \) be an oriented tangent bundle on a 2-dim cellular type
\begin{itemize}
\item A \alert{vector field} is a term \alert{\( X:\pit{m:\mm_1}Tm \)}.
\item It's a \alert{nonvanishing} vector field on the 1-skeleton.
\item We model classical zeros by omitting the faces.
\end{itemize}
\end{column}
\begin{column}{0.3\textwidth}
\begin{tikzpicture}%
  [x={(-0.860769cm, -0.121512cm)},
  y={(0.508996cm, -0.205391cm)},
  z={(-0.000053cm, 0.971107cm)},
  scale=1,
  back/.style={loosely dotted, thin},
  edge/.style={black, thick},
  arrow/.style={black, very thick, solid, -{Stealth[scale=0.8]}},
  facet/.style={fill=blue!95!black,fill opacity=0.0},
  vertex/.style={inner sep=1pt,circle,draw=green!25!black,fill=black,thick}]
%% Drawing the vertices in the front
%%
\begin{scope}[nodes=vertex]
\node[label=above right:\( b \)] at (-1, 1, 0) (b)     {};
\node[label=below:\( y \)] at (0, 0, -1.4) (y)    {};
\node[label=above:\( w \)] at (0, 0, 1.4)  (w)   {};
\node[label=above left:\( g \)] at (1, -1, 0) (g)    {};
\node[label=above left:\( r \)] at (1, 1, 0)  (r)   {};
\node[label=above right:\( o \)] at (-1, -1, 0) (o)    {};
\end{scope}
%% Drawing edges in the back
%%
\draw[edge,back,arrow] (o) -- (b);
\draw[edge,back,arrow] (y) -- (o);
\draw[edge,back] (o) -- (w);
\draw[edge,back] (o) -- (g);
%% Drawing vertices in the back
%%
\node[vertex] at (o)     {};
%% Drawing the facets
%%
\fill[facet] (1, 1, 0) -- (0, 0, -1.4) -- (1, -1, 0) -- cycle {};
\fill[facet] (1, 1, 0) -- (0, 0, 1.4) -- (1, -1, 0) -- cycle {};
\fill[facet] (1, 1, 0) -- (-1, 1, 0) -- (0, 0, 1.4) -- cycle {};
\fill[facet] (1, 1, 0) -- (-1, 1, 0) -- (0, 0, -1.4) -- cycle {};
%% Drawing edges in the front
%%
\draw[edge,arrow] (b) -- (y);
\draw[edge] (b) -- (w);
\draw[edge] (b) -- (r);
\draw[edge] (y) -- (g);
\draw[edge] (y) -- (r);
\draw[edge,arrow] (g) -- (w);
\draw[edge,arrow] (w) -- (r);
\draw[edge,arrow] (r) -- (g);
\end{tikzpicture}

\end{column}
\end{columns}
\end{frame}

\begin{frame}{Pathovers}
\begin{columns}
\begin{column}{0.55\textwidth}
\includegraphics[width=30ex]{figs/pathovers.pdf}
\end{column}
\begin{column}{0.45\textwidth}
\begin{itemize}
\item Recall pathovers (dependent paths).
\item There is an asymmetry: we pick a fiber to display it.
\item Dependent functions map paths to pathovers (\( \apd \)).
\end{itemize}
\end{column}
\end{columns}
\end{frame}

\begin{frame}{Building up a triangle-over}
\begin{tikzcd}[ampersand replacement=\&, row sep=small]
  {T_1} \& {T_2} \& {T_3} \& {T_1} \\
  \&\&\& {T_{13}T_{32}T_{21}X_1} \\
  \&\& {T_{32}T_{21}X_1} \& {T_{13}T_{32}X_2} \\
  \& {T_{21}X_1} \& {T_{32}X_2} \& {T_{13}X_3} \\
  {X_1} \& {X_2} \& {X_3} \& {X_1}
  \arrow["{T_{21}}", from=1-1, to=1-2]
  \arrow["{T_{32}}", from=1-2, to=1-3]
  \arrow["{T_{13}}", from=1-3, to=1-4]
  \arrow["{\alert{T_{13}T_{32}X_{12}}:}", equals, from=3-4, to=2-4]
  \arrow["{\alert{X_{12}}:}"', equals, from=4-2, to=5-2]
  \arrow["{\alert{T_{32}X_{12}}:}", equals, from=4-3, to=3-3]
  \arrow["{\alert{T_{13}X_{23}}:}", equals, from=4-4, to=3-4]
  \arrow["{\alert{X_{23}}:}", equals, from=5-3, to=4-3]
  \arrow["{\alert{X_{31}}:}", equals, from=5-4, to=4-4]
\end{tikzcd}
\end{frame}

\begin{frame}
\begin{columns}
\begin{column}{0.65\textwidth}
\vspace{12pt}
\begingroup
\tikzset{every picture/.style={scale=0.85}}
\begin{tikzpicture}
  [arrow/.style={-{Stealth[scale=1.1]}}, vec/.style={ultra thick, color=black}, vectr/.style={thick, color=black}, vectrtr/.style={thick, dashed, color=black}, vectrtrtr/.style={thick, dotted, color=black}]
  \tikzset{oo/.style={circle, scale=0.6, fill=black}}
  \tikzset{ii/.style={circle, scale=0.3, fill=gray}}
\newlength{\mylen}
  \setlength{\mylen}{3cm}
  \newlength{\mylin}
  \setlength{\mylin}{1.2cm}
    \node[oo, label=below right:\( v_1 \)] (V1) at (0, 0) {};
    \node[oo, label=below:\( v_2 \)] (V2) at (2*\mylen, 0) {};
    \node[oo, label=above:\( v_3 \)] (V3) at (\mylen, 1.732*\mylen) {};

    \draw[arrow] (V2) edge[very thick, color=teal, "\( e_{23} \)"] (V3);
    \draw[arrow] (V1) edge[very thick, color=magenta, "\( e_{12} \)"] (V2);
    \draw[arrow] (V3) edge[very thick, color=blue, "\( e_{31} \)"] (V1);
    
    \node [ii, above right=\mylin of V1,  label=above right:\( v_{11} \)] (V11) {};
    \node [ii, below right=\mylin of V1,  label=below right:\( v_{12} \)] (V12) {};
    \node [ii, below left=\mylin of  V1,  label=below left:\( v_{13} \)] (V13) {};
    \node [ii, above left= \mylin of V1,  label=above left: \( v_{14} \)] (V14) {};

    \node [left=1.3\mylin of  V1,  label=center:\( T_1 \)] {};
    \node [right=1.3\mylin of  V2,  label=center:\( T_2 \)] {};
    \node [left=1.3\mylin of  V3,  label=center:\( T_3 \)] {};

    \node [ii, above right=\mylin of V2, label=above right:\( v_{21} \)] (V21) {};
    \node [ii, below right=\mylin of V2,  label=below right:\( v_{22} \)] (V22) {};
    \node [ii, below left=\mylin of  V2, label=below left:\( v_{23} \)] (V23) {};
    \node [ii, above left= \mylin of V2,  label=above left: \( v_{24} \)] (V24) {};

    \node [ii, above right=\mylin of V3, label=above right:\( v_{31} \)] (V31) {};
    \node [ii, below right=\mylin of V3,  label=below right:\( v_{32} \)] (V32) {};
    \node [ii, below left=\mylin of  V3, label=below left:\( v_{33} \)] (V33) {};
    \node [ii, above left= \mylin of V3,  label=above left: \( v_{34} \)] (V34) {};

    \draw[dashed] (V11) -- (V12);
    \draw[dashed] (V12) -- (V13);
    \draw[dashed] (V13) -- (V14);
    \draw[dashed] (V14) -- (V11);

    \draw[dashed] (V21) -- (V22);
    \draw[dashed] (V22) -- (V23);
    \draw[dashed] (V23) -- (V24);
    \draw[dashed] (V24) -- (V21);
    
    \draw[dashed] (V31) -- (V32);
    \draw[dashed] (V32) -- (V33);
    \draw[dashed] (V33) -- (V34);
    \draw[dashed] (V34) -- (V31);
    
    \draw[arrow] (V1) edge[vec] (V11);
    \draw[arrow] (V2) edge[vectr] (V21);
    \draw[arrow] (V3) edge[vectrtr] (V34);
    \draw[arrow] (V1) edge[vectrtrtr] (V14);

    \draw[arrow] (V2) edge[vec] (V24);
    \draw[arrow] (V3) edge[vectr] (V33);
    \draw[arrow] (V1) edge[vectrtr] (V13);

    \draw[arrow] (V21) edge[thick, color=magenta] (V24);
    \draw[arrow] (V34) edge[thick, color=magenta] (V33);
    \draw[arrow] (V14) edge[thick, color=magenta] (V13);
    \draw[arrow] (V33) edge[thick, color=teal] (V32);
    \draw[arrow] (V13) edge[thick, color=teal] (V12);
    \draw[arrow] (V12) edge[thick, color=blue] (V11);

    \draw[arrow] (V3) edge[vec] (V32);
    \draw[arrow] (V1) edge[vectr] (V12);
\end{tikzpicture}

\endgroup
\end{column}
\begin{column}{0.35\textwidth}
\begin{itemize}
\item \( \partial F\defeq e_{12}\cdot e_{23}\cdot e_{31}.  \)
\item \( \tr \) thins out arrows.
\item \( X \) on a path is drawn in the path's color.
\item \( X(\partial F) \) traces 3 sides of a square.
\end{itemize}
\end{column}
\end{columns}
\end{frame}

\begin{frame}{Angle}
\begin{columns}
\begin{column}{0.3\textwidth}
\begin{tikzpicture}%
  [x={(-0.860769cm, -0.121512cm)},
  y={(0.508996cm, -0.205391cm)},
  z={(-0.000053cm, 0.971107cm)},
  scale=1,
  back/.style={loosely dotted, thin},
  edge/.style={black, thick},
  arrow/.style={black, very thick, solid, -{Stealth[scale=0.8]}},
  facet/.style={fill=blue!95!black,fill opacity=0.0},
  vertex/.style={inner sep=1pt,circle,draw=green!25!black,fill=black,thick}]
%% Drawing the vertices in the front
%%
\begin{scope}[nodes=vertex]
\node[label=above right:\( b \)] at (-1, 1, 0) (b)     {};
\node[label=below:\( y \)] at (0, 0, -1.4) (y)    {};
\node[label=above:\( w \)] at (0, 0, 1.4)  (w)   {};
\node[label=above left:\( g \)] at (1, -1, 0) (g)    {};
\node[label=above left:\( r \)] at (1, 1, 0)  (r)   {};
\node[label=above right:\( o \)] at (-1, -1, 0) (o)    {};
\end{scope}
%% Drawing edges in the back
%%
\draw[edge,back,arrow] (o) -- (b);
\draw[edge,back,arrow] (y) -- (o);
\draw[edge,back] (o) -- (w);
\draw[edge,back] (o) -- (g);
%% Drawing vertices in the back
%%
\node[vertex] at (o)     {};
%% Drawing the facets
%%
\fill[facet] (1, 1, 0) -- (0, 0, -1.4) -- (1, -1, 0) -- cycle {};
\fill[facet] (1, 1, 0) -- (0, 0, 1.4) -- (1, -1, 0) -- cycle {};
\fill[facet] (1, 1, 0) -- (-1, 1, 0) -- (0, 0, 1.4) -- cycle {};
\fill[facet] (1, 1, 0) -- (-1, 1, 0) -- (0, 0, -1.4) -- cycle {};
%% Drawing edges in the front
%%
\draw[edge,arrow] (b) -- (y);
\draw[edge] (b) -- (w);
\draw[edge] (b) -- (r);
\draw[edge] (y) -- (g);
\draw[edge] (y) -- (r);
\draw[edge,arrow] (g) -- (w);
\draw[edge,arrow] (w) -- (r);
\draw[edge,arrow] (r) -- (g);
\end{tikzpicture}

\end{column}
\begin{column}{0.7\textwidth}
\begin{itemize}
\item We want to extract from each \( X_{ij} \) just the \alert{angle}, a non-dependent quantity.
\item e.g. in this example: 3 copies of ``+1 notch'' and 3 of ``--1 notch.''
\item The total swirled angle is 0.
\end{itemize}
\end{column}
\end{columns}
\end{frame}

\begin{frame}{Angle}
\onslide<1->{Observation 1: Use the torsor structure. If we choose \( m:\mm \) then \( T_m=T_m \) acts on all fibers. We can define subtraction \( T_i\times T_i\to(T_m=T_m) \).}

\onslide<2->{Observation 2: Use the vector field. Given \( X_i:T_i \) we can form subtraction \( -X_i:T_i\to (T_m=T_m) \). \( X_{ij}-X_j:T_{ji}X_i-X_j=_{T_m=T_m}0 \).}

\onslide<3->{Observation 3: Use \( \ap \) of addition. We can add \( \alpha:a=_{\ccc(4)}0 \) and \( \beta:b=_{\ccc(4)}0 \) to form \( \alpha+\beta:(a+b)=_{\ccc(4)}0 \).}

\onslide<4->{Together these remove the dependency. We can compute \( \flat, I, X \) on each face independently and total them in \( T_m=T_m \).}
\end{frame}

\begin{frame}
\begin{lemma}
If \( G \) is a higher group with multiplication \( \mu:G\times G\to G \) and proof of commutativity \( \mathsf{is\underscore comm}:\pit{a,b:G}\mu(a, b)=\mu(b, a) \) then \( \mu \) induces a function \( \mu_=:(x=_G y)\times (x'=_G y')\to (\mu(x, x')=_G\mu(y,y')) \).
\end{lemma}
\begin{proof}
If \( p:x=_G y \) and \( p':x'=_G y' \), then we can define \( \mu_=(p, p') \) by concatenating the three paths 
\begin{align*} 
\mu(x',p)&:\mu(x', x)=_G\mu(x', y)\\
\mathsf{is\underscore comm}(x',y)&:\mu(x',y)=_G\mu(y,x')\\
\mu(y, p')&:\mu(y, x')=_G\mu(y, y').\qedhere
\end{align*}
\end{proof}
\end{frame}

\begin{frame}
\[\begin{aligned}
\tr_F&\defeq \tr(\partial F)&&:T_1=_{\Kzt}T_1&&\text{\alert{holonomy}}\\
\flat_F&\defeq \flat(\partial F)&&:\id=_{T_1=T_1}\tr_F&&\text{\alert{flatness}}\\
X_F&\defeq X(\partial F)&&:\tr_F(X_1)=_{T_1}X_1&&\text{\alert{swirling}}\\
\end{aligned}\]
\onslide<2->{
\begin{columns}[c]
\begin{column}{0.7\textwidth}
\begin{mydef}
The \defemph{index of the vector field \( X \) on the face \( F \)} is the integer \( I^X_F\defeq\loopy(\flat_F(X_1)\cdot X_F):\loopy(X_1=_{T_1}X_1) \).
\end{mydef}
\end{column}
\begin{column}{0.3\textwidth}
\begin{tikzpicture}
  [arrow/.style={-{Stealth[scale=1.1]}}, .style={scale=0.7}]
  \tikzset{oo/.style={circle, scale=0.6, fill=black}}
  \tikzset{ii/.style={circle, scale=0.3, fill=gray}}
  \setlength{\mylen}{2cm}
  \setlength{\mylin}{1cm}
  \node[label=center:\( Tv_1 \)] (V1) at (0, 0) {};
  \node [ii, above right=\mylin of V1,  label=right:\( v_{11} \)] (V11) {};
  \node [ii, below right=\mylin of V1,  label=right:\( v_{12} \)] (V12) {};
  \node [ii, below left=\mylin of  V1,  label=left:\( v_{13} \)] (V13) {};
  \node [ii, above left= \mylin of V1,  label=left: \( v_{14} \)] (V14) {};

  \draw[dashed] (V11) edge[ultra thick, solid, arrow, swap, "\( \flat_F(X_1) \)"] (V14);
  \draw[arrow] (V14) edge[thick, swap, color=magenta, "\( X_{12} \)"] (V13);
  \draw[arrow] (V13) edge[thick, swap, color=teal, "\( X_{23} \)"] (V12);
  \draw[arrow] (V12) edge[thick, swap, color=blue, "\( X_{31} \)"] (V11);
\end{tikzpicture}
\end{column}
\end{columns}}
\end{frame}


% \section{Results}

% \begin{frame}
% How do we make these happen?
% \begin{align*}
% \sum_F \flat_F &\leftrightsquigarrow \int_M K\,dA  \\
% \sum_F I^X_F &\leftrightsquigarrow  \sum_{i=1}^{n} \mathsf{index}_{x_i} \\
% ? &\leftrightsquigarrow \frac{1}{2\pi}\int_M K\,dA - \sum_{i=1}^{n} \mathsf{index}_{x_i} = 0 
% \end{align*}
% \end{frame}

\begin{frame}
\begin{itemize}
\item On a single face we have \( I^X_F=\loopy(\flat_F(X_1)\cdot X_F) \).
\item Map \( \flat_F(X_1) \) and \( X_F \) to angles in \( T_m=T_m \).
\item Sum over faces can be performed in \( T_m=T_m \).
\item Assume that each edge is traversed twice, once in each direction.
\item Prove that the total angle \( \sum_F X_F=0 \).
\item Leaving us \( I_\mathrm{tot}=\loopy(\flat_{\mathrm{tot}}) \).
\end{itemize}
\end{frame}

\begin{frame}{Classical proof}
\begin{columns}
\column{0.5\textwidth}
\vspace{12pt}
\begin{figure}
\includegraphics[width=0.9\textwidth]{figs/needham_triangle_circ.pdf}
\caption{{Needham,~T. (2021) Visual Differential Geometry and Forms.}}
\end{figure}
\column{0.5\textwidth}
\vspace{-12pt}
\begin{itemize}
\item The classical proof is discrete-flavored.
\item ``\( \angle Fw_{||} \)'' looked a lot like a pathover.
\item Hopf's \( \Phi \) is defined on edges, not loops. We imitated that too.
\end{itemize}
\end{columns}
\end{frame}


\begin{frame}
\begin{center}
\alert{\huge{Thank you!}}
\end{center}
\end{frame}

\section{Appendix: Conjectural dictionary}
\renewcommand{\arraystretch}{1.4}
\begin{frame}{Conjectionary}

Homotopy realization likely amounts to the shape operator.
\bigskip

\begin{quote}
[G]iven a ``cell complex'' presentation of a classical topological space, if we can convert it into both a specification for a HIT and a colimit decomposition of that space that issufficiently ``cofibrant'', then \( \shape \) will preserve that colimit and take the space to the HIT.
\end{quote}
\begin{flushright}
— Mike Shulman, \emph{Brouwer's Fixed Point Theorem in Real-Cohesive HoTT}
\end{flushright}
\end{frame}

\begin{frame}{Conjectionary}
Homotopy realization (and/or \( \shape \)) may provide the following relationships.

\begin{tabular}{p{17em}|p{17em}}
\hline
Classical octahedron & Homotopy realization, e.g. \( \oo \) \\ \hline
Classical combinatorial manifold \( M \) & \( \shape M \) \\ \hline
Derivative & \( \ap \)\\ \hline
Leibniz rule for \( f,g:M\to\rr \) & Given H-space \( (A,*) \), \( f,g:X\to A \), \( p:x=_X y \) then \( \ap(f * g)(p)=\ap (f)(p) * (g a)\cdot (f b) * \ap (g)(p) \). Because in \( A\times A \), \( (f p, g p) = (f p, \refl)\cdot(\refl, g p) \) \\ \hline
The sphere is not flat, as a \alert{pointwise} statement & Nontrivial flatness on each face
\end{tabular}
\commentout{
"The connection provides a 1-form on the total space P of the principal bundle, but on the base M only a difference of two connections is a 1-form."

The connection T in this paper maps paths in M to paths in EM(Z,1). P is a pullback of this. So what does it mean to look at the connection _on_ P? Well, we could look at a specific path p' over a path p:x=_M y. It therefore has by hypothesis a starting point x':Tx, and is a term p':tr(p)(x')=y'. It has points, hence has groupified the fibers! It furthermore supplies a path to the point, in Ty.

Two connections is two transports. I can subtract these with s to get s(tr_1, tr_2):Tx\to G

The relationship between zeros and the "Euler class" should be explained by my theorem. Study that, buddy frat.
}
\end{frame}

\begin{frame}{Conjectionary}
\begin{tabular}{p{17em}|p{17em}}
\hline
Connections being ``affine'', and not (quite) 1-forms & \( T_{ji}:T_i=T_j \) being a torsor and not (quite) a group \\ \hline
Space of connections for a given \( P \) is contractible. & Two extensions to \( \oo_1 \)...\\ \hline
\end{tabular}
\end{frame}

\begin{frame}{Conjectionary}
\begin{tabular}{p{17em}|p{17em}}
\hline
Maurer-Cartan form. & Hmm, consider the trivial connection on \( M\times G \) or \( G\to * \).\\ \hline
Gauge transformations acting on connections and maybe functions (YM) of connections. & \\ \hline
The based gauge group acts freely on connections. & \\ \hline
\end{tabular}
\commentout{
Maurer-Cartan is pretty torsorial. It doesn't care where the identity is, it will make this point's tangent space be that one. It's pretty empty at the level of points. If you were to supply an identity, it'd send this point to that?
}
\end{frame}

\begin{frame}{Conjectionary}
\begin{tabular}{p{17em}|p{17em}}
\hline
Characteristic classes. & \( B\so\to B^n\zz \).\\ \hline
Chern-Weil theory. & \( \oo\xrightarrow[]{T}B\so\to B^n\zz \). \\ \hline
Hopf fibration. & \( \oo\xrightarrow[]{?}\EMzo \).  \\ \hline
Zeros of \( X = \) Poincare dual of the Euler class. & \\ \hline
\end{tabular}
\end{frame}

\end{document}



