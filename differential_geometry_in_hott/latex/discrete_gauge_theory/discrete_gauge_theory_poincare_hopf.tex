\section{Gauss-Bonnet, Vector Fields, and the Poincaré-Hopf Theorem}

\begin{enumerate}
\item Given Chern-Weil, we can hypothesize that Pfaffians and determinants can be made from the path groupoid structure.
\item Theorem 1: Gauss-Bonnet: curvature vs euler characteristic (2-d)
\item Theorem 2: Gauss-Bonnet-Chern (2n-d, Pfaffian/Euler class, 1944)
\item Theorem 3: Poincaré-Hopf: v.f. index is indep of v.f., and is euler characteristic via a specific v.f.
\item Non-theorem: v.f. on torus is the gradient of a Morse function. Bringing in Morse theory is future work.
\end{enumerate}

% https://q.uiver.app/#q=WzAsNixbMCwyLCJNIl0sWzIsMiwiQlNeMSJdLFsyLDAsIkVTXjEiXSxbMCwwLCJQXFxzdGFja3JlbHtcXG1hdGhybXtkZWZ9fT17XFxkaXNwbGF5c3R5bGVcXHN1bV97YTpNfWYoYSl9Il0sWzAsNCwiXFxPbWVnYV4yTSJdLFsyLDQsIlxcT21lZ2FeMkJTXjEiXSxbMCwxLCJmIiwyXSxbMiwxLCJcXG1hdGhybXtwcn1fMSJdLFszLDAsIlxcbWF0aHJte3ByfV8xIl0sWzMsMiwiRWYiXSxbMywxLCIiLDIseyJzdHlsZSI6eyJuYW1lIjoiY29ybmVyIn19XSxbMCwzLCJ7XFxkaXNwbGF5c3R5bGUgWDpcXHByb2Rfe2E6TX1mKGEpfSIsMCx7ImN1cnZlIjotM31dLFs0LDUsIlxcT21lZ2FeMmYiXSxbMCw0LCJcXE9tZWdhXjIiXSxbMSw1LCJcXE9tZWdhXjIiXSxbMywwLCJcXG1hdGhybXtleHB9IiwwLHsiY3VydmUiOi0zfV1d
\begin{tikzcd}
  {\stackrel{\mathrm{def}}={\displaystyle\sum_{a:M}f(a)}} && {ES^1} \\
  \\
  M && {BS^1} \\
  \\
  {\Omega^2M} && {\Omega^2BS^1}
  \arrow["Ef", from=1-1, to=1-3]
  \arrow["{\mathrm{pr}_1}", from=1-1, to=3-1]
  \arrow["{\mathrm{exp}}", curve={height=-18pt}, from=1-1, to=3-1]
  \arrow["\lrcorner"{anchor=center, pos=0.125}, draw=none, from=1-1, to=3-3]
  \arrow["{\mathrm{pr}_1}", from=1-3, to=3-3]
  \arrow["{{\displaystyle X:\prod_{a:M}f(a)}}", curve={height=-18pt}, from=3-1, to=1-1]
  \arrow["f"', from=3-1, to=3-3]
  \arrow["{\Omega^2}", from=3-1, to=5-1]
  \arrow["{\Omega^2}", from=3-3, to=5-3]
  \arrow["{\Omega^2f}", from=5-1, to=5-3]
\end{tikzcd}

To prove the Gauss-Bonnet theorem one needs a notion of \emph{Euler characteristic} and a notion of curvature. To prove the Poincaré-Hopf theorem, one needs a notion of \emph{the index of a vector field}, and again, a notion of curvature. To prove both and relate them it helps to equate the Euler characteristic to the index of a special vector field. 

\begin{mydef}
A vector field is a section \( X \) of the disk bundle. A zero of \( X \) is a vertex \( x \) where \( X(x)=x:\disk(x) \), the center of the disk.
\end{mydef}

The following proofs follow the explanations given in \cite{needham}. But as we might hope, the proofs can be brought into HoTT in a much abbreviated form!

Note that \( X \) is a pointing of all the fibers. So a second vector field can be lifted to the classifying map to pointed types, which is contractible. That's really all that's going on in these classical proofs!

\begin{mydef}
The index of a vector field at a zero is given by transport composed with exp.
\end{mydef}

\begin{mythm}
The total index is independent of the vector field.
\end{mythm}

\begin{mythm}
(Poincaré-Hopf) The total index is equal to the total curvature.
\end{mythm}

\begin{mythm}
The total index is equal to the Euler characteristic.
\end{mythm}



