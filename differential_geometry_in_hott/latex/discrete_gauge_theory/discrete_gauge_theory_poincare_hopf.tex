\section{Leibniz, Gauss-Bonnet, Poincaré-Hopf}

\subsection{The Leibniz (product) rule}

The Leibniz rule for exterior differentiation states that if \( f, g:M\to \rr \) are two smooth functions to the real numbers then \( d(fg) = fdg + gdf \). Here \( fg \) is the function formed by taking the pointwise product of \( f \) and \( g \). This is an interaction between multiplication in \( \rr \) and the action on vectors of smooth functions (the 1-forms \( df \) and \( dg \)). 

To examine this situation in HoTT we need type-theoretic functions \( f, g:M\to B \) from some type \( M \) to a central H-space \( B \). Let \( \mu:B\to B\to B \) be the H-space multiplication. How does \( \mu \) act on paths? Suppose we have \( a, a', b, b':B \) and \( p:a=_B a', q:b=_B b' \). Then we also have homotopies \( \mu(p, -):\mu(a, -)=_{B\to B}\mu(a', -) \) and \( \mu(-,q):\mu(-,b)=_{B\to B}\mu(-,b'). \) Since \( \mu(a, -):B=B \) is an (unpointed) equivalence of \( B \), and similarly for \( \mu(b, -) \) and so on, this data assembles into the following diagram of higher groupoid morphisms:

\begin{center}
% https://q.uiver.app/#q=WzAsMyxbMiwwLCJCIl0sWzAsMCwiQiJdLFs0LDAsIkIiXSxbMSwwLCJcXG11KGEsLSkiLDAseyJjdXJ2ZSI6LTN9XSxbMSwwLCJcXG11KGEnLC0pIiwyLHsiY3VydmUiOjN9XSxbMCwyLCJcXG11KC0sYikiLDAseyJjdXJ2ZSI6LTN9XSxbMCwyLCJcXG11KC0sYicpIiwyLHsiY3VydmUiOjN9XSxbMyw0LCJcXG11KHAsLSkiLDIseyJzaG9ydGVuIjp7InNvdXJjZSI6MjAsInRhcmdldCI6MjB9fV0sWzUsNiwiXFxtdSgtLHEpIiwyLHsic2hvcnRlbiI6eyJzb3VyY2UiOjIwLCJ0YXJnZXQiOjIwfX1dXQ==
\begin{tikzcd}
  B && B && B
  \arrow[""{name=0, anchor=center, inner sep=0}, "{\mu(a,-)}", curve={height=-18pt}, from=1-1, to=1-3]
  \arrow[""{name=1, anchor=center, inner sep=0}, "{\mu(a',-)}"', curve={height=18pt}, from=1-1, to=1-3]
  \arrow[""{name=2, anchor=center, inner sep=0}, "{\mu(-,b)}", curve={height=-18pt}, from=1-3, to=1-5]
  \arrow[""{name=3, anchor=center, inner sep=0}, "{\mu(-,b')}"', curve={height=18pt}, from=1-3, to=1-5]
  \arrow["{\mu(p,-)}"', shorten <=5pt, shorten >=5pt, Rightarrow, from=0, to=1]
  \arrow["{\mu(-,q)}"', shorten <=5pt, shorten >=5pt, Rightarrow, from=2, to=3]
\end{tikzcd}
\end{center}

And so the two homotopies can be horizontally composed to give a path \[ \mu(p,-)\star\mu(-,q): \mu(a, b)=\mu(a',b'). \] Horizontal composition is given by \[\mu(p,-)\star\mu(-,q)\defeq(\mu(p,-)\cdot_r \mu(-,b))\cdot(\mu(a', -)\cdot_l\mu(-, q))\] where \[ \mu(p,-)\cdot_r\mu(-,b):\mu(a,b)=\mu(a',b) \] and \[ \mu(a',-)\cdot_l\mu(-,q):\mu(a',b)=\mu(a',b') \] are defined by path induction.  See the HoTT book Theorem 2.1.6 on the Eckmann-Hilton argument\cite{hottbook}.

We can recognize the process of using whiskering to form horizontal composition in the Leibniz rule. 

Quick aside: moving from infinitesimal calculus to finite groupoid algebra actually involves two changes. The first is the change from vectors to paths, forms to functions and so on. But it's also the case that tangent vectors have just the one basepoint, whereas paths have two endpoints. You can see this play out in this example, where \( a \) and \( a' \) were distinct points (and \( b \) and \( b' \)).

The horizontal composition we build lives entirely in \( B \) and we didn't make use of \( M \) yet. The Leibniz rule will be a pointwise version of what's going on in \( B \). Denote by \( \mu\circ(f,g):M\to B \) the map which sends \( x\mapsto \mu(f(x),g(x)) \).

\begin{mylemma}
Given \( f, g:M\to B \) and \( p:x=_M y \) then 
\begin{align*}
 \ap(\mu\circ(f,g))(p)&=\mu(f(p),-)\star\mu(-,g(p))\\
 &=\left[\mu(f(p),-)\cdot_r \mu(-,g(x))\right]\cdot \left[\mu(f(y),-)\cdot_l\mu(-,g(p))\right]\\
 &:\mu(f(x),g(x))=\mu(f(y),g(y))
\end{align*}
\end{mylemma}

\subsection{The total curvature}


\subsection{The total index}

\subsection{Equality of total index and total curvature}

\subsection{Identification with Euler characteristic}

