\section{Introduction: Discrete differential geometry}

The observation that sparks the following discussion is this: if we can manage to reformulate differential geometry in discrete terms (i.e. finite, without infinitesimals) then we may also be able to construct it synthetically in homotopy type theory. Furthermore, if we do capture geometry in HoTT then there's a chance that it can become clearer and smaller. We would then have new tools, a new audience, and a new program to (re)explore geometry, gauge theory, low dimensional topology, and mathematical physics.

Applied mathematicians and computer scientists have been developing discrete differential geometry (DDG) for many years. The 2003 Ph.D. thesis of Anil Hirani \cite{hiranidec} (see also the multi-author follow-up \cite{desbrundec}) defines finite versions of vector fields, differential forms, the wedge product, the Hodge star, and several differential operators (exterior derivative, div, grad, curl, Laplace-Beltrami, Lie derivative). Hirani and others cite Whitney's 1957 book \emph{Geometric Integration Theory}\cite{whitney1957} which develops a theory of cochains by integrating smooth forms over chains. In 2004 Melvil Leok, Jerrold Marsden, and Alan Weinstein \cite{leok} defined discrete connections on principal bundles. This is probably the work most spiritually similar to this paper. The motivation for the above constructions was applied mathematics: modeling the differential equations of mechanics and fluid mechanics with the so-called ``finite element'' methods. The theory has been adopted and extended by the computer graphics community as well (see Keenan Crane's course notes \cite{crane_ddg} for a gentle survey).

The applied category theory community has begun to develop category theoretic foundations and software libraries to increase the reusability and compositionality of finite element methods in science and engineering problems. See for example recent work to bring discrete exterior calculus into the AlgebraicJulia library \cite{morris_decapodes} \cite{patterson_diffeq}.

For these classically-minded applied mathematicians DDG is defined on combinatorial manifolds such as simplicial complexes, of any finite dimension. The 0-cells play the role of points, the 1-cells are path segments, and so on. They define \( n \)-forms as functions on the \( n \)-dimensional faces of the manifold into the real numbers, which is then extended by linearity to arbitrary \( n \)-chains. Exterior differentiation is defined by Stokes theorem (which is no longer a theorem in this setting), by which we mean the following. 

\begin{mydef}
(Exterior derivative in DDG.) Let \( \omega \) be an \( n-1 \)-form on a combinatorial manifold \( M \), and let \( \Omega \) be an \( n \)-face of \( M \). Let \( \partial \) be the boundary operator on faces. The exterior derivative \( d \) is defined by 
\[ 
 d\omega(\Omega) = \omega(\partial\Omega).
\]
\end{mydef}

We will take the following path. We will use DDG as a stepping stone between smooth geometry and HoTT. We will learn how to see the finite analogues of infinitesimal constructions, and then use the syntax of type theory to re-present them. It won't always be a two-step process! Sometimes even the smooth arguments are made with finite combinatorial methods, and then a limit is taken. This is especially true where the Gauss-Bonnet and Poincaré-Hopf theorems are concerned.
\[\mathrm{Differential\ geometry}\longrightarrow\mathrm{Discrete\ differential\ geometry}\longrightarrow \mathrm{Homotopy\ type\ theory}\]
