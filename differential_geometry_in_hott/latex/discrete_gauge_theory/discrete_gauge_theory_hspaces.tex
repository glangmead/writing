\section{Torsors}

Classical geometry tells us to look for an appropriate type of torsors to map into. Homotopy type theory tells us to look for a univalent fibration to map into. The type of torsors is not a connected component of the universe, because torsors have additional structure on top of an underlying type. So we'll need to resolve that.

\begin{mydef}
Let \( G \) be a group (a set with the usual classical structure and properties). A \emph{\( G \)-set} is a set \( X \) equipped with a homomorphism \( \phi:G\to\Aut(X) \). If in addition we have a term
\[ 
\mathsf{is\_torsor}:||X||_{-1}\times \pit{g:G}\mathsf{is\_equiv}(\phi(-,x):G\to X)
\] then we call this data a \emph{\( G \)-torsor}. Denote the type of \( G \)-torsors by \( TG \).
\end{mydef}

If \( (X,\phi),(Y,\psi):TG \) then a \( G \)-equivariant map is a function \( f:X\to Y \) such that \( f(\phi(g,x))=\psi(g,f(x)) \). Denote the type of \( G \)-equivariant maps by \( X\to_G Y \).

\begin{mylemma}
There is a natural equivalence \( (X=_{TG}Y) \simeq (X\to_G Y) \).\qed
\end{mylemma}

Denote by \( * \) the torsor given by \( G \) actions on its underlying set by left-translation. This serves as a basepoint for \( TG \) and we have a group isomorphism \( \Omega TG\simeq G \).

\begin{mylemma}
A \( G \)-set \( (X,\phi) \) is a \( G \)-torsor if and only if there merely exists a \( G \)-equivariant equivalence \( *\to_G X \).\qed
\end{mylemma}

\begin{mycor}
The pointed type \( (TG,*) \) is a \( \K(G,1) \).\qed
\end{mycor}

\subsection{Univalent replacement for torsors}

The homotopy type theory of cohomology and bundles tells us that the type of \( G \)-bundles on a type \( M \) is the type \( M\to\K(G,1) \). So we will start there as well. But this is a type of structured types, a connected component of \( G \)-sets rather than a connected component of the universe. The paths \emph{in the universe} between two \( G \)-sets is equivalent to the type of equivalences between the \emph{underlying types}, not just the equivariant equivalences.

We'll resolve this problem with the following discussion, following Scoccola\cite{sco}. We will state the definitions and theorems for a general \( \K(G,n) \) but we will be focusing on \( n=1 \) in this note.

\begin{mydef}
Let \( \EM(G,n)\defeq \BAut(\K(G,n))\defeq \sit{Y:\uni}||Y\simeq \K(G,n)||_{-1}\). A \( \K(G,n)-bundle \) on a type \( M \) is the fiber of a map \( M\to\EM(G,n) \).
\end{mydef}

Scoccola uses the action on the universe of suspension and of forgetting a point to form the composition 
\[ 
\EM(G,n)\xrightarrow[]{\Sigma} \EM_{\bullet\bullet}(G,n)\xrightarrow[]{F_\bullet}\EMp(G,n)
\]
from types to types with two points (north and south), to pointed types (by forgetting the south point).

\begin{mydef}
Given \( f:M\to\EM(G,n) \), the \emph{associated action of \( M \) on \( G \)}, denoted by \( f_\bullet \) is defined to be \( f_\bullet=F_\bullet\circ\Sigma\circ f \).
\end{mydef}

\begin{mythm}
(Scoccola\cite{sco} Proposition 2.39). A \( \K(G,n) \) bundle \( f:M\to\EM(G,n) \) is equivalent to a map in \( M\to\K(G,n+1) \), and so is a principal fibration, if and only if the associated action \( f_\bullet \) is contractible.
\end{mythm}

And so we can continue to work with the classifying space \( \EM(G,1) \) and study \( \K(G,1) \)-fibrations, and then later add the extra propositional requirement when it is needed to prove that we are working with \( M\to\K(G,2) \).

\subsection{Pathovers}
Suppose we have \( T:M\to\EMzo \). Paths in a sigma type \( \sit{x:M}T(x) \) are given by pairs of paths: a path \( p:x=_M y \) in the base, and a pathover \( p':\tr_p(x')=_{T(y)}y' \) between \( x':T(x) \) and \( y':T(y) \) in the fibers. We can't directly compare \( x' \) and \( y' \) since they are of different types, so we apply transport to one of them (which is asymmetrical, but equivalent to the alternatives). We say \( p' \) lies over \( p \). 

The individual fibers of \( T \) are polygons (the link of the vertex of which it is the fiber). Given a path \( p:x=_M y \) in \( M \), one of its pathovers consists of a path in \( T(y) \). And given a face in \( M \), a faceover is a homotopy from a pathover to refl.

