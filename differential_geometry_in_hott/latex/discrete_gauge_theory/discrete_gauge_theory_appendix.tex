\section{Appendix}
\subsection{Adding and removing points from polygons}
\label{sec:polygons}

The 1-type \( C_3 \) we created earlier by pushing out the combinatorial data of a set-based simplicial triangle is clearly an example of a type of marked presented polygons \( C_n, n:\nn \). The standard HoTT circle itself is usually given as a HIT and is a non-example of a combinatorial manifold since it lacks the second vertex of the edge:

\begin{mydef}
The higher inductive type \( \so \) which we can also call \( C_1 \):
\begin{align*}
\so &:\Type \\
\mathsf{base}&:\so \\
\mathsf{loop}&:\mathsf{base}=\mathsf{base}
\end{align*}
\end{mydef}

Denote by \( \Gon \) the set of marked presented \( n \)-gons for some natural number \( n \). We'll see below that the realization of an \( n \)-gon is a mere circle, i.e. we have a forgetful map \( \Gon\to \EMzo \).

Recall that given functions \( \phi,\psi:A\to B \) between two arbitrary types we can form a type family of paths \( \alpha:A\to\uni \) by \( \alpha(a)\defeq(\phi(a)=_B\psi(a)) \). Transport in this family is given by concatenation as follows, where \( p:a=_A a' \) and \( q:\phi(a)=\psi(a) \) (see Figure~\ref{fig:transport_family_of_paths}):
\[ 
\tr(p)(q) = \phi(p)^{-1}\cdot q\cdot \psi(p)
\]
which gives a path in \( \phi(a')=\psi(a') \) by connecting dots between the terms \( \phi(a'), \phi(a), \psi(a), \psi(a') \). This relates a would-be homotopy \( \phi\sim\psi \) specified at a single point, to a point at the end of a path. We will use this to help construct such homotopies.

\begin{figure}[h]
\centering
\begin{tikzpicture}[
node distance = 20mm and 20mm,
V/.style = {circle, fill, draw=black, inner sep=1pt},
every edge quotes/.style = {auto},
arrow/.style={->,semithick}
]
\begin{scope}[nodes=V]
  \node[label=above left:\( \phi(a) \)] (1) {};
  \node[label=above right:\( \phi(a') \)] (2) [right=of 1]  {};
  \node[label=below right:\( \psi(a') \)] (3) [below=of 2]  {};
  \node[label=below left:\( \psi(a) \)] (4) [below=of 1]  {};
  \node[label=below:\( a \)] (5) [below=of 4]  {};
  \node[label=below:\( a' \)] (6) [below=of 3]  {};
\end{scope}
\draw[arrow]
        (2)  edge[swap, "\( \phi(p)^{-1} \)"] (1)
        (4)  edge["\( \psi(p) \)"] (3)
        (1)  edge[swap, "\( q \)"] (4)
        (5)  edge["\( p \)"] (6);
\end{tikzpicture}
\caption{Transport along \( p \) in the fibers of a family of paths. The fiber over \( a \) is \( \phi(a)=\psi(a) \) where \( \phi,\psi:A\to B \).}
\label{fig:transport_family_of_paths}
\end{figure}

\begin{mylemma}\label{lem:addpoints}
Let \( C_n \) be the marked presented polygon 1-type with \( n \) vertices. Then \( C_2\simeq C_1 \) and in fact \( C_n\simeq C_{n-1} \).
\end{mylemma}
\begin{proof}
(Compare to \cite{hottbook} Lemma 6.5.1.) In the case of \( C_1 \) we will denote its constructors by \( \base \) and \( \loopo \). For \( C_2 \) we will denote the points by \( v_1, v_2 \) and the edges by \( \ell_{12}, r_{21} \). For \( C_3 \) and higher we will denote the points by \( v_1,\ldots,v_n \) and the edges by \( e_{i,j}:v_i=v_j \) where \( j=i+1 \) except for \( e_{n,1} \). 

First we will define \( f:C_2\to C_1 \) and \( g:C_1\to C_2 \), then prove they are inverses.
\begin{align*}
f(v_1)=f(v_2)&=\base &\quad g(\base)&=v_1\\
f(\ell_{12})&=\loopo&\quad g(\loopo)&=\ell_{12}\cdot r_{21}\\
f(r_{21}) &= \refl_{\base}& & \\
\end{align*}

We need to show that \( f\circ g\sim \id_{C_1} \) and \( g\circ f\sim\id_{C_2} \).
Think of \( f \) as sliding \( v_2 \) along \( r_{21} \) to coalesce with \( v_1 \). This may help understand why the unfortunately intricate proof is working.

We need terms \( p:\pit{a:C_1}f(g(a))=a \) and \( q:\pit{a:C_2}g(f(a))=a \). We will proceed by induction, defining appropriate paths on point constructors and then checking a condition on path constructors that confirms that the built-in transport of these type families respects the definition on points.

Looking first at \( g\circ f \), which shrinks \( r_{21} \), we have the following data to work with:
\begin{align*}
g(f(v_1))=g(f(v_2))&=v_1\\
g(f(\ell_{12}))&=\ell_{12}\cdot r_{21}\\
g(f(r_{21})) &= \refl_{v_1}.
\end{align*}
We then need to supply a homotopy from this data to \( \id_{C_2} \), which consists of a section and pathovers over \( C_2 \):
\begin{align*}
p_1&:g(f(v_1))=v_1\\
p_2&:g(f(v_1))=v_2\\
H_\ell&:\tr(\ell_{12})(p_1)=p_2\\
H_r&:\tr(r_{21})(p_2)=p_1.
\end{align*}
which simplifies to
\begin{align*}
p_1&:v_1=v_1\\
p_2&:v_1=v_2\\
H_\ell&:g(f(\ell_{12}))^{-1}\cdot p_1\cdot \ell_{12}=p_2\\
H_r&:=g(f(r_{21}))^{-1}\cdot p_2\cdot r_{21}= p_1
\end{align*}
and then to 
\begin{align*}
p_1&:v_1=v_1\\
p_2&:v_1=v_2\\
H_\ell&:(\ell_{12}\cdot r_{21})^{-1}\cdot p_1\cdot \ell_{12}=p_2\\
H_r&:\refl_{v_1}\cdot p_2\cdot r_{21}= p_1
\end{align*}

To solve all of these constraints we can choose \( p_1\defeq\refl_{v_1} \), which by consulting either \( H_\ell \) or \( H_r \) requires that we take \( p_2\defeq{r_{21}}^{-1}\).

Now examining \( f\circ g \), we have
\begin{align*}
f(g(\base))&=\base&\\
f(g(\loopo))&=f(\ell_{12}\cdot r_{21})=\loopo
\end{align*}
and so we have an easy proof that this is the identity.

The proof of the more general case \( C_n \simeq C_{n-1}\) is very similar. Take the maps \( f:C_n\to C_{n-1} \), \( g:C_{n-1}\to C_n \) to be
\begin{align*}
f(v_i)=v_i&\quad(i=1,\ldots,n-1) & g(v_i)&=v_i&\quad(i=1,\ldots,n-1)\\
f(v_n)=v_1&\quad& g(e_{i,i+1})&=e_{i,i+1}&\quad(i=1,\ldots,n-2)\\
f(e_{i,i+1})=e_{i,i+1}&\quad(i=1,\ldots,n-1)& g(e_{n-1,1})&=e_{n-1,n}\cdot e_{n,1}&\\
f(e_{n-1,n})=e_{n-1,1}&&&&\\
f(e_{n,1})=\refl_{v_1}&&&&
\end{align*}
where \( f \) should be thought of as shrinking \( e_{n,1} \) so that \( v_n \) coalesces into \( v_1 \).

The proof that \( g\circ f\sim\id_{C_n} \) proceeds as follows: the composition is definitionally the identity except 
\begin{align*}
g(f(v_n))&=v_1\\
g(f(e_{n-1,n}))&=e_{n-1,n}\cdot e_{n,1}\\
g(f(e_{n,1}))&= \refl_{v_1}.
\end{align*}
Guided by our previous experience we choose \( {e_{n,1}}^{-1}:g(f(v_n))=v_n \), and define the pathovers by transport.

The proof that \( f\circ g\sim\id_{C_{n-1}} \) requires only noting that \( f(g(e_{n-1,1}))=f(e_{n-1,n}\cdot e_{n,1})=e_{n-1,1}\cdot\refl_{v_1}=e_{n-1,1} \).
\end{proof}

\begin{mycor}
\label{cor:gon}
All polygons are equivalent to \( \so \), i.e. we have a term in \( \pit{n:\nn}||C_n=S^1|| \), and hence \( \Gon \) is a subtype of \( \EMzo \).
\end{mycor}
\begin{proof}
We can add \( n-1 \) points to \( S^1 \) and use Lemma~\ref{lem:addpoints}.
\end{proof}

\begin{mydef}
For \( k:\nn \) define \( m_k:\Gon\to\Gon \) where \( m_k:C_n\to C_{kn} \) adds \( k \) vertices between each of the original verticies of \( C_n \).
\end{mydef}

With \( m_k \) we can start with a collection of pentagons and hexagons and make the collection homogeneous: by applying \( m_6 \) to the pentagons and \( m_5 \) to the hexagons we obtain a collection of 30-gons. This will be useful when we work more with the link function.

