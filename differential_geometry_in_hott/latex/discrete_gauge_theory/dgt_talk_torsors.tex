\section{Torsors}

\begin{frame}
What type families \( \mm\to\uni \) will we consider? Families of \alert{torsors}, also called \alert{principal bundles}.
\end{frame}

\begin{frame}{Torsors}
Let \( G \) be a (higher) group.
\onslide<2->{\begin{definition}
\begin{itemize}
\item<2-> A \defemph{right \( G \)-object} is a type \( X \) equipped with a homomorphism \( \phi:G^{\mathrm{op}}\to\Aut(X) \).
\item<3-> \( X \) is furthermore a \alert{\( G \)-torsor} if it is inhabited and the map \( (\pr_1, \phi):X\times G\to X\times X \) is an equivalence.
\item<4-> The inverse is \( (\pr_1, s) \) where \( s:X\times X\to G \) is called \alert{subtraction} (when \( G \) is commutative).
\item<5-> Let \alert{\( BG \)} be the type of \( G \)-torsors.
\item<5-> Let \( \reg{G} \) be the \( G \)-torsor consisting of \( G \) acting on itself on the right.
\end{itemize}
\end{definition}}
\end{frame}

\begin{frame}{Facts}
\begin{enumerate}
\item<2-> \( \loopy(BG,\reg{G}) \simeq G \) and composition of loops corresponds to multiplication in \( G \).
\item<3-> \( BG \) is connected.
\item<4-> 1 \& 2 \( \implies \) \( BG \) is a \( \K(G,1) \).
\end{enumerate}
\onslide<5->{See the Buchholtz et. al. H-spaces paper for more.}
\end{frame}

\begin{frame}{How to map into \( BS^1 \)}
To construct maps into \( BS^1 \) we \alert{lift} a family of \alert{mere circles}.
% https://q.uiver.app/#q=WzAsNSxbMSwwLCJCU14xIl0sWzIsMCwiXFxCQXV0IChTXjEpXFxzdGFja3JlbHtcXG1hdGhybXtkZWZ9fXs9fVxcc3VtX3tZOlxcdW5pfXx8WT1TXjF8fF97LTF9Il0sWzMsMCwiXFx1bmkiXSxbMiwyLCJcXG1tIl0sWzAsMSwiXFx0ZXh0e2ZhbWlsaWVzIG9mOn0iXSxbMywwLCJcXG1hdGhybXt0b3Jzb3JzfSJdLFszLDEsIlxcc3Vic3RhY2t7XFxtYXRocm17bWVyZX1cXFxcIFxcbWF0aHJte2NpcmNsZXN9fSIsMV0sWzMsMiwiXFxtYXRocm17dHlwZXN9IiwyXSxbMCwxXSxbMSwyXV0=
\[\begin{tikzcd}[ampersand replacement=\&, row sep=small]
  \& {BS^1} \& {\alert{\BAut (S^1)}\stackrel{\mathrm{def}}{=}\sum_{Y:\uni}||Y=S^1||_{-1}} \& \uni \\
  {\text{families of:}} \\
  \&\& \mm
  \arrow[from=1-2, to=1-3]
  \arrow[from=1-3, to=1-4]
  \arrow["{\mathrm{torsors}}", from=3-3, to=1-2]
  \arrow["\begin{array}{c} \substack{\mathrm{mere}\\ \mathrm{circles}} \end{array}"{description}, from=3-3, to=1-3]
  \arrow["{\mathrm{types}}"', from=3-3, to=1-4]
\end{tikzcd}\]
We will assume we have such a lift when we need it. (Remark: the lift is a choice of \alert{orientation}.)

\onslide<2->{
Other names:
\begin{itemize}
\item \( \BAut (S^1)=BO(2)=\EMzo \) (where \( \EM(G,n)\defeq \BAut(\K(G,n)) \))
\item \( BS^1=BSO(2)=\Kzt \)
\end{itemize}}
\end{frame}

% \begin{frame}{A connected component of \( \uni \)?}
% \begin{definition}
% The \alert{type of Eilenberg-Mac Lane spaces \( \EM(G,n) \)} is the connected component of \( \K(G,n) \):
% \[ \EM(G,n)\defeq \BAut(\K(G,n))\defeq \sit{Y:\uni}||Y\simeq \K(G,n)||_{-1} \]
% \end{definition}
% It is a property of a map \( f:A\to\EM(G,n) \) to factor through \( \K(G,n+1) \). See the Scoccola paper.
% \end{frame}

% \begin{frame}{Coincidences of 2 dimensions}
% \begin{itemize}
% \item \( S^1 \) is a \( \Kzo \) since \( \loopy(S^1, \base)\simeq \zz \).
% \item So \( \EMzo \) is a type of \alert{mere circles}.
% \item But \( S^1=_{\EMzo}S^1 \) contains an order 2 \alert{flip}, so \( \not\simeq S^1 \).
% \item For a map \( f:A\to\EMzo \) to factor through  \( \Kzt \), it must somehow avoid flips.
% \item This deserves to be called \alert{orientability}.
% \item \( \link:\mm_0\to\EMzo \) is a great starting point.
% \end{itemize}
% \end{frame}
% 

