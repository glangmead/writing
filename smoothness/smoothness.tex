\documentclass[12pt]{article}
\usepackage[margin=1in]{geometry}
\usepackage[utf8]{inputenc}
\usepackage{amsfonts}
\usepackage{amsmath}
\usepackage{amsthm}
\usepackage{tikz-cd}

%%% Common categories
\newcommand{\tp}{\;\mathrm{type}}
\newcommand{\Hom}{\mathrm{Hom}}
\newcommand{\G}{\Gamma}
\newcommand{\pit}[1]{\prod_{(#1)}} % pi-type
\newcommand{\pitxa}{\pit{x:A}}
\newcommand{\sit}[1]{\sum_{(#1)}} % sigma-type
\newcommand{\ent}{\vdash}
\newcommand{\adj}{\dashv}
\newcommand{\refl}{\ensuremath{\textsf{refl}}}
\newcommand{\ap}{\ensuremath{\textsf{ap}}}
\newcommand{\ind}{\ensuremath{\textsf{ind}}}
\newcommand{\lift}{\ensuremath{\textsf{lift}}}
\newcommand{\inv}{\ensuremath{\textsf{inv}}}
\newcommand{\concat}{\ensuremath{\textsf{concat}}}
\newcommand{\transport}{\ensuremath{\textsf{transport}}}
\renewcommand{\sec}{\ensuremath{\textsf{sec}}}
\newcommand{\retr}{\ensuremath{\textsf{retr}}}
\newcommand{\total}{\ensuremath{\textsf{total}}}
\newcommand{\isequiv}{\ensuremath{\textsf{is\_equiv}}}
\newcommand{\fib}{\ensuremath{\textsf{fib}}}
\newcommand{\id}{\ensuremath{\text{id}}}
\newcommand{\judgeq}{\ensuremath{:\equiv}}
\newcommand{\reflfx}{\ensuremath{\refl_{f(x)}}}

\newcommand{\rr}{\ensuremath{\mathbb{R}}}
\newcommand{\rrn}{\ensuremath{\mathbb{R}^n}}
\newcommand{\rrm}{\ensuremath{\mathbb{R}^m}}
\newcommand{\rrx}{\ensuremath{\mathbb{R}[x]/x^2}}
\newcommand{\rry}{\ensuremath{\mathbb{R}[y]/y^2}}
\newcommand{\cc}{\ensuremath{\mathbb{C}}}
\newcommand{\nn}{\ensuremath{\mathbb{N}}}
\newcommand{\dd}{\ensuremath{\mathbb{D}}}
\newcommand{\vv}{\ensuremath{\mathbb{V}}}
\newcommand{\cinfty}{\ensuremath{C^{\infty}}}
\newcommand{\smfd}{\textsf{SmoothMfd}}
\newcommand{\calg}{\textsf{CAlg}_{\rr}}
\newcommand{\cart}{\textsf{CartSp}}
\newcommand{\formalcart}{\textsf{FormalCartSp}}
\newcommand{\formalsmoothset}{\textsf{FormalSmoothSet}}
\newcommand{\smoothset}{\textsf{SmoothSet}}
\newcommand{\setcat}{\textsf{Set}}
\newcommand{\psh}[1]{\textsf{Psh}(#1)}
\newcommand{\sh}[1]{\textsf{Sh}(#1)}
\newcommand{\pshcart}{\psh{\cart}}
\newcommand{\rmodal}{\Re}
\newcommand{\imodal}{\Im}

\newcommand{\gc}[1]{\marginpar{\bf $\leftarrow$ {#1}}}
%\newcommand{\gc}[1]{}

\newtheorem{mydef}{Definition}
\newtheorem{mythm}{Theorem}
\newtheorem{mylemma}{Lemma}
\newtheorem{myprop}{Proposition}
\newtheorem{myclaim}{Claim}

\title{Categories and modalities for smoothness}
\author{Greg Langmead}
\begin{document}
\maketitle
\begin{abstract}
Differential cohesion and modal HoTT promise to give access to constructions from differential geometry. In this series of talks I will tell a story that takes us from the classical definition of a smooth manifold to differential cohesion in a category where we can interpret modal HoTT. I will assume only basic topology and category theory and will provide details that are sometimes waved away. I will also motivate the study of smooth manifolds as a worthy field with surprising results and tough open questions such as the smooth four-dimensional Poincaré conjecture.
\end{abstract}
\tableofcontents
\section{Introduction}\label{sec:introduction}
As someone ``classically trained'' in differential geometry, my recent adventures in homotopy type theory have made me very excited to explore an entirely new angle on a topic I love. It is the case today that modal HoTT allows us to access smoothness and prove theorems about manifolds, Lie groups and more. In order to participate in this new and exciting field, someone like me needs a bridge. More than one, in fact. The current document is intended as one of those bridges. Starting from the classical definition of a smooth manifold and the category containing these objects and the smooth maps between them, we will define several enlargements of the category until we arrive at something where modal HoTT can be interpreted.

To make the most of this process, we will always try to keep track of where smoothness is entering in. It's not always obvious!

\begin{mydef}\label{def:smoothmfd}
A smooth manifold is a Hausdorff topological space $M$ with countable basis, and with a maximal smooth atlas. An atlas is a covering of $M$ by open sets $U_i$, each of which is the image of a homeomorphism $f_i:\mathbb{R}^n\to M$ (called a \emph{chart}), such that $f_j^{-1}\circ f_i$ is smooth wherever it is defined. Two atlases are compatible if their union is again an atlas. The maximal atlas is the union of all compatible atlases.
\end{mydef}

Note that this definition expresses $M$ as a colimit of a diagram of arrows from $\rr^n\to M$, more specifically a pushout. More on this shortly.

A continuous function $f:M\to N$ between two smooth manifolds is defined to be smooth if the appropriate composition with charts is a smooth function between cartesian spaces.

The category \smfd\ has objects the smooth manifolds of any finite dimension, and morphisms the smooth maps.

This category, innocuous though it seems, has some strange properties! There are topological manifolds that are missing from it, and some that are way over-represented! You can't just take a topological manifold and assume you can give it a unique maximal smooth atlas and stick it into \smfd.

The table below contains some examples.

\begin{center}
\label{table:smoothstructures}
\begin{tabular}{|l|l|}
\hline
kind of manifold & number of inequivalent smooth atlases \\ \hline
dim $\leq 3$ & 1 \\ \hline
dim $\geq 5$ & $\geq 0$, finite \\ \hline
$\rr^n, n\neq 4$ & 1 \\ \hline
$\rr^4$ & uncountably many \\ \hline
$S^4$ & unknown, $\geq 1$ \\ \hline
dim $= 4$, compact without boundary & $\geq 0$, at most countably many \\ \hline
dim $= 4$, otherwise & $\geq 1$, at most uncountably many \\ \hline
\end{tabular}
\end{center}

The claim that $S^4$ has exactly one smooth structure is known as the four-dimensional smooth Poincaré conjecture. Its status is ``extremely unknown''. For more of the beautiful story around four-dimensional topology and smooth structures see the accessible and detailed book by Scorpan \cite{scorpan_wild_2005}.

The spheres behave surprisingly as well.

\begin{center}
\label{table:smoothspheres}
\begin{tabular}{|l|p{4cm}|}
\hline
$n$ & number of inequivalent smooth atlases on $S^n$ \\ \hline
1 & 1 \\ \hline
2 & 1 \\ \hline
3 & 1 \\ \hline
4 & $\geq 1$ \\ \hline
5 & 1 \\ \hline
6 & 1 \\ \hline
7 & 28 \\ \hline
8 & 2 \\ \hline
9 & 8 \\ \hline
10 & 6  \\ \hline
11 & 992  \\ \hline
12 & 1 \\ \hline
13 & 3 \\ \hline
14 & 2 \\ \hline
15 & 16256 \\ \hline
16 & 2 \\ \hline
17 & 16 \\ \hline
18 & 16 \\ \hline
19 & 523264 \\ \hline
20 & 24 \\ \hline
\end{tabular}

(from https://en.wikipedia.org/wiki/Differential\_structure)
\end{center}

Is \smfd\ a nice category? It does have finite products (cartesian product) and coproducts (disjoint unions). Manifolds are themselves certain colimits. However in this category does not have all limits or colimits, nor is it cartesian closed.

For example, consider the equalizer $\left\{(x, y) | xy = 0\right\}\subset \rr^2$. As a set this must be equal to the union of the $x$-axis and $y$-axis. But this cannot be given the structure of a topological manifold nor a smooth one, due to the failure to be homeomorphic to any $\rr^n$ at the origin. Full argument: the forgetful functor $U:\smfd\to\setcat$ preserves limits and colimits as it has both left and right adjoints. The equalizer as a set is this union of axes, and so the manifold must have that underlying set. But this set cannot be promoted to a smooth manifold.

Consider the coequalizer $\left(\mathbb{R}^2 \coprod \mathbb{R}\right)/(0,0)\sim 0$. Similarly this cannot be given the structure of a topological manifold nor a smooth one, also at the origin.

Is \smfd\ cartesian closed, i.e. does it contain function spaces of smooth maps? This has been attempted with some success, but we'll skip over that story because we're going to do it with sheaves later. In fact right now.

\section{Sheaves on \cart}\label{sec:sheaves}

My take on presheaves and sheaves is inspired by Daniel Dugger's informal notes \emph{Sheaves and Homotopy Theory}\cite{dugger_sheaves_1999}. His analogy to generators and relations is the one that clicked for me in this context.

Define \cart\ to be the full subcategory of \smfd\ containing just the $\rr^n$ for finite $n$ (recall that ``full'' means include every morphism). In the case of $\rr^4$ let's just bring the standard smooth structure. We can try to build the fake $\rr^4$s later on with surgery and such.

Due to the existence of real-valued functions such as $\frac{1}{1+e^{-x}}$ we can map all of \rr\ to the open interval $(0,1)$. So note that this category is not as pitifully empty as I, at least, thought at first. You can cover $\rr^n$ with balls each of which is a diffeomorphic image of $\rr^n$. We'll be doing just that.

To generalize \smfd\ we'll use \cart\ in a new way. In the original atlas-based definition of a smooth manifold we were characterizing manifolds by the slogan ``locally isomorphic to $\rr^n$.'' This is a way to cash out the more general concept ``modeled on $\rr^n$.'' But we will now shift to the paradigm ``can be probed by $\rr^n$'' or equivalently ``has specified maps from the test spaces $\rr^n$.'' In this framework we will specify probes from all dimensions of cartesian spaces, not just the one that has the same dimension as the manifold (if any such concept as dimension survives our shenanigans).

A \emph{presheaf} on \cart\ is a functor $$\cart^{\mathrm{op}}\to \mathrm{Set}$$ Think of one presheaf $M$ (suggestive name) as a would-be space together with the specification of what probing functions are smooth. Sometimes these probes are called \emph{plots}. These are different from charts. Charts are specific homeomorphisms that cover $M$. Here we must provide the entire set of all possible smooth functions into $M$ from $\rr^n$, for all $n$.

The category \pshcart\ consists of all such functors, together with the natural transformations between them. This is an interesting category! If $C$ is any locally small category, there is an embedding (a full and faithful functor that is injective on objects) $C\to\psh{C}$ called the \emph{Yoneda embedding}. In fact \psh{C} is the \emph{free cocompletion} of $C$, meaning it freely adds all colimits from $C$. In fact \psh{C} is a \emph{topos} which is a checklist of wonderful properties, and we'll be using them all by the end because HoTT and toposes are closely related. In a nutshell, \psh{C} inherits these nice properties from $\mathsf{Set}$.

BUT.

If $C$ already has some colimits, \psh{C} will not respect these. It is \emph{too free}. 

Consider two subobjects of $\rr^n$ in \cart, say two disjoint open balls $U$ and $V$. \gc{complete the example}

Following Duggers we think of sheaves as the subcategory of presheaves that preserve the existing colimits that we had. There is an analogy to specifying a group as the quotient of a free group by a subgroup generated by some relations.

It's illustrative to understand the layers of exoticness that are introduced by moving from \smfd\ to sheaves. Manifolds embed in diffeological spaces which are the concrete sheaves (see Baez and Hoffnung \cite{baez_convenient_2008}).\gc{give these subcategories}

\section{The algebra of functions}\label{sec:algebras}

For a smooth manifold $M$ the set of smooth real-valued maps $f:M\to\rr$ has the structure of a vector space, a ring, and in fact a commutative algebra. An algebra is a vector space over a base field (hence it has a commutative addition operation) which also has a multiplication operation that satisfies the distributive law when it interacts with addition. If the multiplication is commutative then we say the algebra is commutative. In this case the operations are all pointwise and so the algebra properties are inherited from \rr. For example $fg(x)=f(x)g(x)$ where the right hand side is multiplication in \rr.

\begin{myprop}\label{prop:algebrafunctor} Taking the algebra $\cinfty(M)$ of smooth functions of a smooth manifold $M$ gives a contravariant functor $\smfd\xrightarrow[]{\cinfty}\calg^{\mathrm{op}}$.
\end{myprop}

There are important subcategories of $\calg$ that themselves contain the image of this functor, notably the ``smooth algebras'' but we won't make use of their special properties, so we might as well use the simpler larger category.

Crucially this functor is actually \emph{faithful}, meaning if you fix two manifolds then the functor is injective on the hom set between them. Said another way, any algebra homomorphism $\cinfty(N)\to\cinfty(M)$ comes from a unique smooth map $M\to N$. Therefore we can embed \smfd\ into $\calg$ and treat its image as an equivalent copy of \smfd\ (with arrows reversed) and find new ways to extend it and build new categories!

The proof of faithfulness is interesting and we will explore it fully because it's challenging to piece it together from the references that exist easily to hand today. Since we have a couple other such ``smooth facts'' to make use of let's handle them together.

\section{The lemmas about smoothness}\label{sec:smoothnesslemmas}

\begin{mylemma}\label{lemma:smoothfact1}(``Milnor's Exercise'') The map $M\xrightarrow[]{\mathrm{ev}}\Hom(\cinfty(M),\rr)=\Hom(\cinfty(M),\cinfty(\rr^0))$ given by evaluation at a point of $M$ is a bijection.
\end{mylemma}
\begin{proof}
See also Kolář, Michor and Slovák \cite{kolar_natural_1993}, especially the preamble before section 35 and Corollary 35.9. Let $\phi\in\Hom(\cinfty(M), \rr)$. Then $\ker(\phi)$ is an ideal of codimension 1. (An ideal of an algebra is a vector subspace which is closed under multiplication by an arbitrary element of the algebra.) Consider the collection of zero sets $Z_f, f\in \ker(\phi)$. This collection of sets is closed under intersection: given $Z_f, Z_g$ then $Z_{(f^2+g^2)} = Z_f\cup Z_g$ and $f^2+g^2\in\ker(\phi)$. We have satisfied some of the hypotheses of the following topological lemma: 

\begin{mylemma}If $X$ is Hausdorff and $C\subseteq X$ compact, and $F=\{C_i\}$ any collection of closed subsets of $C$ with the property that $C_i, C_j\in F\implies C_i\cap C_j\in F$, then $\bigcap_i C_i \neq\emptyset$.\end{mylemma}

We have a collection of sets closed under intersection, so to satisfy the hypotheses of this new lemma it suffices to show that there is some $Z_f$ compact and nonempty, for then we can use this as the set $C$. 

Suppose for a moment that we had this $Z_f$. Let's see how we can finish proving Milnor's exercise. Suppose $\bigcap_{f\in\ker(\phi)}Z_f\neq\emptyset$, and so it contains some point $x$. Now consider an arbitrary $g\in\cinfty(M,\rr)$. Construct the function $g'=g-\phi(g)1$ where the latter term means the constant function with value $\phi(g)$. Then $g'\in\ker(\phi)$, so $g'(x)=0$, i.e. $g(x)-\phi(g)1(x)=0$ and hence $\phi(g)=g(x)$, and so $\phi=\mathrm{ev}_x$ as desired.

Now to construct the function $f$ whose zero set is compact and nonempty. Back in Definition~\ref{def:smoothmfd} we assumed all our smooth manifolds are Hausdorff with countable basis. This in turn implies paracompactness, a property that was put on this Earth to supply the existence of a \emph{partition of unity}, a countable collection of smooth bump functions, each with compact support, and with the collective property that in a neighborhood of any point only finitely many of them are nonzero, and they sum everywhere to the constant function 1. So let's fix a partition of unity $\{f_i\}_{i\in\nn}$. Consider the function $$g=f_1+2f_2+3f_1+\cdots.$$
This function is everywhere positive. We claim that for every $c>0$ that $U=g^{-1}([0,c])$ is compact. By continuity $U$ is closed. And it must lie in the union of finitely many supports, as the terms $nf_n$ grow beyond $c$ eventually. So $U$ is closed and contained in a finite union of compact sets, which is compact, and so $U$ is compact. One more trick: this argument works for $g^2$ as well, possibly with a different value of $c$. Since the codimention of $\ker(\phi)$ is 1, some linear combination $f=ag+bg^2=0, a, b\in\rr$. This $f$ carves out a compact nonempty zero set. \gc{what was the gap here at the end? That a, b might be negative? Surely one of them *is* negative?}
\end{proof}
It is perhaps notable that there is a larger class of topological manifolds for which Milnor's exercise holds, which goes by the name \emph{realcompact}, in case you want to look that up.

\begin{mylemma}\label{lemma:smoothfact2} For two smooth manifolds $M_1$ and $M_2$ the map
\begin{align}
\cinfty(M_1, M_2) &\to \Hom(\cinfty(M_2),\cinfty(M_1)) \\
f &\mapsto (g\mapsto g\circ f)
\end{align}
given by precomposition is a bijection, i.e. $\smfd\xrightarrow[]{\cinfty}\calg^{\mathrm{op}}$ is faithful.
\end{mylemma}
\begin{proof}
Given $\phi\in\Hom(\cinfty(M_2),\cinfty(M_1))$ and $x_1\in M_1$ we need to produce a point $x_2\in M_2$ and to show that this mapping $x_1\mapsto x_2$ is smooth. Consider $\mathrm{ev}_{x_1}\circ\phi\in\Hom(\cinfty(M_2),\rr)$: $$\cinfty(M_2)\xrightarrow[]{\phi}\cinfty(M_1)\xrightarrow[]{\mathrm{ev_{x_1}}} \rr.$$
By Lemma~\ref{lemma:smoothfact1} this is equal to evaluation at some point $x_2\in M_2$, so define $f:M_1\to M_2$ by $x_1\mapsto x_2$. We have shown that $\phi(g)=g\circ f$ and so $g\circ f$ is smooth for all $g$, which suffices to prove that $f$ is smooth.
\end{proof}

Note that Lemma~\ref{lemma:smoothfact2} is a generalization of Lemma~\ref{lemma:smoothfact1} but we used the more specific one to prove the more general one.

For our purposes these lemmas serve to allow us to pass to the image of $\cinfty$ in the category of algebras and proceed to enlarge it in new ways that are made possible by the algebraic setting. We will do that by extending the algebras with nilpotent elements!

\section{Nilpotents}\label{sec:nilpotents}

An element $x\in A$ of a real algebra is nilpotent if some power of it is zero. For example in the 4-dimensional algebra of 2x2 real matrices the element
\[
  M=\begin{pmatrix}
    0 & 1 \\
    0 & 0
  \end{pmatrix}
\]
satisfies $M^2=0$. A more important example for us is $\rr[x]/x^2$, which is the polynomial algebra over \rr\ modulo the ideal generated by $x^2$, which is equivalent to adding the relation $x^2=0$. This truncates the polynomials to have degree at most 1, so this is a 2-dimensional real algebra. (Contrast this with $\rr[x]/x^2+1$ which is also 2-dimensional but does not add any nilpotent elements; it in fact is isomorphic to \cc.) What if we add nilpotents to the story about algebras of smooth functions?

Now is a good time to expand our list of facts about smoothness with two more.

\begin{mylemma}(Hadamard)\label{lemma:hadamard} If $f:\rr^n\to\rr$ is smooth then in some open neighborhood $U\supseteq (0,\ldots,0)$
\[f(x_1,\ldots,x_n) = f(0) + \sum_{i=1}^n x_i g_i(x_1,\ldots,x_n)\]
for smooth functions $g_i$ satisfying
\[g(0,\ldots,0) = \left.\frac{\partial f}{\partial x_i}(x_1,\ldots,x_n)\right|_{(0,\ldots,0)}\]
\end{mylemma}
\begin{proof}Elementary special case of Taylor's theorem.\end{proof}

\begin{mylemma}\label{lemma:tangent}$\Hom(\cinfty(M),\rr[x]/x^2)$ is in bijection with derivations on $M$.
\end{mylemma}
\begin{proof}
\begin{align*}
f &\mapsto A(f) + B(f)x \\
fg &\mapsto (A(f) + B(f)x)(A(g)+B(g)x)&\quad \\
&= A(f)A(g) + (A(f)B(g) + B(f)A(g))x &\mathrm{so\ } A \mathrm{\ is\ itself\ an\ algebra\ homomorphism} \\
&= f(m)g(m) + (f(m)B(g) + B(f)g(m))x &\mathrm{for\ some\ } m\in M\mathrm{\ by\ Lemma~\ref{lemma:smoothfact1}}
\end{align*}
So $B$ is a derivation at $m$.
\end{proof}
\begin{mydef}Let $m\in M$ and let $U$ be a coordinate chart around $m$ with map $\phi:U\cong\rr^n$. A tangent vector $v_m$ at $m$ is an equivalence class of maps $\gamma:\rr\to M$ satisfying $\gamma(0)=m$ under the equivalence relation $\gamma_1\sim\gamma_2$ iff $\phi\circ\gamma_1$ and $\phi\circ\gamma_2$ have the same derivative at 0.\end{mydef}

\begin{mylemma}\label{lemma:derivationsaretangentvectors}For $m\in M$, derivations at $m$ are in bijection with tangent vectors at $m$.\end{mylemma}
\begin{proof}Hadamard's Lemma plus the definition of a vector field as $\sum_{i=1}^n v_i(x_1,\ldots,x_n)\frac{\partial}{\partial x_i}$.\end{proof}

Moral: in the dual world of spaces, this algebra with nilpotents is probing both the points of $M$ and its tangent vectors. It is a point with "nilpotent fuzz". It is an abstract tangent vector, just long enough to point in some direction.

Let me emphasize the choices we have available with nilpotents. If we were to probe with $\rr[x,y]/x^2,y^2$ we'd have an equivalence class of smooth maps $\rr^2\to M$ instead of tangent vectors. Meanwhile if we used a single variable but a higher power such as $\rr[x]/x^3$, the equivalence relation between curves would require agreement of the second derivative. We could carry through the rest of our discussion with any of these choices.

\section{Adding nilpotents to \cart}\label{sec:nilpotentstocartsp}
We define $\formalcart$ as $$\formalcart := \mathrm{opposite\ of\ }\left\{\cinfty)\rr^n)\otimes\rr\oplus\mathbb{V}, \mathbb{V}^k=0, k,n\in\nn\right\}\hookrightarrow \calg^{\mathrm{op}}$$

This dual description is the one we can get our hands on, but there is an intuitive picture we should keep in mind: think of the objects in this category as spaces $$\rr^n\times\dd\leftrightarrow\cinfty(\rr^n)\otimes\rr\oplus\mathbb{V}\mathrm{\ with\ }\mathbb{V}^m=0\mathrm{\ for\ some\ }m.$$

\dd is an "infinitesimal disk", a point with a nilpotent cloud or fuzz around it. That fuzz can probe the tangent directions of a manifold. In the dual algebra space it can take derivatives.

Why the tensor product? It's teh coproduct in the category of commutative algebras.

We will now make $\formalcart$ into a site so as to take sheaves over it. We define a Grothendieck pre-topology with covering families $$\left\{U_i\times\dd\xrightarrow[]{\phi_i\times\mathrm{id}}\rr^n\times\dd\right\}_i\mathrm{\ such\ that\ }\left\{U_i\xrightarrow[]{\phi_i}\rr^n\right\}_i$$
is a covering family in \cart. So the only way to cover the formal disk parts is via the identity.

We define $$\formalsmoothset:=\sh{\formalcart}.$$ These are spaces defined by a consistent declaration of all teh smooth maps from $\rr^n\times\dd.$

\section{Functors and Kan extension}\label{sec:functors}
First we'll look at the functors in \formalcart, and then extend them to sheaves with Kan extension.

Projection:
\begin{align*}
pr:\formalcart &\to\cart \\
\rr^n\times\dd &\mapsto \rr^n \\
\cinfty(\rrn)\otimes\rrx &\mapsto \cinfty(\rrn)\otimes\rr \\
\end{align*}

Inclusion:

\begin{align*}
i:\cart &\hookrightarrow\formalcart \\
\rrn &\mapsto \rrn\times\dd_0 \\
\cinfty(\rrn)\otimes\rr &\mapsto\cinfty(\rrn)\otimes\rrx \\
\end{align*}

Tangent:

\begin{align*}
T:\formalcart &\hookrightarrow\formalcart \\
\rrn\times\dd_1 &\mapsto \rrn\times\dd_1\times\dd_2 \\
\cinfty(\rrn)\otimes\rrx &\mapsto\cinfty(\rrn)\otimes\rrx\otimes\rry \\
\end{align*}

A tangent bundle $TM$ comes equipped with a projection map $p:TM\to M$ and a zero map $0:M\to TM$. In our setting $p$ and $0$ become natural transformations $p:T\to\id$ and $0:\id\to T$. We can see that $0=i\circ pr$ and $p=$.

\begin{myclaim}$i\adj p$ and so \cart\ is coreflective inside \formalcart.\end{myclaim}

\begin{proof}
\begin{align*}
\Hom_{\formalcart}(i(\rrn), \rrm\times\dd) &= \Hom_{\formalcart}(\rrn\times\dd_0, \rrm\times\dd) \\
&\cong \Hom_{\calg}(\cinfty(\rrm)\otimes\rr\oplus\vv, \cinfty(\rrn)) \\
&\cong \Hom_{\calg}(\cinfty(\rrm)\otimes\rr, \cinfty(\rrn))\quad\mathrm{(nilpotents\ must\ map\ to\ 0)} \\
&\cong \Hom_{\cart}(\rrn, \rrm) \\
&\cong \Hom_{\cart}(\rrn, p(\rrm\times\dd))
\end{align*}
\end{proof}
One must also show these are all natural in both slots.

Note the finding from line 2-3: maps from a nilpotent algebra to one with no nilpotents must be trivial on the nilpotents, i.e. \emph{a space without fuzz cannot probe fuzz}.

Functors, like $i$, induce three functors on presheaves. For an elementary introduction see Awodey \cite{awodey_introduction_2010}, especially Corollary~9.17.

$$
\begin{tikzcd}[row sep=huge, column sep=huge]
\smoothset \arrow[r, "\mkern-16mu i_!"{name=F}, bend left=25]
\arrow[r, "\quad i^*"{name=G}]
\arrow[phantom, from=F, to=G, "\ \dashv" rotate=-90]
& \formalsmoothset \arrow[l, "\mkern-12mu i_*"{name=H}, swap, bend left=25]
\arrow[phantom, from=G, to=H, "\ \dashv" rotate=-90] \\
\cart \arrow[u, "y"] \arrow[r, "i"{name=I}, shift left=2.0ex] 
& \formalcart \arrow[l, "p"{name=P}, shift left=2.0ex, swap] 
\arrow[u, "y", swap]
\arrow[phantom, from=I, to=P, "\dashv" rotate=-90]
\end{tikzcd}
$$

What do these extensions do?

$i^*$ is the precomposition functor. Let $fss\in\formalsmoothset$ and $\rrn\in\cart$. Then $$i^*(fss)(\rrn)=fss(i\rrn).$$ And on representables
\begin{align*}
\Hom_{\formalsmoothset}(y\rrm\times\dd, i_*y\rrn) &\cong \Hom_{\smoothset}(i^*y\rrm\times\dd, y\rrn) \\
&\cong \Hom_{\smoothset} (yp\rrm\times\dd, y\rrn)\quad(\mathrm{because\ }i^*y=yp)\\
&\cong \Hom_{\cart} (\rrm, \rrn)
\end{align*}

This is a hint that it is taking a space to one that cannot be probed by the infinitesials.

We can package all of these considerations into endofunctors on just FSS:
\begin{align*}
\rmodal &= i_!\circ i^* \\
\imodal &= i_*\circ i^*
\end{align*}
On representables
\begin{align*}
\rmodal(y\rrm\times\dd) &= i_!i^*(y\rrm\times\dd) \\
&= i_!y\rrm \\
&= yi\rrm \\
&= y\rrm\times\dd_0
\end{align*}
and so "reduction" just projects away the nilpotent parts of a (representable) formal smooth set.

We also have $\rmodal\adj\imodal$:
\begin{align*}
\Hom_{\formalsmoothset}(U, \imodal M)&\cong\Hom_{\formalsmoothset}(U,i_* i^*M) \\
&\cong \Hom_{\smoothset}(i^*U, i^*M) \\
&\cong \Hom_{\formalsmoothset}(i_!i^*U, M) \\
&\cong \Hom_{\formalsmoothset}(\rmodal U, M)
\end{align*}
And so what if we start with $\Hom_{\formalcart}(\rrm\times\dd, \rrn\times\dd')$ and t hen hit the right hand space with $\imodal$?
\begin{align*}
\Hom_{\formalsmoothset}(y\rrm\times\dd, \imodal y\rrn\times\dd') &\cong \Hom_{\formalsmoothset}(\rmodal y\rrm\times\dd, y\rrn\times\dd') \\
&\cong \Hom_{\formalsmoothset}(y\rrm\times\dd_0, y\rrn\times\dd') \\
&\cong \Hom_{\formalcart}(\rrm\times\dd_0, \rrn\times\dd')
\end{align*}
and so a coreduced FSS is one whose probes, even with nilpotents, cannot see the tangent directions. Its tangent directions are collapsed or identified. 

“Infinitesimally close points are identified.”

So in spirit there is a relation “x and y are infinitesimally close” which connects with SDG. But now we can get at it with a modal operator.



\section{The modality}\label{sec:modality}
Differential cohesion is most gently introduced by Khavkine and Schreiber in \cite{khavkine_synthetic_2017} and plays a large role in Schreiber's magnum opus \cite{schreiber_differential_2013}.

\section{Towards connections}
To have connections we need both $TM$ and $TTM$. Eventually we need to define vector bundles $E$ as a slice category, and then we'll need $TE$. But in the case of $TTM$ how can we get there with an idempotent modality?

I think the key is to introduce a variable into the modal operator: $\imodal_x, \imodal_y$. At the algebra level these will come from $$\cinfty(\rrn)\otimes\rrx\otimes\rry = \cinfty(\rrn)\otimes\rr[x, y]/(x^2, y^2)$$. Each modal operator is idempotent. Adding the $x$ variable again doesn't provide more fuzziness, or more tangency, or more directions to differentiate in. But adding $y$ does.

It's crucial that the differentiation these variables are doing will commute with each other, so that it doesn't matter which one was tensored outermost.

Having the $x$ lets you differentiate everything, and adding $y$ lets you differentiate everything including the $x$ (i.e. there are now terms in $xy$).

If $\rrx$ leads to an interpretation of $\imodal_x$ then we can indeed have two of them.

I also want to add a functor $T$ with $TA=A[x]/x^2$ which I hope is equal to $A\otimes_{\rr}\rrx$. Perhaps the unit of this is the zero section. Then we would have two functors, $I$ and $T$, and the units are $p$ and $i$. We'll need better names. $T$ and $I$ are inverses on the nose. The Kan extensions will convert between sets of probes that can see the tangent space and those that cannot.

When there are two modalities, the Kan extensions will move between sets of probes that can see 0, 1 or 2 iterated copies of the tangent space.

There is always a map $TTM\to T_2 M=TM\times_M TM$, where the rhs is the pullback of $TM$ by $p$, i.e. each fiber over $M$ is two copies of the tangent bundle.\gc{what about Lang's claim that $T_2 M$ is only a fiber bundle not a vector bundle?}

A connection is a section of this map, so $L:T_2 M\to TTM$. Given two tangent vectors you can lift that pair up to a vector in $TTM$, in fact a horizontal one. The image of the connection map is the horizontal sub-bundle of $TTM$. The kernel of the bundle projection is the vertical sub-bundle.

What types will we have? Apparently we'll have ones that "are $T$ of something" and ones that are not. If you are $T$ of something then you can be probed by nilpotents, and you are not coreduced. Else you are coreduced. To be $T$ of something you must be differentiable everywhere. Perhaps this is why $\sum_{a:A}B(a)$ is coreduced whenever $A$ is.

Infinitesimal closeness is preserved by morphisms. Equivalently, morphisms are bundle morphisms, they preserve the fibers. Is it also equivalent to stipulate that at the algebra level the morphisms hold $x$ fixed?

\bibliographystyle{unsrt}
\bibliography{smoothness} 

\end{document}