\documentclass[12pt]{article}
\usepackage{greg}

\title{Connections on a principal bundles}
\author{Greg Langmead}

\begin{document}
\maketitle
\begin{abstract}
These notes are intended as a didactic presentation of connections on principal bundles. I assume the reader knows some toplogy, such as the definition of compactness, and the preliminaries of differential geometry such as the definition of smooth manifolds and Lie groups. The goal is to define connections and Chern-Weil theory and then go through the paper by Freed and Hopkins \cite{freed2013chernweil} and finally sketch how we can import all of these ideas into type theory.
\end{abstract}
%\tableofcontents
\section{Introduction}
If you are interested in these notes but worry about whether you have the prerequisites, let me offer two of my favorite resources. First is the book by John Baez and Javier Muniain \cite{baez1994gauge} which is a compact and beautiful introduction to manifolds, connections, and the physics that uses these concepts. I'm still not sure why fate did not put this book in front of me in graduate school, when I was voraciously hunting on an almost daily basis for books just like it.

If, like me, you do even better with recorded lectures, I strongly recommend the playlist of lectures by Frederic Schuller \cite{schullerYoutube2015}. His course is efficient but very precise, and he is a great teacher who offers helpful intuition along the way. The scope is similar to the book by Baez and Muniain: he begins with topology and manifolds, defines connections and curvature, and then gets into physics.

A very complete treatment, with the most proofs, is found in Kobayashi and Nomizu's text \cite{kobayashinomizu}, Volume 1. Lastly, I'm cautiously optimistict that much of the material is treated clearly and cleanly in a more recent book by Hamilton \cite{hamilton2017}.

\section{Geometric intuition}
The 20th century method of capturing curvature is to imagine sliding a tangent vector around a loop on a curved surface.

\myfig{Alexandru Scorpan, ``The Wild World of Four-Manifolds'', 2015 \cite{scorpan_wild_2005}}{scorpan_sphere}

This is a picture of a specific embedding, which has selected a metric on S2, pulled back from R3. A metric is actually a little more structure than a connection. A connection captures this notion of transport without a full metric, but the picture is still useful.

A Rubik's Cube is a discrete version of the sphere, and a discrete version of curvature arises. You can cause pieces to rotate by sending them away by a certain path and bringing them back by another.

\section{Infinitesimal trivialization}
A fiber bundle is a quadruple $(E, p, M, F)$ such that
\begin{enumerate}
    \item $E, M, F$ are smooth manifolds
    \item $p:E\to M$ is a smooth surjective map
    \item $E$ is locally a product with the standard fiber $F$. i.e. there is an atlas of open sets $\{U_i\}$ on $M$ such that we have this commutative diagram
    \[\begin{tikzcd}
        {p^{-1}U_i} && {U_i\times F} \\
        & {U_i}
        \arrow["p", from=1-1, to=2-2]
        \arrow["{\phi_i}", from=1-1, to=1-3]
        \arrow["\cong"', from=1-1, to=1-3]
        \arrow["{\mathrm{pr}_1}"', from=1-3, to=2-2]
    \end{tikzcd}\]
	\item On intersections $U_{ij} = U_i \cap U_j$ the above data gives us a map $\phi_{ij}:U_{ij}\to\diffeo(F)$, satisfying the cocycle condition $\phi_{ij} = \phi_{jk}\circ\phi_{ij}$.
	\item A fast way to say what a vector bundle is is to say $F$ is a vector space and the descent maps factor through $GL(V)$.
\end{enumerate}

The setting for connections is $TE$, the tangent bundle of the total space. Since $E$ sort-of factorizes into $F$ and $M$, the tangent bundle sort-of factorizes into $TF$ and $TM$. 

Consider $Tp$, the derivative of the projection map. What is the kernel? It consists of vectors we can justifiably describe as ``vertical-pointing'', i.e. along the fiber. If that makes intuitive sense to you, you are well on your way to understanding connections.

Let's examine this short exact sequence:
\begin{equation}\label{seq}
    0\to VE\hookrightarrow TE \xrightarrow{Tp} TM\to 0\quad\mathrm{(on\ }E)
\end{equation}
where

A connection in a fiber bundle is a splitting of this sequence, a choice of decomposition at each point of $E$ of $TE$ as the direct sum of $VE$ and $TM$. That's our first waypoint.

We can talk about this splitting in three ways.

\begin{enumerate}
    \item A choice of complementary horizontal subspace of $TE$ at every point, i.e. a bundle $HE$ such that $TE=VE\oplus HE$. The horizontal subspace is isomorphic to $p^* TM$ by dimension-counting.
    
    \item A projection $TE\to VE$. The kernel of this operator is the horizontal subspace. This operator is in fact a 1-form because it takes as input tangent vectors (vector-valued, in fact with values in $VE$). We usually denote a connection 1-form with $\omega\in\Omega^1(E, VE)$. It's a 1-form \emph{upstairs}.
    
    \item A lift. Given $HE$ we can form a lifting map $C:TM\times_M E \to TE$ by using the isomorphism between $H_u E$ and $T_{m} M$. We have $\omega\circ C = 0$ because we lift vectors to horizontal ones.
\end{enumerate}

\myfig{From Rupert Way, ``Introduction to connections on principal fibre bundles'', 2010}{way_horiz_field}

Here's another good picture of the splitting.

\myfig{ibid.}{way_upstairs}

Connections exist if $M$ is nice enough. For example on a paracompact manifold we can 
\begin{enumerate}
	\item Create a smooth inner product using a partition of unity
	\item split $HE$ usign orthogonal complement
\end{enumerate}


\section{Transport}
We can parlay the lifting of tangent vectors into the lifting of curves. This is strictly 1-dimensional stuff!

Given a smooth curve $c:[0, 1]\to M$:
\begin{enumerate}
	\item Lift all the tangent vectors of $c$ upstairs, giving a horizontal vector field above $p^{-1}(c([0,1]))$
	\item Choose a point $u$ over $c(0)$.
	\item Flow along the vector field (requires a little proof that we can solve this differential equation -- straightforward in a local triv).
\end{enumerate}

\begin{tcolorbox}
Simple proof using curvature and connection pullbacks: Given a curve in $M$, say $c:(-1, 1)=:I \to M$ where $c(0)=:x$, we can pull back $E$ and its connection to $I$, called $c^*E$. The horizontal distribution must be integrable here since the Lie bracket vanishes on vectors pointing along the same linear subspace. So this connection is flat and we can form the vector field $X$ along $c$ in the base, lift $X$ to a horizontal vector field, then integrate it. Then we can push it back into $E$. This is the lift of $c$ to a horizontal curve (i.e. its tangent vectors are horizontal).
\end{tcolorbox}

Lifting curves provides a map between fibers: parallel transport!

\myfig{Alexandru Scorpan, ``The Wild World of Four-Manifolds'', 2005 \cite{scorpan_wild_2005}}{scorpan_transport}

Let \pathm\ be the set of piecewise-smooth curves $[0, 1]\to M$. Let $\pp(M, x, y)$ be paths from $x$ to $y$.

\pathm\ is a groupoid under concatenation of paths (and rescaling back into $[0, 1]$).

$$\transport: (\omega : \Omega_1(E; VE))\to (c: \pp(M,x,y)) \to E_{x} \to E_{y}$$

Some results
\begin{enumerate}
    \item Reversing the curve provides an inverse map. Transport gives an isomorphism on fibers. $$\transport (\omega, c_{\leftarrow}) = \lift (\omega, c_{\rightarrow})^{-1}$$
	\item Given $c:[0,1]\to M$ and $d:[0,1]\to M$ with $c(1) = d(0)$, we can compose the curves and form $c\cdot d$. We have $$\transport (\omega, c\cdot d) = \transport (\omega, d)\circ\transport (\omega, c).$$
    \item Reparameterizing the curve with any piecewise-smooth function $f:[0, 1]\to [0, 1]$ such that $f(0)=0$ and $f(1)=1$ preserves the holonomy map. So we can quotient \pathm\ by such reparameterizations. Baez and Schreiber use paths that slow down near the endpoints instead of piecewise-smooth and reparameterizations called thin homotopies, which sweep out no area (singular differential).
    \item In the case of a loop $f(0)=f(1)=x$, we carve out a subgroup of $\Aut(E_x)$ called the holonomy group, denoted $\Phi(x)$. $$\Phi(x)=\image(\lift(\omega, \{\mathrm{loops}\}) )\subset \Aut(E_x).$$
    \item Restricting to contractible loops gives the \emph{restricted holonomy} group denoted $\Phi_0(x)$. $$\Phi_0(x)=\image(\lift(\omega, \{\mathrm{contractible\ loops}\}) )\subset \Aut(E_x).$$
\end{enumerate}

If we require that holonomy preserve additional structure on $F$ such as a linear structure, or a metric, or a metric + orientation, then we prescribe a subgroup of $\Aut(F)$. Putting the group at the center of the discussion segues us to principal bundles.

\section{Curvature}
On to curvature.

In the general context of smooth manifolds, we have the Lie bracket of vector fields, which gives another vector field. The space of vector fields is hence a Lie algebra.

\myfig{Vector fields don't commute but the bracket is a new vector field}{lie_bracket_nonzero}

But given a sub-bundle like $VE$ or $HE$, will vector fields in that infinitesimal object be closed under Lie bracket?

If the sub-bundle can be integrated -- if it comes from a system of local sub-manifolds, i.e. a \emph{foliation}, then sure, because the vector fields live on a manifold where bracket is closed.

\myfig{Foliation of $E$ by $M$-dimensional slices}{foliation}

The implication goes the other way too -- closure under Lie bracket of all vector fields implies the distribution arises from a foliation. (Frobenius theorem, 19th c.)

For us the obstruction to integrability is the verticality of the bracket of two horizontal fields, and we have a formula for that! $$\omega([X, Y]), \mathrm{for\ }X, Y\mathrm{\ horizontal}$$

The function $\omega([\lift X, \lift Y])$ on pairs of tangent vectors in $TM$ is called the curvature.

If we have two horizontal vector fields, will their bracket be horizontal? If so, the horizontal bundle is integrable and you can slide around in there locally and stay horizontal. The connection is said to be \emph{flat}.

This is an entirely local condition, or obstruction.

Note that bundles are locally a product, but not locally a foliation. This is because the foliation is a coordinate-invariant notion. Even adding the connection, which is an infinitesimal product structure, may not integrate to a foliation.

Let's talk about how flatness relates to the fundamental groupoid.

More facts:
\begin{enumerate}
\item If $M$ is simply connected and the connection is flat, then $E$ is trivializable.
\item Ambrose-Singer Theorem: Define $E(u)$ upstairs as the image of $\transport(\omega, \{loops\})$ and $\Phi(u)$ as the corresponding image in $G$. Then the Lie algebra of $\Phi(u)$ is the image of curvature, all elements of the form $\omega([X, Y])$ for $X, Y\in T_uE$.
\item If $c$ is a small loop in $M$, $\lift c$ must live in one leaf of the foliation hence is a loop. Same for any contractible loop. Needs lemmas, such as 
\item Two curves respresenting the same class of $\pi_1(M, x)$ lift to the same holonomy element.
\item So $\transport$ factors through the fundamental groupoid. $$\transport: (\omega : \Omega_1(E; VE))\to (c: \Pi_1(M, x, y)) \to E_{x} \to E_{y}$$
\end{enumerate}


Formulas:

$$F(X, Y) = \omega(\lift(X), \lift(Y))$$
$$d\omega(X, Y) = -\frac{1}{2}[\omega(X), \omega(Y)] + F(X, Y)$$
$$dF(X, Y, Z) = 0 \mathrm{\ (if\ }X, Y, Z\mathrm{\ horizontal)}$$
$$R^*_g(\omega)$$

\section{In principal bundles}
A principal bundle is a fiber bundle where $F$ is a Lie group, plus more conditions:
\begin{enumerate}
    \item $G$ acts freely and transitively on $E$ on the right
	\item $M$ is the quotient of this action: $M=E/G$.
    \item The local trivializations respect the action: $\phi(e) = (m, g)\iff \phi(eh) = (m, gh)$.
\end{enumerate}

The fibers of $E$ are affine versions of $G$. They lack an identity point. The main example is the set of bases of a vector space. Choosing one basis and associating that to $e\in GL(V)$, we get a diffeomorphism between the bases and $GL(V)$ given by the transformation that maps the selected basis there.

$G$ adds a lot of structure to the proceedings.

There exists $\gamma:M\times\mathfrak{g}\cong VE$ because $TG\cong\mathfrak{g}$ and so each chart has trivialization of $VE$ that glues because the gluing is $G$-invariant.

$E\times G \to G$, and the orbit of $u\in E$ is a diffeo with $G$.

Given $X\in\mathfrak{g}$ we get a left-invariant vector field on $E_m$. This also gives the trivialization $VE\cong M\times\mathfrak{g}$.

The Maurer-Cartin form $\theta\in\Omega^w(G;\mathfrak{g})$ is given by $\theta(X)=X$ and then extended by left-translation. It sends a tangent vector to its translate at $e$.

Connections in this setting are required to satisfy:
\begin{enumerate}
	\item The distribution $HE$ is $G$-invariant.
	\item Therefore $\gamma^{-1}\omega(\gamma(m, X)) = X$. i.e. $\omega$ is the identity on $G$-invariant vertical vector fields.
\end{enumerate}

More results:
\begin{enumerate}
    \item $G$ acts on horizontal curves: composing a horizontal curve with the right group action produces another horizontal curve.
    \item Take $u\in E_x$ in the principal bundle $E$ and again consider all horizontal loops. This gives a subgroup of $G$, denoted similarly $\Phi(u)$ with restricted version $\Phi_0(u)$. The type of $u:E$ versus $x:M$ determines the type of the holonomy group we're referring to. $$\Phi(u) = \image(\transport(\omega)(\cdot)(u)).$$
    
	For example, imagine on the sphere if $P$ was the frame bundle. Imagine the connection is the one we implicitly used, which preserves length and angles. Then this image would in fact be contained in $O(2)$ instead of $GL(2)$.
    \item If $M$ is connected, all $\Phi_u$ are conjugate and hence isomorphic.
    \item If $M$ is connected and paracompact, $\Phi_0(u)$ is a connected subgroup of $G$ and a normal subgroup of $\Phi(u)$ and $\Phi(u)/\Phi_0(u)$ is countable. 
    \item If $M$ is connected and paracompact, we can form the subset $E(u)$ of all points joinable to $u$ by horizontal curves. This is a principal bundle with structure group $\Phi(u)$ and there is an induced connection. This is a reduction of the original bundle, licensed by the connection. In the earlier example this would form the $O(2)$ bundle.
    \item Ambrose-Singer Theorem: The Lie algebra of $\Phi_0(u)$ is the span of all elements of $\mathfrak{g}$ of the form $\Omega_v(X, Y)$ for $v\in P(u)$ with $X, Y$ horizontal. Slogan: curvature is infinitesimal holonomy.
\end{enumerate}

\section{Covariant derivative}
Precomposing with $\lift$ lets us define a souped-up differential operator called covariant derivative. We can differentiate sections of $E$ in the direction of tangent vector of $M$.
\section{Gauge transformations}
The action of an element $g\in G$ on $E$ is to rotate or move each fiber by the same amount.

There is a variable version: imagine choosing a different $g$ at each point, smoothly varying. Morally like a smooth map $M\to G$. These are called \emph{gauge transformations}. This is the group $\Aut(E)$, as any $G$-equivariant automorphism of $E$ that is the identity on $M$ is given by such a transformation.

Gauge transformations. A gauge transformation is a $G$-equivariant diffeomorphism $P\to P$ that is the identity on $M$. So these act on each fiber separately in a smooth way. You can also see it as a smooth choice of element of $G$ acting on each fiber, but the association with $G$ can only be chosen locally of course.

We can think of GTs as analogous to changes of coordinate chart. Physicists seek theories that are gauge-invariant, i.e. invariant under the group of gauge transformations. This is a giagintic infinite-dimensional space, whose type is morally similar to $M\to G$.

GTs act on connections. At a point they slide the horizontal subspaces up and down, and because they vary point to point, they may add some tilt.

GTs conjugate the curvature by a variable element of $G$. They cannot make a non-flat bundle flat.

The type of connections mod gauge is of paramount importance in physics and in fancy high-powered physics-derived invariants of $M$, such as in Donaldson theory on a 4-manifold, which starts with an $SU(2)$ bundle over $M$ and looks at $\mathcal{A}/\mathcal{G}$.




One last thing you can do with horizontal lifts of vectors is precompose lifting with exterior differentiation upstairs.

$$h:\Gamma(TM)\to \Gamma(HE)$$
$$d:\Omega^\bullet(T^*E)\to \Omega^{\bullet+1}(T^*E)$$
$$d\circ h: \Omega^\bullet(T^*M)\to \Omega^{\bullet+1}(T^*M)$$

$g^{-1}dg$ and such.

Affine. Lie algebra-valued 1-forms.

Classifying space.

Theorems from gauge theory!

Quantum gauge theory.

\section{Morphisms}
Given $f:P'(M', G') \to P(M, G)$ and a corresponding morphism $G'\to G$ and which is a diffeo on the base, we can push connections forward with the derivative of $f$.
% https://q.uiver.app/?q=WzAsOCxbMCwxLCJQJyJdLFswLDIsIk0nIl0sWzEsMSwiUCJdLFsxLDIsIk0iXSxbMCwwLCJHJyJdLFsxLDAsIkciXSxbMCwzLCJcXEdhbW1hJyJdLFsxLDMsIlxcR2FtbWEiXSxbMCwyXSxbMSwzLCJcXGNvbmciXSxbMCwxXSxbMiwzXSxbNCwwXSxbNSwyXSxbNCw1XSxbNiw3XV0=
\[\begin{tikzcd}
	{G'} & G \\
	{P'} & P \\
	{M'} & M \\
	{\Gamma'} & \Gamma
	\arrow[from=2-1, to=2-2]
	\arrow["\cong", from=3-1, to=3-2]
	\arrow[from=2-1, to=3-1]
	\arrow[from=2-2, to=3-2]
	\arrow[from=1-1, to=2-1]
	\arrow[from=1-2, to=2-2]
	\arrow[from=1-1, to=1-2]
	\arrow[from=4-1, to=4-2]
\end{tikzcd}\]

Given $f:P'(M', G') \to P(M, G)$ and a corresponding \emph{isomorphism} $G'\to G$ and which is a smooth map on the base, we can pull connections back by pulling back the 1-form.
% https://q.uiver.app/?q=WzAsOCxbMCwxLCJQJyJdLFswLDIsIk0nIl0sWzEsMSwiUCJdLFsxLDIsIk0iXSxbMCwwLCJHJyJdLFsxLDAsIkciXSxbMCwzLCJcXEdhbW1hJyJdLFsxLDMsIlxcR2FtbWEiXSxbMCwyXSxbMSwzXSxbMCwxXSxbMiwzXSxbNCwwXSxbNSwyXSxbNCw1LCJcXGNvbmciXSxbNyw2XV0=
\[\begin{tikzcd}
	{G'} & G \\
	{P'} & P \\
	{M'} & M \\
	{\Gamma'} & \Gamma
	\arrow[from=2-1, to=2-2]
	\arrow[from=3-1, to=3-2]
	\arrow[from=2-1, to=3-1]
	\arrow[from=2-2, to=3-2]
	\arrow[from=1-1, to=2-1]
	\arrow[from=1-2, to=2-2]
	\arrow["\cong", from=1-1, to=1-2]
	\arrow[from=4-2, to=4-1]
\end{tikzcd}\]


\section{In terms of downstairs bundles}
The sequence $$VE\to TE\to p^*TM$$ are bundles over $E$, but $M$ is the quotient $E/G$ so can all these structures be modded out by $G$ and become vector bundles directly on $M$? Yes!

First let's replace $VE$ with an isomorphic vector bundle that's constructed directly from a representation of $G$. Actually $VE$ is trivial, for a similar reason to the well-known fact that $TG$ is trivial for a Lie group.

Given any smooth right action of $G$ on any manifold, which we'll suggestively call $E$, we can differentiate it to get an infinitesimal version of the action. If the action is $$R:G\times E \to E$$ then differentiating with respect to the $G$ component gives $$TR:\mathfrak{g}\times E \to TE.$$ We can interpret this as follows: given an element of the Lie algebra $X:\mathfrak{g}$ and a point $u:E$, we get a vector $TR(X, u):T_u E$ by differentiating the $G$-action in the direction of $X$. In other words, take a smooth curve in $G$ through the identity, $c:(-1, 1)\to G$ such that $c(0)=e$ and $c'(0)=X$. Then examine the curve $R\circ c$ which is a curve in $E$ and take its derivative at 0 (with respect), yielding a tangent vector at $u$. This is the infinitesimal action of $X$ on $E$ at $u$. Then $TR(X, \cdot)$ is a vector field on $E$ which is an infinitesimal transformation of all of $E$ by $X$. This vector field is sometimes denoted $X^*$ and is known as the \emph{fundamental vector field} of $X$.

If we curry $R$ a little bit to get $R:G\to (E \to E),$ i.e. $$R:G\to \Aut(E)$$ then $TR:\mathfrak{g}\to T\Aut(E)=\Gamma TE$, meaning each Lie algebra element gives us an infinitesimal diffeomorphism of $E$ which is a vector field. This is another way of saying that the fundamental vector field of $X:\mathfrak{g}$ is an infinitesimal diffeomorphism of $E$.

We need to ask the question: what is $TR\circ TR$? How does $G$ act on fundamental vector fields? 
\begin{mylemma}
If $R:G\times E \to E$ is a smooth right action of $G$ on $E$, and if $X^*$ is the fundamental vector field of $X:\mathfrak{g}$, then $TR(g)(X^*)=(\ad(g^{-1})X)^*$.
\end{mylemma}
\begin{proof}
This is saying that pushing $X^*$ forward with $TR$ gives a new fundamental vector field that comes from a possibly different Lie algebra element determined by $\ad$. After all, $\ad$ is the derivative of the conjugation action of $G$ on itself, and conjugation is how we can move tangent structures from $e$ to some other element $g$: go back to $e$, build the structure, then move it back to $g$ (i.e., conjugate by $g$). In detail, if $c:(-1,1)\to G$ is a curve with $c(0)=e$ and $c'(0)=X$, and if $u:E$, then $X^*(u)=(TR\circ c)'(0)$. Therefore $TR(g)(X^*)$ maps $X^*(u)$ to a vector attached to $u\cdot g$ which is some translated point elsewhere in $E$. 
\end{proof}

Note that although $TR:\mathfrak{g}\times E\to TE$, the image lies in $VE$, and in fact $TR:\mathfrak{g}\times E \xrightarrow{\cong} VE$ because at each point $(X, u)$, the image of $TR$ exhausts the whole vertical vector space at $u$. So $TR$ is a trivialization of $VE$! But by the Lemma, this isomorphism transports the action of $G$ from $VE$ to the action on $E\times \mathfrak{g}$ given by $(u, X)\cdot g = (u\cdot g, ad(g^-1)(X))$. And so the quotient of this action is the associated bundle $E\times_G \mathfrak{g}$ known as the \emph{adjoint bundle} $\ad E$, which is a $\mathfrak{g}$-vector bundle associated to the adjoint representation of $G$ on $\mathfrak{g}$.

So we have a $G$-action and equivariant maps $$E\times \mathfrak{g}\xrightarrow{\cong} VE\to TE\to p^*TM$$ and we want to quotient this by the $G$-action. We will skip $VE$ and obtain the sequence $$\ad E\to TE/G \to TM$$ of vector bundles over $M$. This is known as the \emph{Atiyah sequence} and is defined in Atiyah-Bott (\cite{atiyah1983yang} p. 547). This remains an exact sequence, and since we defined a connection to be $G$-invariant, a connection is the same as a splitting of this downstairs sequence.

The total picture is thus
% https://q.uiver.app/?q=WzAsMTYsWzEsMSwiXFxtYXRocm17YWR9RSJdLFswLDEsIjAiXSxbMywxLCJURS9HIl0sWzUsMSwiVE0iXSxbNywxLCIwIl0sWzEsMCwiMCJdLFs3LDAsIjAiXSxbNiwwLCJwXiogVE0iXSxbNCwwLCJURSJdLFsyLDAsIkVcXHRpbWVzXFxtYXRoZnJha3tnfVxceHJpZ2h0YXJyb3dbXFxjb25nXXtUUn1WRSJdLFsxLDMsIk0iXSxbMywzLCJNIl0sWzUsMywiTSJdLFsyLDIsIkVcXHF1YWRcXHF1YWQiXSxbNCwyLCJFIl0sWzYsMiwiRSJdLFs1LDldLFs4LDddLFs3LDZdLFs5LDAsIi8gRyIsMl0sWzgsMiwiLyBHIiwyXSxbNywzLCIvIEciLDJdLFsxLDBdLFswLDJdLFsyLDNdLFszLDRdLFsyLDAsIlxcb21lZ2FfTSIsMix7ImxhYmVsX3Bvc2l0aW9uIjo0MCwiY3VydmUiOjEsInN0eWxlIjp7ImJvZHkiOnsibmFtZSI6ImRhc2hlZCJ9fX1dLFs5LDEzLCIiLDIseyJvZmZzZXQiOjR9XSxbOCwxNF0sWzcsMTVdLFswLDEwXSxbMiwxMV0sWzMsMTJdLFsxNCwxMSwicCJdLFsxNSwxMiwicCJdLFs5LDhdLFs4LDksIlxcb21lZ2FfRSIsMix7ImN1cnZlIjoxLCJzdHlsZSI6eyJib2R5Ijp7Im5hbWUiOiJkYXNoZWQifX19XSxbMTMsMTAsInAiLDAseyJvZmZzZXQiOjF9XV0=
\[\begin{tikzcd}
	& 0 & {E\times\mathfrak{g}\xrightarrow[\cong]{TR}VE} && TE && {p^* TM} & 0 \\
	0 & {\mathrm{ad}E} && {TE/G} && TM && 0 \\
	&& E\quad\quad && E && E \\
	& M && M && M
	\arrow[from=1-2, to=1-3]
	\arrow[from=1-5, to=1-7]
	\arrow[from=1-7, to=1-8]
	\arrow["{/ G}"', from=1-3, to=2-2]
	\arrow["{/ G}"', from=1-5, to=2-4]
	\arrow["{/ G}"', from=1-7, to=2-6]
	\arrow[from=2-1, to=2-2]
	\arrow[from=2-2, to=2-4]
	\arrow[from=2-4, to=2-6]
	\arrow[from=2-6, to=2-8]
	\arrow["{\omega_M}"'{pos=0.4}, curve={height=6pt}, dashed, from=2-4, to=2-2]
	\arrow[shift right=4, from=1-3, to=3-3]
	\arrow[from=1-5, to=3-5]
	\arrow[from=1-7, to=3-7]
	\arrow[from=2-2, to=4-2]
	\arrow[from=2-4, to=4-4]
	\arrow[from=2-6, to=4-6]
	\arrow["p", from=3-5, to=4-4]
	\arrow["p", from=3-7, to=4-6]
	\arrow[from=1-3, to=1-5]
	\arrow["{\omega_E}"', curve={height=6pt}, dashed, from=1-5, to=1-3]
	\arrow["p", shift right=1, from=3-3, to=4-2]
\end{tikzcd}\]

As we said, a connection gives a decomposition $TE/G=\ad E\oplus TM$. Consider two connections, say $\omega$ and $\omega'$. These are required to agree on vertical vectors by hypothesis, and so $\omega(a, b) = \omega'(a, b)$ which implies that $\omega-\omega'$ is independent of $a$ and so gives a function $TM\to \ad E$. Therefore $\omega-\omega':\Omega^1(M;\ad E)$. This is a concrete description of the space of connections: it's an infinite-dimensional affine space modeled on this space of 1-forms with values in $\ad E$.

So connections are not themselves directly 1-forms on $M$ due to the dependency on the vertical components of $TE/G$. But the curvature is directly a 2-form because it takes horizontal input. We can define curvature downstairs by $$F_\omega(X, Y) = \omega([\lift X, \lift Y])$$ for vector fields $X,Y:\Gamma TM$. 

In fact it's illuminating to look at sections of all the bundles in the sequence $$\Gamma\ad E\to \Gamma TE/G\to \Gamma TM$$ and note that this is an exact sequence of Lie algebras, and that nonvanishing curvature implies this sequence does \emph{not} split in general. Curvature measures the failure to split.

\section{Gluing, and local coordinates}
A sheaf is a functorial assignment of an object to each open subset of a manifold, such that if the assignment agrees on overlaps, then it can be extended to the union.

Do connections satisfy this property? What about equivalence classes of connections? What is the equivalence relation on connections that I mean when I ask that?

Consider a connection with zero curvature, and let's examine the gluing question. KN Theorem 9.1 states that a connection on $P(M, G)$ is flat iff the curvature vanishes identically. Let's prove the direction where the curvature is 0. Fix a point $x:M$ and a simply connected neighborhood $x:U$. Consider the induced connection on $P|U$. 

KN Theorem 4.2: Assume $M$ is connected and paracompact. Let $u:P$ and let $\Gamma$ be an arbitrary connection. The reduced holonomy $\Phi_0(u)$ is a connected Lie subgroup of $G$, $\Phi_0(u)$ is normal in $\Phi(u)$, and $\Phi(u)/\Phi_0(u)$ is countable. In particular, $\Phi(u)$ is a Lie subgroup of $G$ and $\Phi_0(u)$ is its identity component.

KN Theorem 8.1: Under the same hypotheses as Theorem 4.2, let $\Omega$ be the curvature of $\Gamma$. Let $P(u)$ be the holonomy subbundle through $u$. Then the Lie algebra of $\Phi(u)$ is the sub-Lie algebra of the Lie algebra of $G$ spanned by elements of the form $\Omega_v(X, Y)$ for $v:P(u)$, and $X, Y$ horizontal at $v$.

KN Theorem 7.1 (Reduction theorem): Under the same hypotheses as Theorem 8.1, $P(u)$ is a reduced bundle with structure group $\Phi(u)$, and the connection is reducible to a connection in $P(u)$.

KN Lemma during the proof of Theorem 7.1: Let $Q$ be a subset of $P(M, G)$ and $H$ a Lie subgroup of $G$. Assume: (1) the projection $p: P \to M$ maps $Q$ onto $M$; (2) $Q$ is stable by $H$, i.e., $R_a(Q) = Q$ for each $a\in H$; (3) if $u,v\in Q$ and $p(u) = p(v)$, then there is an element $a\in H$ such that $v = ua$; and (4) every point $x$ of $M$ has a neighborhood $U$ and a cross section $\sigma: U \to P$. such that $\sigma(U) \subset Q$. Then $Q(M, H)$ ts a reduced subbundle of $P(M, G)$.

The reduction theorem can be used to show that equivalence classes of connections do not form a sheaf. See Freed-Hopkins for an example where the base is $S^1$. In a local trivialization the connection is flat, hence the reduction theorem tells us that we can reduce it to a trivial connection. This reduction can be witnessed by a gauge transformation \gc{oops, why?} and so we can glue two representatives of a trivial connection to a nontrivial connection on all of $S^1$.

\section{Principal bundles in HoTT}
We adopt the HoTT perspective on higher groups as pointed connected types. The group $G$ is installed as the loop group at the base point of $BG$. This space $BG$ already has the properties of the classical classifying space of $G$.

Given a map $X\to BG$, how do we connect back to a locally trivial bundle structure?

\bibliography{connections}
\end{document}